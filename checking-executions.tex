%!TEX root = draft.tex
\section{Checking Linearizability of Priority Queue Executions}
\label{sec:checking inclusion by recursive procedure}

While checking linearizability of a given execution is known to be NP-complete in general~\cite{journals/siamcomp/GibbonsK97} we show in this section that it is polynomial-time for data-differentiated executions of the priority queue. For pedagogical reasons, Section~\ref{ssec:seq_exec} introduces a polynomial-time recursive procedure for checking whether a data-differentiated \emph{sequential} execution is admitted by the priority queue which is then extended to the concurrent case in Section~\ref{ssec:conc_exec}.

\subsection{Checking Data-Differentiated Sequential Executions}\label{ssec:seq_exec}

Checking whether a data-differentiated sequential execution belongs to $\seqPQ$ could be implemented by checking membership into the language accepted by $LTS??$. The recursive procedure $\textit{Check-PQ}$ outlined in Algorithm~\ref{alg:seq_check} is an alternative to this membership test. Roughly, it selects one or two operations in the input execution, checks whether their return values are correct by ignoring the order between the other operations other than how they are ordered w.r.t. the selected ones, and calls itself recursively on the execution without the selected operations. Since the $\textit{rm}(\textit{empty})$ operations are read-only (they don't affect the state of the queue), they are selected first. Ensuring that an operation $o=\textit{rm}(\textit{empty})$ is correct boils down to checking that every $\textit{put}(a,p)$ operation before $o$ is matched to a $\textit{rm}(a)$ operation which also occurs before $o$. Once there are no more $\textit{rm}(\textit{empty})$ operations, the procedure selects either a $\textit{put}$ operation adding a value with maximal priority which is not removed, or a pair of $\textit{put}$ and $\textit{rm}$ operations adding and removing the same maximal priority value. Their correctness can be checked using a similar argument as for $\textit{rm}(\textit{empty})$ operations. 

\begin{algorithm}[t]
\KwIn {A data-differentiated sequential execution $e$}
\KwOut{$\mathsf{true}$ iff $e\in \seqPQ$}

\If {$e = \epsilon$}
{\Return $\mathsf{true}$\;}

\If {$\mathsf{HasEmptyRemoves}(e)$}
{
    \If {$\exists\ o=\textit{rm}(\textit{empty})\in e$ such that $\mathsf{EmptyRemoveSeq}(e,o)$ holds}
    {
        \KwRet $\textit{Check-PQ}(e \setminus o)$\;
    }
    \Else {\KwRet $\mathsf{false}$\;}
}
\ElseIf{$\mathsf{HasUnmatchedMaxPriority}(e)$}
{
    \If {$\exists\ x \in \textit{values}(e)$ such that $\mathsf{UnmatchedMaxPrioritySeq}(e,x)$ holds}
    {
        \KwRet $\textit{Check-PQ}(e \setminus x)$\;
    }
    \Else {\KwRet $\mathsf{false}$\;}
}

\Else
{
    \If {$\exists\ x \in \textit{values}(e)$ such that $\mathsf{MatchedMaxPrioritySeq}(e,x)$ holds}
    {
        \KwRet $\textit{Check-PQ}(e \setminus x)$\;
    }
    \Else {\KwRet $\mathsf{false}$\;}
}
\caption{$\textit{Check-PQ}$}
\label{alg:seq_check}
\end{algorithm}


In formal terms, the operations which are selected depend on the following set of predicates on executions:
{\small
\begin{align*}
\mathsf{HasEmptyRemoves}(e)=\mathsf{true} & \mbox{ iff  $e$ contains a $\textit{rm}(\textit{empty})$ operation} \\
\mathsf{HasUnmatchedMaxPriority}(e)=\mathsf{true} & \mbox{ iff $p\in \textit{unmatched-priorities}(e)$ for a maximal priority} \\
&\hspace{4mm}\mbox{$p\in priorities(e)$}
\end{align*}}
where $\textit{priorities}(l)$, resp., $\textit{unmatched-priorities}(l)$, is the set of priorities occurring in $e$, resp., in $\textit{put}$ operations of $e$ for which there is no $\textit{rm}$ operation removing the same value. We call the latter \emph{unmatched} put operations. A put operation which is not unmatched is called \emph{matched}. For readability, we consider the following syntactic sugar $\mathsf{HasMatchedMaxPriority}(e)=\neg \mathsf{HasEmptyRemoves}(e)\land \neg \mathsf{HasUnmatchedMaxPriority}(e)$.

The predicates defining the correctness of the selected operations are defined as follows:
{\small
\begin{align*}
\mathsf{EmptyRemoveSeq}(e,o)=\mathsf{true} & \mbox{ iff  $e= u\cdot o\cdot v$ and $\textit{matched}(u)$} \\
\mathsf{UnmatchedMaxPrioritySeq}(e,x)=\mathsf{true} & \mbox{ iff  $e= u\cdot \textit{put}(x,p)\cdot v$, $p\not\prec \textit{priorities}(u\cdot v)$, and $p\not\in \textit{priorities}(v)$} \\
\mathsf{MatchedMaxPrioritySeq}(e,x)=\mathsf{true} & \mbox{ iff  $e= u\cdot \textit{put}(x,p)\cdot v\cdot \textit{rm}(x)\cdot w$, $p\not\prec \textit{priorities}(u\cdot v\cdot w)$,} \\
&\hspace{4mm}\mbox{$p\not\preceq \textit{unmatched-priorities}(u\cdot v\cdot w)$, $\textit{matched}_\prec(u\cdot v,p)$,} \\
&\hspace{4mm}\mbox{and $p\not\in \textit{priorities}(v\cdot w)$} 
\end{align*}
}

\vspace{-3mm}
\noindent
where $p\prec \textit{priorities}(e)$ when $p\prec p'$ for some $p'\in \textit{priorities}(e)$ (and similarly for $p\prec \textit{unmatched-priorities}(e)$ or $p\preceq \textit{unmatched-priorities}(e)$), 
$\textit{matched}_\prec(e,p)$ holds when each value with priority strictly smaller than $p$ is removed in $e$, and $\textit{matched}(e)$ holds when $\textit{matched}_\prec(e,p)$ holds for each $p\in\mathbb{P}$.

When $o$ is a $\textit{rm}(\textit{empty})$ operation, we write $e\setminus o$ for the maximal subsequence of $e$ which doesn't contain $o$. 


% and items of unmatched put in $l$, respectively. $p \prec \textit{priorities}(l)$ (resp., $p \preceq \textit{priorities}(l)$), if $p \prec p'$ (resp., $p \preceq p'$)for some $p' \in \textit{priorities}(l)$. Similarly we can define $p \preceq \textit{unmatched-priorities}(l)$.
%
%
%Essentially, this procedure chooses either a $\textit{rm}(\textit{empty})$ operation or a set of operations adding/removing a maximal priority in the input execution and checks whether they satisfy certain properties
%
%
% proceeds by 
%
% The projection $e \vert{O}$ of $e$ into $O$ is obtained from $e$ by erasing all call and return actions of non-$O$ operations. We write $e \setminus o$ for the projection $e \vert_{ O_e \setminus \{ o \} }$, where $O_e$ is the set of operations of $e$.
%
%Given a data-differentiated sequential execution $e$, it is costly to check whether $e$ belongs to priority queue, since for each element of $e$, we should keep the current content of priority queue and try to modify the priority queue according to this element. Such process is too complex. In this section, we propose another method for checking inclusion of priority queue by using a recursive procedure $\textit{check-PQ}$. For checking a sequence, every time we only check a much simpler property, and then recursively check the remanning sequences obtained by erasing one or two elements. Each step of our method is easier to deal with.
%
%$\textit{put}(a,p)$ matches $\textit{rm}(b)$, if $a = b$. To introduce $\textit{check-PQ}$, let us introduce several predicates. Given sequential execution $l$ and priority $p$:
%
%\begin{itemize}
%\setlength{\itemsep}{0.5pt}
%\item[-] $\textit{rm}(\textit{empty}) \notin l$ is satisfied when $l$ does not contain $\textit{rm}(\textit{empty})$.
%
%\item[-] Let $\textit{priorities}(l)$ and $\textit{unmatched-priorities}(l)$ be the set of priorities of putted items and items of unmatched put in $l$, respectively. $p \prec \textit{priorities}(l)$ (resp., $p \preceq \textit{priorities}(l)$), if $p \prec p'$ (resp., $p \preceq p'$)for some $p' \in \textit{priorities}(l)$. Similarly we can define $p \preceq \textit{unmatched-priorities}(l)$.
%
%\item[-] $\textit{matched}_{\prec}(l,p)$ is satisfied, if for each item of $l$ with priorities smaller than $p$, it has matched $\textit{put}$ and $\textit{rm}$. Similarly, we can define $\textit{matched}(l)$, where we consider all items in $l$ instead of items with certain priorities.
%\end{itemize}
%
%Let us introduce some notations: given sequential executions $u,v,w$ and priority $p$,
%
%\begin{itemize}
%\setlength{\itemsep}{0.5pt}
%\item[-] $\textit{PQ}_1(u,v,w,p) \equiv
%(\textit{rm}(\textit{empty}) \notin u \cdot v \cdot w) \wedge
%(p \not\prec \textit{priorities}(u \cdot v \cdot w)) \wedge
%(p \not\preceq \textit{unmatched-priorities}(u \cdot v \cdot w)) \wedge
%(\textit{matched}_{\prec}(u \cdot v,p) ) \wedge
%(p \notin \textit{priorities}(v \cdot w))$
%
%\item[-] $\textit{PQ}_2(u,v,p) \equiv
%(\textit{rm}(\textit{empty}) \notin u \cdot v) \wedge
%(p \not\prec \textit{priorities}(u \cdot v \cdot w)) \wedge
%(p \notin \textit{priorities}(v \cdot w))$
%
%\item[-] $\textit{PQ}_3(u,v) \equiv
%\textit{matched}(u \cdot v)$
%\end{itemize}
%
%
%Let us defined set $\textit{MS}(R)$ ($R \in \textit{PQ}_1,\textit{PQ}_2,\textit{PQ}_3$), which is the set of data-differentiated sequential executions that respect $R$.
%
%\begin{itemize}
%\setlength{\itemsep}{0.5pt}
%\item[-] $e \in \textit{MS}(\textit{PQ}_1)$, if $e = u \cdot \textit{put}(x,p) \cdot v \cdot \textit{rm}(x) \cdot w$ and $\textit{PQ}_1(u,v,w,p)$ holds for some item $x$ and a maximal priority $p$,
%
%\item[-] $e \in \textit{MS}(\textit{PQ}_2)$, if $e = u \cdot \textit{put}(x,p) \cdot v$ and $\textit{PQ}_2(u,v,p)$ holds for some item $x$ and a maximal priority $p$,
%
%\item[-] $e \in \textit{MS}(\textit{PQ}_3)$, if $e = u \cdot \textit{rm}(\textit{empty}) \cdot v$ and $\textit{PQ}_3(u,v)$ holds.
%\end{itemize}
%
%We call such $x$ and $\textit{rm}(\textit{empty})$ (if exists) the witness of $e$. Note that sequences in $\textit{MS}(R)$ does not guaranteed to be a priority queue execution. Let us introduce the notion $\textit{last}(e)$, which is a set and used as the branch condition of $\textit{check-PQ}$. Here $e$ is a data-differentiated sequential execution.
%
%\begin{itemize}
%\setlength{\itemsep}{0.5pt}
%\item[-] If $e$ contains $\textit{rm}(\textit{empty})$, then $\textit{last}(e) = \{ \textit{PQ}_3 \}$.
%
%\item[-] Else, if for a maximal priority $p$ of $e$, $\textit{unmatched-priorities}(e) \neq \emptyset$, then $\textit{PQ}_2 \in \textit{last}(e)$.
%
%\item[-] Else, if for a maximal priority $p$ of $e$, $\textit{unmatched-priorities}(e) = \emptyset$, then $\textit{PQ}_1 \in \textit{last}(e)$.
%\end{itemize}
%
%When $\textit{last}(e)$ contains only one element $R$, we write $\textit{last}(w)=R$ for simplicity.
%
%The projection $e \vert{\mathcal{D}}$ of $e$ into $\mathcal{D} \subseteq \mathbb{D}$ is obtained from $e$ by erasing all elements with a data value not in $\mathcal{D}$. We write $e \setminus x$ for the projection $e \vert_{ \mathbb{D} \setminus \{ x \} }$. The projection $e \vert{O}$ of $e$ into $O$ is obtained from $e$ by erasing all call and return actions of non-$O$ operations. We write $e \setminus o$ for the projection $e \vert_{ O_e \setminus \{ o \} }$, where $O_e$ is the set of operations of $e$.
%
%The recursive procedure $\textit{con-check-EPQ}$ is given as follows. Here $P_i(e,o)$ and $P_i(e,x)$ is the predicates of $e \in \textit{MS}(\textit{PQ}_i)$ withe witness $o$ or $x$, respectively.
%
%\begin{minipage}{.5\textwidth}
%\begin{algorithm}[H]
%\KwIn {a data-differentiated sequential execution $e$}
%
%\If {$e = \epsilon$}
%{\KwRet $\textit{true}$;}
%
%\If {$\textit{last}(e) = \textit{PQ}_3$}
%{
%    \If {$\exists o=\textit{rm}(\textit{empty})$, $P_3(e,o)$ holds}
%    {
%        \KwRet $\textit{check-PQ}(e \setminus o)$;
%    }
%    \KwRet $\textit{false}$;
%}
%
%\If {$\textit{PQ}_i \in \textit{last}(e)$ for $i \in \{1,2\}$}
%{
%    \If {$\exists l$, $P_i(e,x)$ holds}
%    {
%        \KwRet $\textit{check-PQ}(e \setminus x)$;
%    }
%    \KwRet $\textit{false}$;
%}
%\caption{$\textit{check-PQ}$}
%\label{Method-check-PQ}
%\end{algorithm}
%\end{minipage}
%\hfill


TODO MODIFY THE FOLLOWING EXAMPLE

Let us use an example to show how $\textit{check-PQ}$ works. Here we use $w_1 \xrightarrow{R} w_2$ to denote that in $\textit{check-PQ}(w_1)$, we choose the branch of $R$ and recursively call $\textit{check-PQ}(w_2)$.

\begin{example}\label{example:generate extended priority queue executions}
Given priorities $p_1,p_2,p_3$ with orders $p_1 \prec p_2$ and $p_1 \prec p_3$,

\noindent $\textit{put}(c,p_2) \cdot \textit{put}(a,p_1) \cdot \textit{rm}(a) \cdot \textit{rm}(c) \cdot \textit{rm}(\textit{empty}) \cdot \textit{put}(d,p_2) \cdot \textit{put}(e,p_3) \cdot \textit{rm}(e) \cdot \textit{put}(b,p_1)$

$\xrightarrow{\textit{EPQ}_3}$ $\textit{put}(c,p_2) \cdot \textit{put}(a,p_1) \cdot \textit{rm}(a) \cdot \textit{rm}(c) \cdot \textit{put}(d,p_2) \cdot \textit{put}(e,p_3) \cdot \textit{rm}(e) \cdot \textit{put}(b,p_1)$

$\xrightarrow{\textit{EPQ}_1}$ $\textit{put}(c,p_2) \cdot \textit{put}(a,p_1) \cdot \textit{rm}(a) \cdot \textit{rm}(c) \cdot \textit{put}(d,p_2) \cdot \textit{put}(b,p_1)$

$\xrightarrow{\textit{EPQ}_2}$ $\textit{put}(c,p_2) \cdot \textit{put}(a,p_1) \cdot \textit{rm}(a) \cdot \textit{rm}(c) \cdot \textit{put}(b,p_1)$

$\xrightarrow{\textit{EPQ}_1}$ $\textit{put}(a,p_1) \cdot \textit{rm}(a) \cdot \textit{put}(b,p_1)$

$\xrightarrow{\textit{EPQ}_2}$ $\textit{put}(a,p_1) \cdot \textit{rm}(a)$

$\xrightarrow{\textit{EPQ}_1}$ $\epsilon$
\end{example}

The following lemma states the correctness of $\textit{Check-PQ}$ (the proof can be found in Appendix~\ref{}).

\begin{restatable}{lemma}{EPQRulesAndSemantics}
\label{lemma:EPQ rules and semantics}
Algorithm~\ref{alg:seq_check} returns $\mathsf{true}$ on an execution $e$ iff $e\in \seqPQ$.
\end{restatable}

%
%\begin{lemma}
%Algorithm~\ref{alg:seq_check} returns $\mathsf{true}$ on an execution $e$ iff $e\in \seqPQ$.
%\end{lemma}


%Let $\textit{PQ}$ be set of sequences obtained by renaming some data-differentiated sequential execution that is accepted by $\textit{check-PQ}$. To persuade readers that $\textit{PQ}$ is indeed the set of behaviors of priority queue, in Appendix \ref{sec:appendix in section inductive rules of extended priority queue}, we give a semantical version definition $\textit{PQ}_s$ of priority queue, and shows that $\textit{PQ} = \textit{PQ}_s$. To model the possible behaviors of priority queue, we model it as an labelled transition system (shortened as LTS) $\textit{LTS}_e$. Each state of $\textit{LTS}_e$ is a function from $\mathbb{P}$ into sequences over $\mathbb{D}$, and represents a snapshot of contents of priority queue. $\textit{PQ}_s$ is the set of traces of $\textit{LTS}_e$. The definition of $\textit{PQ}_s$ and the proof of the following lemma can be found in Appendix \ref{sec:appendix in section inductive rules of extended priority queue}.
%
%\begin{restatable}{lemma}{EPQRulesAndSemantics}
%\label{lemma:EPQ rules and semantics}
%$\textit{PQ} = \textit{PQ}_s$.
%\end{restatable}
%
%By Example \ref{example:generate extended priority queue executions} we can see how to generate a sequence $e \in \textit{EP}$ from $\epsilon$ as follows:
%
%\begin{itemize}
%\setlength{\itemsep}{0.5pt}
%\item[-] First we add non-$\textit{rm}(\textit{empty})$ elements with a loop: Let $P$ be the set of priorities of items in $e$. In the first round of the loop, we choose a minimal priorities $p \in P$, we add items by using $\textit{PQ}_1$ and $\textit{PQ}_2$, let $P = P \setminus \{p\}$, and begins the next round of loop.
%
%\item[-] Then, we add $\textit{rm}(\textit{empty})$ by using $\textit{PQ}_3$.
%\end{itemize}

\subsection{Checking Data-Differentiated Concurrent Executions}\label{ssec:conc_exec}

Let us extend $\textit{check-PQ}$ with linearizability to deal with linearizability w.r.t $\textit{PQ}$ as follows: 

\begin{itemize}
\setlength{\itemsep}{0.5pt}
\item[-] The input $e$ is a concurrent execution. 

\item[-] $\textit{last}(e)$ is the union of $\textit{last}(u)$ for any sequential execution $u$ such that $e \sqsubseteq u$. $e \sqsubseteq \textit{MS}(R)$ with witness $x$ (resp., $o$), if $e \sqsubseteq u \in \textit{MS}(R)$ and $x$ (resp., $o$) is the witness of $u$. 
    
\item[-] $P_i(e,o)$ and $P_i(e,x)$ holds, if $e \sqsubseteq \textit{MS}(\textit{PQ}_i)$ withe witness $o$ or $x$, respectively. 
\end{itemize} 

The following lemma states that $\textit{check-PQ}$ is correct. 

\begin{restatable}{lemma}{ConCheckEPQIsCorrect}
\label{lemma:con-check-EPQ is correct}
Given a data-differentiated execution $e$, $e \sqsubseteq \textit{EPQ}$, if and only if, $\textit{check-PQ} = \textit{true}$.
\end{restatable} 

\begin {proof} (Sketch)
It is not hard to see that $\textit{check-PQ}(e) = \textit{false} \Rightarrow e \not\sqsubseteq \textit{PQ}$, since $\textit{PQ}_{\neq} \subset \textit{MS}(R)_{\neq}$. To ensure that $\textit{check-PQ}(e) = \textit{true} \Rightarrow e \sqsubseteq \textit{PQ}$, we require $\textit{PQ}$ to satisfy the following property.  

for any data-differentiated execution $e$,
\begin{itemize}
\setlength{\itemsep}{0.5pt}
\item[-] if $e \sqsubseteq \textit{MS}(R)$ ($R \in \{ \textit{PQ}_1, \textit{PQ}_2 \}$) with witness $x$, we have: $e \setminus x \sqsubseteq \textit{PQ} \Rightarrow e \sqsubseteq \textit{PQ}$.

\item[-] if $e \sqsubseteq \textit{MS}(\textit{PQ}_3)$ and $o$ is a $\textit{rm}(\textit{empty})$, we have: $e \setminus o \sqsubseteq \textit{PQ} \Rightarrow e \sqsubseteq \textit{PQ}$.
\end{itemize}

With this property, we can build a linearization of a whole execution by increasingly construct linearization of sub-execution from $\epsilon$, and this convince us that $\textit{check-PQ}(e) = \textit{true} \Rightarrow e \sqsubseteq \textit{PQ}$. 

Let us briefly explain the idea of proving this property for $\textit{PQ}_1$ with an example. Given a data-differentiated concurrent execution $e \sqsubseteq u \cdot \textit{put}(x,p) \cdot v \cdot \textit{rm}(x) \cdot w \in \textit{MS}(\textit{PQ}_1)$ with witness $x$ and assume that $e \setminus x \sqsubseteq l' \in \textit{PQ}$, as shown in \figurename~\ref{fig:concurrent execution for EPQ1} (a), we explicitly construct a sequence $l''= l''_1 \cdot \textit{put}(x,p) \cdot l''_2 \cdot \textit{rm}(x) \cdot l''_3$ and prove that $e \sqsubseteq l'' \in \textit{PQ}$. In this example, $p_1 \prec p$, $p_1 \prec p_2$, $u=\epsilon$, $w$ contains $\textit{rm}(z_2), \textit{put}(x_3,p_2), \textit{rm}(z_1)$ (emphasize by $\textit{rm}(z_2)-w$), and the remanning operations are in $v$. We also explicitly draw the linearization points according to $l'$. We construct $l''$ as follows, and this method does not rely on the locations of linearization points of $l'$ and operations of $x$. Here we implicitly mix operation $o$ (when $\textit{call}_o(m,a)$, $\textit{ret}_o(m,b)$ in $e$) with $m(a,b)$.

\begin{itemize}
\setlength{\itemsep}{0.5pt}
\item[-] Let $p$-comparable operations (resp., $p$-incomparable operations) be the set of operations with items whose priority is comparable with $p$ (resp., incomparable with $p$). It seems that we can construct $l''_1$, $l''_2$ and $l''_3$ as the projection of $l'$ into operations of $u$, $v$ and $w$, respectively. However, this is incorrect, since $\textit{PQ}_1$ has no restriction to $p$-incomparable operations operations in $u \cdot v$, and thus, there is no guarantee that the projection of $l'$ into $p$-incomparable operations operations in $u \cdot v$ being correct. In this example, such projection is $\textit{put}(z_1,p_3) \cdot \textit{rm}(z_2)$ and is incorrect.

\item[-] Let us construct sets $U'$, $V'$ and $W'$, such then make $l''_1$, $l''_2$ and $l''_3$ the projections of $l'$ into $U'$, $V'$ and $W'$, respectively. This contains two steps:

\item[-] The first step is to define $W'$. The $p$-comparable operations in $W'$ is same as that in $w$. To obtain $p$-incomparable operations in $W'$, we try to find a $p$-incomparable operation $o$ which either happens before some $p$-comparable operations in $w$, or with linearization points after $\textit{rm}(x)$. Then, we put $o$ and all $p$-incomparable operations after $o$ in $l'$ into $W'$ (emphasized by boxes in example). In this example, $o$ is $\textit{rm}(z_1)$, and $W'$ contains $\textit{put}(x_3,p)$, $\textit{rm}(z_1)$ and $\textit{rm}(z_2)$. We use boxes to emphasize they are put into $W'$.

\item[-] The second step is to define $U'$ and $V'$. $U'$ contains the following two kinds of operations: (1) Operations whose linearization points are before $\textit{ret}(\textit{put},x)$, and (2) other $\textit{put}$ operations with priority $p$. $V'$ contains the remanning operations. In this example, $U'$ contains $\textit{put}(z_1,p_3)$ and $\textit{put}(x_2,p_2)$.

\item[-] In this example, $l''_1 = \textit{put}(z_1,p_2) \cdot \textit{put}(x_2,p)$, $l''_2 = \textit{put}(z_2,p_2) \cdot \textit{rm}(x_2) \cdot \textit{put}(y_1,p_1) \cdot \textit{rm}(y_1)$, and $l''_3 = \textit{put}(x_3,p) \cdot \textit{rm}(z_1) \cdot \textit{rm}(z_2)$. In \figurename~\ref{fig:concurrent execution for EPQ1} (b), we add linarization points according to $l''$, and we can see that $l''$ holds as required.
\end{itemize}

This completes the proof of this lemma. 


\qed
\end {proof}



\begin{figure}[htbp]
  \centering
  \includegraphics[width=1 \textwidth]{figures/PIC-HIS-EPQ1-TwoHis.pdf}
%\vspace{-10pt}
  \caption{The process of obtaining linearization of $e$}
  \label{fig:concurrent execution for EPQ1}
\end{figure}




