\documentclass[a4paper,UKenglish]{lipics-v2016}

%\newtheorem{theorem}{Theorem}[section]
%\newtheorem{conjecture}[theorem]{Conjecture}
%\newtheorem{corollary}[theorem]{Corollary}
%\newtheorem{proposition}[theorem]{Proposition}
%\newtheorem{lemma}[theorem]{Lemma}
%\newdef{definition}[theorem]{Definition}
%\newdef{remark}[theorem]{Remark}
%\newdef{example}[theorem]{Example}



\usepackage{epsfig}
\usepackage{amsmath}
\usepackage{color}
\usepackage{amsfonts,amssymb}
%\usepackage{txfonts}
%\usepackage{pxfonts}
%\usepackage{mathabx}
%\usepackage{makeidx}  % allows for indexgeneration
\usepackage{verbatim}
%\usepackage{url}
%\usepackage{hyperref}
\usepackage{times}
\usepackage{ulem}    %(\emph{....} in the texfile) is underlined instead of italic
\normalem
\usepackage{enumerate}
%\usepackage[square, comma, sort&compress, numbers]{natbib}

%added by Yi Lv%
%\usepackage[linesnumbered,ruled,procnumbered,noend]{algorithm2e}
%\usepackage[linesnumbered,ruled,procnumbered,vlined]{algorithm2e}
\usepackage[linesnumbered,ruled,procnumbered,noend]{algorithm2e}
\usepackage{stmaryrd}
%\usepackage{hyperref}
%added by Wang Chao%
%\usepackage{mathrsfs}
%\usepackage{extarrows}
\usepackage{thmtools}
\usepackage{thm-restate}

%add for TIkz
\usepackage[version=0.96]{pgf}
\usepackage{tikz}
\usetikzlibrary{arrows,shapes,snakes,automata,backgrounds,petri}
%\usepackage[latin1]{inputenc}
%\usepackage{placeins}
%\usepackage[bookmarks=false]{hyperref}

%\declaretheorem[name=Claim]{clm}
%\declaretheorem[name=Lemma]{lem}
%\declaretheorem[name=Theorem]{thm}


\DeclareSymbolFont{largesymbolsA}{U}{txexa}{m}{n}
\SetSymbolFont{largesymbolsA}{bold}{U}{txexa}{bx}{n}
\DeclareFontSubstitution{U}{txexa}{m}{n}
\DeclareMathSymbol{\bigsqcupplus}{\mathop}{largesymbolsA}{"02}

% "Import" \bigsqcupplus from txfonts without loading txfonts (since
% it changes the default font and replaces many symbols)

% Commands by Catuscia

%\DeclareMathOperator*{\minarg}{{\rm minarg}}

%Previous notation for the probabilistic choice
%\newcommand{\probsum}{\bigoplus}
%\newcommand{\smallprobsum}[1]{\mkern4mu{\textstyle\circ\mkern-15.5mu\sum_{#1}\:}}
%\newcommand{\bigprobsum}[1]{{\;\displaystyle\odot\mkern-21mu\sum_{#1}}}
%\newcommand{\probsum}[1]{\mathchoice{\bigprobsum{#1}}{\smallprobsum{#1}}{\smallprobsum{#1}}{\smallprobsum{#1}}}
%For the uniform distribution
%\newcommand{\usmallprobsum}[1]{\mkern4mu{\textstyle\circ\mkern-15.5mu\sum^\mathcal{U}_{#1}\:}}
%\newcommand{\ubigprobsum}[1]{{\;\displaystyle\odot\mkern-21mu\sum^\mathcal{U}_{#1}}}
%\newcommand{\uprobsum}[1]{\mathchoice{\ubigprobsum{#1}}{\usmallprobsum{#1}}{\usmallprobsum{#1}}{\usmallprobsum{#1}}}

%\newcommand{\nondsum}{\bigbox}
\newcommand{\nondsum}{\bigsqcupplus}
\newcommand{\probplus}[1]{\oplus_{#1}}
%\newcommand{\nondplus}{\square}
\newcommand{\bang}{!\,}
\newcommand{\nondplus}{{\textstyle\bigsqcupplus}}
\newcommand{\partmap}{\rightharpoonup}
\newcommand{\map}{\rightarrow}
%\newcommand{\exc}{\phi}
\newcommand{\exc}{\alpha}
\newcommand{\exec}{\mathit{exec}}
\newcommand{\execp}{\mathit{execp}}
\newcommand{\Act}{\mathit{Act}}
\newcommand{\Sec}{\mathit{Sec}}
\newcommand{\Obs}{\mathit{Obs}}
\newcommand{\etree}{\mathit{etree}}
\newcommand{\lstate}{\mathit{lst}}
\newcommand{\fstate}{\mathit{fst}}
%\newcommand{\STATE}{\mathcal{P}r}  $ original marked
%\newcommand{\STATE}{\mathcal{P}}   $ marked by me
%\newcommand{\st}{P}
\newcommand{\trans}{\mathcal{T}}
\newcommand{\Aut}{\mathcal{M}}
\newcommand{\init}{\mathit{init}}
\newcommand{\perr}{\mathcal{P}}

\newcommand{\calo}{\mathcal{O}}
\newcommand{\cals}{\mathcal{S}}
\newcommand{\sseq}{\vec s}
\newcommand{\oseq}{\vec o}
\newcommand{\ccsp}{CCS$_p$}

\newcommand{\bigfrac}[2]{\frac{\raisebox{1ex}{$#1$}}{\raisebox{-1.5ex}{$#2$}}}
\newcommand{\nondarr}[1]{\overset{#1}{\longrightarrow}}
\newcommand{\Nondarr}[1]{\overset{#1}{\Longrightarrow}}
\newcommand{\vectorArrow}[1]{\stackrel{\longrightarrow}{\mbox{#1}}}
\newcommand{\probarr}[1]{\overset{#1}{\dashrightarrow}}
\newcommand{\paral}{\,|\,}
\newcommand{\outp}[1]{\overline{#1}}
\renewcommand{\Pr}{{\rm Pr}}

%Commands by Chao Wang

%memory models%
\newcommand{\TSO}{\textrm{TSO}}
\newcommand{\PSO}{\textrm{PSO}}

%correctness conditions%
\newcommand{\lin}{\textrm{linearizability}}
\newcommand{\slin}{\textrm{static linearizability}}
\newcommand{\qlin}{\textrm{quasi linearizability}}
\newcommand{\TTlin}{\textrm{TSO-to-TSO linearizability}}

\newcommand{\pair}[2]{\langle #1 , #2 \rangle}% pairs
\newcommand{\setof}[2]{\{ \, #1 \mid #2 \, \}}% Sets
\newcommand{\set}[1]{\{ {#1}  \}  }
%\newcommand{\map}[3]{{#1} \colon {#2} \longmapsto {#3}} %functions
\newcommand{\den}[1]{[\![#1]\!]}% Denotation of
\newcommand{\mean}[1]{|\!|#1|\!|}
\newcommand{\forget}[1]{}

%%%%%%%%%%%%%GENERAL%%%%%%%%%%%%%%%%%%%%%%%%%%%
%%%%%%%%%%%%%%%%%%%%%%%%%%%%%%%%%%%%%%%%%%%%%%%%%%%%%%%%%%%

\newcommand{\itbox}[1]{{\it #1\/}}
\newcommand{\un}[1]{\uline{#1}}%\underline{#1}}
\newcommand{\ov}[1]{\overline{#1}}
\newcommand{\smallspace}{\vspace{10mm}}
\newcommand{\is}{\mbox{$\Longleftarrow\ $}}
\newcommand{\pright}[1]{\hfill{#1}}
\newcommand{\bnfor}{\;\;\mid\;\;}

\newcommand{\ar}[1]{\stackrel{\scriptstyle #1}{\longrightarrow}}

%%%%%%%%%italics in math mode
%%%%%%%%%%%%%%%%%%%%%%%%%%%%%%%%%%%%%%%%%

%\newcommand{\true}{{\it true}}
%\newcommand{\false}{{\it false}}
\newcommand{\calB}{{\cal B}}
\newcommand{\calF}{{\cal F}}
\newcommand{\calP}{{\cal P}}
\newcommand{\order}{{\cal O}}
\newcommand{\size}[1]{|#1|}

%other notations%
\newcommand{\LTS}{\textit{LTS}}
\newcommand{\bedt}[1]{{\color{blue}#1}}
\newcommand{\redt}[1]{{\color{red}#1}}

%\pagestyle{empty}
%\pagestyle{plain}


%\def\lastname{Xu,Palamidessi}

\title{Checking Linearizability of Concurrent Priority Queues}




\author[1]{Ahmed Bouajjani}
%\author[2]{Michael Emmi}
\author[1]{Constantin Enea}
\author[1]{Chao Wang}
\affil[1]{Institut de Recherche en Informatique Fondamentale, \\ \texttt{\{abou,cenea,wangch\}@irif.fr}}
%\affil[2]{Nokia Bell Labs, \\ \texttt{michael.emmi@nokia.com}}



\begin{document}

\maketitle

\begin{abstract}
I am abstract
\end{abstract}

\forget{
\noindent Keywords: weak memory model, $\textit{linearizability}$,
$\textit{TSO-to-TSO linearizability}$
}



%!TEX root = draft.tex
\section{Introduction}
\label{sec:introduction}

Efficient implementations of concurrent objects are essential and hard to get right. Verifying them is difficult, checking a single execution is NP-complete~\cite{journals/siamcomp/GibbonsK97} and checking a finite-state implementation is undecidable~\cite{conf/esop/BouajjaniEEH13}.

The set of objects considered in previous work consists of stacks, queues, registers.



Here, we consider another important object, the priority queues, which are essential for applications such as task scheduling and discrete event simulation, and for which numerous implementations have been proposed, e.g.,~\cite{DBLP:conf/ppopp/AlistarhKLS15,DBLP:conf/wdag/CalciuMH14,DBLP:conf/opodis/LindenJ13,DBLP:conf/podc/ShavitZ99,DBLP:conf/ipps/ShavitL00}. This object is much less studied compared to the other ones in the verification literature.

Priority queues are collections and in that sense, they could be found similar to stacks and queues. The main difference which complicates the process of inventing verification techniques is that the order in which elements are removed is not fixed by the happens-before like in stacks and queues but by parameters, the priorities, which come from an unbounded domain. TODO an example. Say that for generality, we consider partially-ordered priorities and that for single priority we take FIFO. In java, same priority values are removed in an arbitrary order, however, this can be modeled using our partial orders. TODO SAY HOW

Building on previous work~\cite{DBLP:conf/icalp/BouajjaniEEH15}, we define decision procedures for priority queues. These procedures are designed in several steps:
1) defining recursive procedures recognizing valid sequences, and extending them to concurrent executions. These procedures take values one by one and check some property local to that value, ignoring how operations on other values are ordered between them. This works only for data-differentiated, but this is complete by data-independence.
2) checking whether a fixed value violates that property can be done using a specific class of register automata, TODO say how simple they are
3) we distinguish between that value being removed or not, and we have to also consider the case of remove(empty). This results in 3-4 automata describing all the possible violations.

Although we follow the same schema as in~\cite{DBLP:conf/icalp/BouajjaniEEH15}, these results require establishing key results which are specific to priority queues and which are not implied by those for stacks and queues (say that more details are given in the related work)

This gives a reduction to reachability that works for arbitrary implementations, and a decision procedure for finite-state implementations




\smallskip

\noindent {\bf Related work}
\noindent The most related work of our paper is ~\cite{DBLP:conf/icalp/BouajjaniEEH15}. Compared to ~\cite{DBLP:conf/icalp/BouajjaniEEH15}, our work are specific to priority queue and thus have points not occur in queues and stacks. Due to the partial-order priorities, to prove the correctness of recursive procedure, we need to alter the positions of linearization points of incomparable items while keeps the linearization points of comparable items unchanged. {\color {blue}In the proof process we use the notion of left-right constraint, which is inspired by left-right constraint of queue \cite{Bouajjani:2015}.} For checking violations of one local property, we just need to find a value, such that at every possible time point to locate its remove operation, there is already values with smaller priorities in priority queue. For checking violations of another local property, we just need to find two values with maximal priority, such that every possible time point to locate remove operation of one value is disabled by remove operation of another value. Such phenomenons are unique for priority queue.





%!TEX root = draft.tex
\newcommand{\seqPQ}{\mathsf{SeqPQ}}

\section{The Priority Queue ADT}
\label{sec:priority queue and data-independence}

We consider priority queues whose interface contains two methods $\textit{put}$ and $\textit{rm}$ for adding and respectively, removing a value. Each value is assigned with a priority when being added to the data structure (by calling $\textit{put}$) and the remove method $\textit{rm}$ removes a value with a minimal priority. For generality, we assume that the set of priorities is partially-ordered. Incomparable priorities can be removed in any order. When multiple values are assigned with the same priority, $\textit{rm}$ returns the least recent value. Also, when the set of values stored in the priority queue is empty, $\textit{rm}$ returns the distinguished value $\textit{empty}$. In this section, we formalize (concurrent) executions and implementations, introduce a set of properties satisfied by all the implementations we are aware of, and recall the standard correctness criterion for concurrent implementations of ADTs known as \emph{linearizability}~\cite{journals/toplas/HerlihyW90}.

\subsection{Executions}\label{ssec:exec}

We fix a (possibly infinite) set $\mathbb{D}$ of data values, a (possibly infinite) set $\mathbb{P}$ of priorities, a partial order $\prec$ among elements in $\mathbb{P}$, and an infinite set $\mathbb{O}$ of operation identifiers.
The latter are used to match call and return actions of the same invocation. Call actions $\textit{call}_o(\textit{put},a,p)$ and $\textit{call}_o(\textit{rm},a')$ with $a\in \mathbb{D}$, $a'\in \mathbb{D}\cup\{\textit{empty}\}$, $p \in \mathbb{P}$, and $o \in \mathbb{O}$, combine a method name and a set of arguments with an operation identifier. The return value of a remove is transformed to an argument value for uniformity~\footnote{Method return values are guessed nondeterministically, and validated at return points.
This can be handled using {\tt assume} statements, which only admit executions satisfying a given predicate.}.
The return actions are denoted in a similar way as $\textit{ret}_o(\textit{put},a,p)$ and respectively, $\textit{ret}_o(\textit{rm},a')$.

An \emph{execution} $e$ is a sequence of call and return actions which satisfy the following well-formedness properties: each return is preceded by a matching call (having the same operation identifier), and each operation identifier is used in at most one call/return. We assume every set of executions is closed under isomorphic renaming of operation identifiers. An $m(a)$-operation in an execution $e$ is an operation identifier $o$ s.t. $e$ contains the actions $\textit{call}_o(m,a)$ and $\textit{ret}_o(m,a)$.
An execution is called \emph{sequential} when no two operations overlap, i.e., each call action is immediately followed by its matching return action, and \emph{concurrent} otherwise. For readability, we write a sequential execution as a sequence of $\textit{put}(a,p)$ and $\textit{rm}(a)$ symbols representing a pair of actions $\textit{call}_o(\textit{put},a,p)\cdot \textit{ret}_o(\textit{put},a,p)$ and $\textit{call}_o(\textit{rm},a)\cdot \textit{ret}_o(\textit{rm},a)$, respectively ($o\in\mathbb{O}$). For example, given two priorities $p_1 \prec p_2$, $\textit{put}(a,p_2) \cdot \textit{put}(b,p_1) \cdot \textit{rm}(b)$ is a sequential execution of the priority queue ($\textit{rm}$ returns $b$ because it has smaller priority).

We define $\seqPQ$, the set of sequential priority queue executions, semantically via a labelled transition system (LTS, for short). An LTS is a tuple $A=(Q,\Sigma,\rightarrow,q_0)$, where $Q$ is a set of states, $\Sigma$ is an alphabet of transition labels, $\rightarrow\subseteq Q\times\Sigma\times Q$ is a transition relation, and $q_0$ is the initial state. We model the priority queue as an LTS $\textit{PQ}$ where states are mappings associating priorities in $\mathbb{P}$ with sequences of values in $\mathbb{D}$, representing a snapshot of the priority queue (for each priority, the values are ordered as they were inserted), and the transition labels are $\textit{put}(a,p)$ and $\textit{rm}(a)$. Each transition modifies the state as expected. For example, $q_1 \xrightarrow{\textit{rm}(\textit{empty})} q_2$ if $q_1 = q_2$, and $q_1$ and $q_2$ map each priority to the empty sequence $\epsilon$. Then, $\seqPQ$ is the set of traces (words) accepted by $\textit{PQ}$. The detailed definition of $\textit{PQ}$ can be found in Appendix \ref{sec:appendix definition of seqPQ and proof of Lemma EQP rules and semantics}.


An implementation $\mathcal{I}$ is a set of executions. Implementations represent libraries whose methods are called by external programs. In the remainder of this work, we consider only \emph{completed} executions, where each call action has a corresponding return action. This simplification is sound when the method invocations can always make progress in isolation.









\subsection{Semantic Properties of Priority Queues}\label{ssec:semantic_prop}

We define two properties which are
important for our results: (1) \emph{data independence}~\cite{conf/popl/Wolper86,conf/tacas/AbdullaHHJR13} states that priority queue behaviors do not depend on the actual values which are added to the queue, and (2) \emph{closure under projection}~\cite{DBLP:conf/icalp/BouajjaniEEH15} states that executions remain valid by removing all the operations adding or removing certain values.

An execution $e$ is \emph{data-differentiated} if every value is added at most once, i.e., for each $d \in \mathbb{D}$, $e$ contains at most one action $\textit{call}_o(\textit{put},d,p)$ with $o\in\mathbb{O}$ and $p\in \mathbb{P}$. Note that this property concerns only values, a data-differentiated execution $e$ may contain more than one value with the same priority. The subset of data-differentiated executions of a set of executions $E$ is denoted by $E_{\neq}$.

A renaming function $r$ is a function from $\mathbb{D}$ to $\mathbb{D}$. Given an execution $e$, we denote by $r(e)$ the execution obtained from $e$ by replacing every data value $x$ by $r(x)$. Note that $r$ renames only the values and keeps the priorities unchanged. Intuitively, renaming values has no influence on the behavior of the priority queue, contrary to renaming priorities.

\begin{definition}\label{def:priority-value data-independence}
A set of executions $E$ is \emph{data independent} iff
\begin{itemize}
\setlength{\itemsep}{0.5pt}
\item[-] for all $e \in E$, there exists $e' \in E_{\neq}$ and a renaming function $r$, such that $e=r(e')$,

\item[-] for all $e \in E$ and for all renamings $r$, $r(e) \in E$.
\end{itemize}
\end{definition}

The following lemma is a direct consequence of definitions.

\begin{lemma}
$\seqPQ$ is data independent.
\end{lemma}

Beyond sequential executions, every concurrent priority queue implementation that we are aware of is data-independent. From now on, we consider only data-independent implementations. This assumption enables a reduction from checking the correctness of an implementation $\mathcal{I}$ to checking the correctness of its data-differentiated executions in $\mathcal{I}_{\neq}$.

Besides data independence, the sequential executions of the priority queue satisfy the following closure property: an execution remains valid when removing all the operations with an argument in some set of values $D \subseteq \mathbb{D}$ and any $\textit{rm}(\textit{empty})$ operation (since they are read-only and they don't affect the queue's state).
To distinguish between different $\textit{rm}(\textit{empty})$ operations while simplifying the technical exposition, we assume that they receive as argument a value, i.e., call actions are of the form $\textit{call}_o(\textit{rm},\textit{empty},a)$ for some $a\in \mathbb{D}$. We will make explicit this argument only when needed in our technical development. The projection $e \vert D$ of an execution $e$ to a set of values $D \subseteq \mathbb{D}$ is obtained from $e$ by erasing all the call/return actions with an argument not in $D$. We write $e \setminus x$ for the projection $e \vert_{ \mathbb{D} \setminus \{ x \} }$. Let $\textit{proj}(e)$ be the set of all projections of $e$ to a set of values $D \subseteq \mathbb{D}$. 

The proof of the following lemma can be found in Appendix \ref{sec:appendix proofs in section priority queue and data-independence}.

\begin{lemma}
\label{lem:closure_proj}
$\seqPQ$ is closed under projection, i.e., $\textit{proj}(e)\subseteq \seqPQ$ for each $e\in \seqPQ$.
\end{lemma}


\subsection{Linearizability}\label{ssec:lin}

We recall the notion of \emph{linearizability}~\cite{journals/toplas/HerlihyW90} which is the \emph{de facto} standard correctness condition for concurrent data structures.
Given an execution $e$, the happen-before relation $<_{\textit{hb}}$ between operations~\footnote{In general, we refer to operations using their identifiers.} is defined as follows: $o_1 <_{\textit{hb}} o_2$, if the return action of $o_1$ occurs before the call action of $o_2$ in $e$. The happens-before relation is an interval order \cite{DBLP:conf/popl/BouajjaniEEH15}: for distinct $o_1,o_2,o_3,o_4$, if $o_1 <_{\textit{hb}} o_2$ and $o_3 <_{\textit{hb}} o_4$, then either $o_1 <_{\textit{hb}} o_4$, or $o_3 <_{\textit{hb}} o_2$. Intuitively, this comes from the fact that concurrent threads share a notion of global time.

Given a (concurrent) execution $e$ and a sequential execution $s$, we say that $e$ is linearizable w.r.t $s$, denoted $e \sqsubseteq s$, if there is a bijection $f: O_1 \rightarrow O_2$, where $O_1$ and $O_2$ are the set of operations of $e$ and $s$, respectively, such that (1) the call and return actions with identifier $o$ and $f(o)$, respectively, are the same
and (2) if $o_1 <_{\textit{hb}} o_2$, then $f(o_1) <_{\textit{hb}} f(o_2)$.
A (concurrent) execution $e$ is linearizable w.r.t. a set $S$ of sequential executions, denoted $e \sqsubseteq S$, if there exists $s \in S$ such that $e \sqsubseteq s$. A set of concurrent executions $E$ is linearizable w.r.t. $S$, denoted $E \sqsubseteq S$, if $e \sqsubseteq S$ for all $e \in E$.

The following lemma states that by data-independence, it is enough to consider only data-differentiated executions when checking linearizability (see Appendix \ref{sec:appendix proofs in section priority queue and data-independence}).
Section~\ref{sec:checking inclusion by recursive procedure} will focus on characterizing linearizability for data-differentiated executions.

\begin{lemma}\label{lemma:data differentiated is enough for PQ}
A data-independent implementation $\mathcal{I}$ is linearizable w.r.t. a data-independent set $S$ of sequential executions, if and only if $\mathcal{I}_{\neq}$ is linearizable w.r.t. $S_{\neq}$.
\end{lemma}














%!TEX root = draft.tex
\section{Checking Linearizability of Priority Queue Executions}
\label{sec:checking inclusion by recursive procedure}

We define a recursive procedure for checking linearizability of a data-differentiated execution w.r.t. $\seqPQ$.
To ease the exposition, Section~\ref{ssec:seq_exec} introduces a recursive procedure for checking whether a data-differentiated \emph{sequential} execution is admitted by the priority queue which is then extended to the concurrent case in Section~\ref{ssec:conc_exec}.

\subsection{Characterizing Data-Differentiated Sequential Executions}\label{ssec:seq_exec}

The recursive procedure $\textit{Check-PQ-Seq}$ outlined in Algorithm~\ref{alg:seq_check} checks whether a data-differentiated sequential execution belongs to $\seqPQ$ (i.e., if it is accepted by the LTS $PQ$).
 Roughly, it selects one or two operations in the input execution, checks whether their return values are correct by ignoring the order between the other operations other than how they are ordered w.r.t. the selected ones, and calls itself recursively on the execution without the selected operations.

We explain how the procedure works on the following execution:
\begin{align}
\hspace{-5mm}\textit{put}(c,p_2)\hspace{-.5mm} \cdot \textit{put}(a,p_1) \cdot \textit{rm}(a) \cdot \textit{rm}(c) \cdot \textit{rm}(\textit{empty}) \cdot \textit{put}(d,p_2) \cdot \textit{put}(\mathit{f},p_3) \cdot \textit{rm}(\mathit{f}) \cdot \textit{put}(b,p_1)\label{eq:ex_rec1}
\end{align}
where $p_1$, $p_2$, $p_3$ are priorities such that $p_1 \prec p_2$ and $p_1 \prec p_3$, and $p_2$ and $p_3$ are incomparable. Since the $\textit{rm}(\textit{empty})$ operations
are read-only (they don't affect the queue's state), they are selected first. An $\textit{rm}(\textit{empty})$-operation $o$ is correct when every $\textit{put}(x,p)$ operation before $o$ is matched to a $\textit{rm}(x)$ operation which also occurs before $o$. This is true in this case for $x\in \{a,c\}$. Thus, the correctness of (\ref{eq:ex_rec1}) reduces to the correctness of
\begin{align}
\textit{put}(c,p_2) \cdot \textit{put}(a,p_1) \cdot \textit{rm}(a) \cdot \textit{rm}(c) \cdot \textit{put}(d,p_2) \cdot \textit{put}(f,p_3) \cdot \textit{rm}(f) \cdot \textit{put}(b,p_1)\label{eq:ex_rec2}
\end{align}
When the execution contains no $\textit{rm}(\textit{empty})$-operation, the procedure selects a $\textit{put}$ operation adding a value that is not removed and that has a maximal priority. For (\ref{eq:ex_rec2}), it selects $\textit{put}(d,p_2)$ because $p_2$ is a maximal priority. This operation is correct since $d$ is the last value with priority $p_2$ in the execution, and the correctness of (\ref{eq:ex_rec2}) reduces to the correctness of
\begin{align}
\textit{put}(c,p_2) \cdot \textit{put}(a,p_1) \cdot \textit{rm}(a) \cdot \textit{rm}(c) \cdot \textit{put}(f,p_3) \cdot \textit{rm}(f) \cdot \textit{put}(b,p_1)\label{eq:ex_rec3}
\end{align}
If no operations like above can be found, $\textit{Check-PQ-Seq}$ selects a pair of $\textit{put}$ and $\textit{rm}$ operations adding and removing the same maximal priority value. For (\ref{eq:ex_rec2}), it can select
$\textit{put}(c,p_2)$ and $\textit{rm}(c)$. The value returned by $\textit{rm}(c)$ is correct if all the values of priority smaller than $p_2$ added before $\textit{rm}(c)$ are also removed before $\textit{rm}(c)$. In this case, $a$ is the only value of priority smaller than $p_2$ and it satisfies this property. Applying a similar reasoning for all the remaining values, it can be proved that this execution is correct.

\begin{algorithm}[t]
\footnotesize{
\KwIn {A data-differentiated sequential execution $e$}
\KwOut{$\mathsf{true}$ iff $e\in \seqPQ$}

\If {$e = \epsilon$}
{\Return $\mathsf{true}$\;}

\If {$\mathsf{Has\text{-}EmptyRemoves}(e)$}
{
    \If {$\exists\ o=\textit{rm}(\textit{empty})\in e$ such that $\mathsf{EmptyRemove\text{-}Seq}(e,o)$ holds}
    {
        \KwRet $\textit{Check-PQ-Seq}(e \setminus o)$\;
    }
}
\ElseIf{$\mathsf{Has\text{-}UnmatchedMaxPriority}(e)$}
{
    \If {$\exists\ x \in \textit{values}(e)$ such that $\mathsf{UnmatchedMaxPriority\text{-}Seq}(e,x)$ holds}
    {
        \KwRet $\textit{Check-PQ-Seq}(e \setminus x)$\;
    }
}

\Else
{
    \If {$\exists\ x \in \textit{values}(e)$ such that $\mathsf{MatchedMaxPriority\text{-}Seq}(e,x)$ holds}
    {
        \KwRet $\textit{Check-PQ-Seq}(e \setminus x)$\;
    }
    \Else {\KwRet $\mathsf{false}$\;}
}}
\caption{$\textit{Check-PQ-Seq}$}
\label{alg:seq_check}
\end{algorithm}


Formally, the selected operations depend on the following set of predicates on executions:
\begin{align*}
& \mathsf{Has\text{-}EmptyRemoves}(e)=\mathsf{true} \mbox{ iff  $e$ contains a $\textit{rm}(\textit{empty})$-operation} \hspace{1cm}\\
& \mathsf{Has\text{-}UnmatchedMaxPriority}(e)=\mathsf{true} \mbox{ iff $p\in \textit{unmatched-priorities}(e)$ for a maximal $p$}
\end{align*}
where $\textit{priorities}(e)$, resp., $\textit{unmatched-priorities}(e)$, is the set of priorities occurring in $\textit{put}$ operations of $e$, resp., in $\textit{put}$ operations of $e$ for which there is no $\textit{rm}$ operation removing the same value. We call the latter \emph{unmatched} put operations. A put operation which is not unmatched is called \emph{matched}. For simplicity, we consider the following syntactic sugar $\mathsf{Has\text{-}MatchedMaxPriority}(e)=\neg \mathsf{Has\text{-}EmptyRemoves}(e)\land \neg \mathsf{Has\text{-}UnmatchedMaxPriority}(e)$. By an abuse of notation, we assume  $\mathsf{Has\text{-}UnmatchedMaxPriority}(e) \Rightarrow \neg \mathsf{Has\text{-}EmptyRemoves}(e)$ (this is sound by the order of the conditionals in $\textit{Check-PQ-Seq}$).

The predicates defining the correctness of the selected operations are defined as follows:
\begin{align*}
&\mathsf{EmptyRemove\text{-}Seq}(e,o)=\mathsf{true} \mbox{ iff  $e= u\cdot o\cdot v$ and $\textit{matched}(u)$} \\
&\mathsf{UnmatchedMaxPriority\text{-}Seq}(e,x)=\mathsf{true}  \mbox{ iff  $e= u\cdot \textit{put}(x,p)\cdot v$, $p\not\prec \textit{priorities}(u\cdot v)$,} \\
&\hspace{6.4cm}p\not\in \textit{priorities}(v) \\
&\mathsf{MatchedMaxPriority\text{-}Seq}(e,x)=\mathsf{true} \mbox{ iff  $e= u\cdot \textit{put}(x,p)\cdot v\cdot \textit{rm}(x)\cdot w$, $\textit{matched}_\prec(u\cdot v,p)$,} \\
&\hspace{4.3cm}\mbox{$p\not\preceq \textit{unmatched-priorities}(u\cdot v\cdot w)$, $p\not\prec \textit{priorities}(u\cdot v\cdot w)$,} \\
&\hspace{4.3cm}\mbox{and  $p\not\in \textit{priorities}(v\cdot w)$}
\end{align*}
where $p\prec \textit{priorities}(e)$ when $p\prec p'$ for some $p'\in \textit{priorities}(e)$ (and similarly for $p\prec \textit{unmatched-priorities}(e)$ or $p\preceq \textit{unmatched-priorities}(e)$),
$\textit{matched}_\prec(e,p)$ holds when each value with priority strictly smaller than $p$ is removed in $e$, and $\textit{matched}(e)$ holds when $\textit{matched}_\prec(e,p)$ holds for each $p\in\mathbb{P}$. Compared to the example presented at the beginning of the section, these predicates take into consideration that multiple values with the same priority are removed in FIFO order: the predicate $\mathsf{MatchedMaxPrioritySeq}(e,x)$ holds when $x$ is the last value with priority $p$ added in $e$.

When $o$ is an $\textit{rm}(\textit{empty})$-operation, $e\setminus o$ is the maximal subsequence of $e$ which doesn't contain $o$. For an execution $e$, $\textit{values}(e)$ is the set of values  in call/return actions of $e$.





The following lemma states the correctness of $\textit{Check-PQ-Seq}$ (see Appendix~\ref{sec:appendix definition of seqPQ and proof of Lemma EQP rules and semantics} for the proof).

\begin{lemma}
\label{lemma:EPQ rules and semantics}
$\textit{Check-PQ-Seq}(e)=\mathsf{true}$ iff $e\in \seqPQ$, for every data-differentiated sequential execution $e$.
\end{lemma}




\subsection{Checking Linearizability of Data-Differentiated Concurrent Executions}\label{ssec:conc_exec}

The extension of $\textit{Check-PQ-Seq}$ to concurrent executions, checking whether they are linearizable w.r.t. $\seqPQ$, is obtained by replacing every predicate $\Gamma\mathsf{\text{-}Seq}$ with
\begin{align*}
\Gamma\mathsf{\text{-}Conc}(e,\alpha) = \mathsf{true}\mbox{ iff there is a sequential execution $s$ such that $e\sqsubseteq s$ and $\Gamma\mathsf{\text{-}Seq}(s,\alpha)$}
\end{align*}
for each $\Gamma\in \{\mathsf{EmptyRemove}, \mathsf{UnmatchedMaxPriority}, \mathsf{MatchedMaxPriority}\}$. The obtained procedure is denoted by
$\textit{Check-PQ-Conc}$ (recursive calls are modified accordingly).


The following lemma states the correctness of $\textit{Check-PQ-Conc}$. Completeness follows easily from the properties of $\seqPQ$. If $\textit{Check-PQ-Conc}(e) = \textit{false}$, then there exists a set $D$ of values s.t. either  $\mathsf{EmptyRemove\text{-}Conc}(e \vert D)$ is false, or $\mathsf{UnmatchedMaxPriority\text{-}Conc}(e \vert D,x)$ is false for all the values $x$ of maximal priority that are not removed (and there exists at least one such value), or $\mathsf{MatchedMaxPriority\text{-}Conc}(e \vert D,x)$ is false for all the values $x$ of maximal priority (and these values are all removed in $e \vert D$). It can be easily seen that we get $e \vert D\not\sqsubseteq \seqPQ$ in all cases, which by the closure under projection of $\seqPQ$ implies, $e \not\sqsubseteq \seqPQ$ (since every linearization of $e$ includes as a subsequence a linearization of $e \vert D$).

\begin{lemma}\label{lemma:con-check-EPQ is correct}
$\textit{Check-PQ-Conc}(e)=\mathsf{true}$ iff $e \sqsubseteq \seqPQ$, for every data-differentiated $e$.
\end{lemma}


Proving soundness is highly non-trivial and one of the main technical contributions of this paper (see Appendix~\ref{sec:appendix subsection proof of lemma con-check-EPQ is correct} for a complete proof). The main technical difficulty is showing that for any execution $e$, any linearization of $e\setminus x$ for some maximal priority value $x$ can be extended to a linearization of $e$ provided that $\mathsf{UnmatchedMaxPriority}$ or $\mathsf{MatchedMaxPriority}$ holds (depending on whether there are values with the same priority as $x$ in $e$ which are not removed).



We explain the proof of this property on the execution $e$ in \figurename~\ref{fig:concurrent execution for EPQ1}(a) where $p_1 \prec p$, $p_1 \prec p_2$, and the predicate $\mathsf{Has\text{-}MatchedMaxPriority}(e)$ holds. Assume that there exist two sequential executions $l$ and $l'$ such that $e \sqsubseteq l=u \cdot \textit{put}(x,p) \cdot v \cdot \textit{rm}(x) \cdot w$, $\mathsf{MatchedMaxPriority\text{-}Seq}(l,x)$ holds, and $e \setminus x \sqsubseteq l' \in \seqPQ$. Let $u=\epsilon$, $w$ be any sequence formed of $\textit{put}(z_2,p_2)$ and $\textit{rm}(z_1)$ (we distinguish them by adding the suffix ``$-w$'' to their name, e.g., $\textit{rm}(z_1)-w$), and $v$ be any sequence containing the remaining operations. In general, the linearization $l'$ can be defined by choosing for each operation, a point in time between its call and return, called \emph{linearization point}. The order between the linearization points defines the sequence $l'$. \figurename~\ref{fig:concurrent execution for EPQ1}(a) draws linearization points for the operations in $e \setminus x$ which define $l'$~\footnote{In general, there may exist multiple ways of choosing linearization points to define the same linearization. Our construction is agnostic to this choice.}.
We show how to construct a sequence $l''= l''_1 \cdot \textit{put}(x,p) \cdot l''_2 \cdot \textit{rm}(x) \cdot l''_3\in\seqPQ$ s.t. $e \sqsubseteq l''$.
\begin{itemize}
\item[-] An operation is called $p$-comparable (resp., $p$-incomparable) when it receives as argument a value of priority comparable to $p$ (resp., incomparable to $p$). Defining $l''_1$, $l''_2$, and $l''_3$ as the projection of $l'$ to the set of operations in $u$, $v$ and $w$, respectively, leads to a sequence $l''\not\in\seqPQ$. This is because $\mathsf{MatchedMaxPriority\text{-}Seq}(l,x)$ imposes no restriction on  $p$-incomparable operations in $u \cdot v$, and the projection of $l'$ to $p$-incomparable operations in $u \cdot v$ is not in $\seqPQ$. In this example, this projection is $\textit{put}(z_1,p_2) \cdot \textit{rm}(z_2)$.

\item[-] We define the sets of operations $U'$, $V'$ and $W'$ such that $l''_1$, $l''_2$ and $l''_3$ are the projections of $l'$ to $U'$, $V'$, and $W'$, respectively. This is done in two steps:
\begin{enumerate}
\item The first step is to define $W'$. The $p$-comparable operations in $W'$ are the same as in $w$. To identify the $p$-incomparable operations in $W'$, we search for a $p$-incomparable operation $o$ which either happens before some $p$-comparable operation in $w$, or whose linearization point occurs after $\textit{ret}(\textit{rm},x)$. We add to $W'$ the operation $o$ and all the $p$-incomparable operations occurring after $o$ in $l'$. In this example, $o$ is $\textit{rm}(z_1)$ and the only $p$-incomparable operation occurring after $o$ in $l'$ is $\textit{rm}(z_2)$ (they are surrounded by boxes in the figure). In this process, whether a $p$-incomparable operation is in $W'$ or not only relies on whether it is before or after such an $o$ in $l'$.
\item The second step is to define $U'$ and $V'$. $U'$ contains two kinds of operations: (1) operations whose linearization points are before $\textit{ret}(\textit{put},x,p)$, and (2) other $\textit{put}$ operations with priority $p$. $V'$ contains the remaining operations. In this example, $U'$ contains $\textit{put}(z_1,p_2)$ and $\textit{put}(x_2,p)$.
\end{enumerate}
\item[-] In conclusion, we have that $l''_1 = \textit{put}(z_1,p_2) \cdot \textit{put}(x_2,p)$, $l''_2 = \textit{put}(z_2,p_2) \cdot \textit{rm}(x_2) \cdot \textit{put}(y_1,p_1) \cdot \textit{rm}(y_1)$, and $l''_3 = \textit{rm}(z_1) \cdot \textit{rm}(z_2)$. \figurename~\ref{fig:concurrent execution for EPQ1}(b) draws linearization points for each operation in $e$ defining the linearization $l''$.
\end{itemize}




\begin{figure}[t]
  \centering
  \includegraphics[width=.8\textwidth]{figures/PIC-HIS-EPQ1-TwoHis-2.pdf}
  \caption{A concurrent execution $e$ exemplifying the soundness of $\textit{Check-PQ-Conc}$.}
  \label{fig:concurrent execution for EPQ1}
\end{figure}




Section~\ref{sec:co-regular of extended priority queues} introduces a characterization of concurrent priority queue violations using a set of \emph{non-recursive} automata (whose states consist of a fixed number of registers), whose standard synchronized product is equivalent to $\textit{Check-PQ-Conc}$ (modulo a renaming of values which is possible by data-independence). Since $\seqPQ$ is closed under projection (Lemma~\ref{lem:closure_proj}), the recursion in $\textit{Check-PQ-Conc}$ can be eliminated by checking that each projection of a given execution $e$ passes a non-recursive version of $\textit{Check-PQ-Conc}$ where every recursive call $\text{{\bf return}}\ \textit{Check-PQ-Conc}(\ldots)$ is
replaced by  $\text{{\bf return}}\ \mathsf{true}$. Let $\textit{Check-PQ-Conc-NonRec}$ be the thus obtained procedure.

\begin{lemma}
\label{lemma:EPQ as multi in MRpri for history}
Given a data-differentiated execution $e$, $e \sqsubseteq \seqPQ$ if and only if for each $e' \in \textit{proj}(e)$, $\textit{Check-PQ-Conc-NonRec}(e')$ returns $\mathsf{true}$.
\end{lemma}

%\section{Checking Linearizability By Recursive Procedure}
\label{sec:checking linearizablity by recursive procedure}

In this section, we extend $\textit{check-PQ}$ with linearizability, and use it to simplify checking linearizability w.r.t $\textit{PQ}$. This inspired us to partition the concurrent executions which are not linearizable w.r.t $\textit{PQ}$ into violation of $\textit{MS}(R)$. To ensure the correctness of $\textit{check-PQ}$, we step-by-step linearizability to incrementally build linearization. Our step-by-step linearizability is inspired by the step-by-step linearizabilility of queue and stacks in \cite{Bouajjani:2015}. The related definitions and proofs of lemmas in this section can be found in Appendix \ref{sec:appendix in section step-by-step linearizability of extended priority queues}.


\subsection{Linearizability}
\label{subsec:linearizability}

A (concurrent) execution $e$ is a sequence of call and return actions which satisfy a well-formedness property: every return has a matching call action before it in $e$, and an operation $o$ can be used only twice in $e$, once in a call action, and once in a return action. The definition of data-differentiated, renaming function, data-independence and projection extends to (concurrent) executions. For instance, an execution is data-differentiated if, for all $d \in \mathbb{D}$, there is at most one $\textit{cal}_{\_}(\textit{put},d,\_)$.

An implementation $\mathcal{I}$ is a set of concurrent executions. Implementations represent libraries whose methods are called by external programs. In the remainder of this work, we consider only completed executions, where each call action has a corresponding return action. This simplification is sound when implementation methods can always make progress in isolation \cite{Henzinger:2013}: formally, for any execution $e$ with pending operations, there exists an execution $e'$ obtained by extending $e$ only with the return actions of the pending operations of $e$. Intuitively this means that methods can always return without any help from outside threads, avoiding deadlock.

Given execution $e$, the happen-before relation $<_{\textit{hb}}$ is defined as follows: $o_1 <_{\textit{hb}} o_2$, if the return action of $o_1$ is before the call action of $o_2$ in $e$. The happens-before relation are interval orders \cite{Bouajjani:2015POPL}: for distinct $o_1,o_2,o_3,o_4$, if $o_1 <_{\textit{hb}} o_2$ and $o_3 <_{\textit{hb}} o_4$, then either $o_1 <_{\textit{hb}} o_4$, or $o_3 <_{\textit{hb}} o_2$. Intuitively, this comes from the fact that concurrent threads share a notion of global time.

Linearizability \cite{Herlihy:1990} is the \emph{de facto} standard correctness condition for concurrent data structures. Given concurrent execution $e$ and sequential execution $s$ of length $n$, we say that $e$ is linearizable w.r.t $s$, denoted $e \sqsubseteq s$, if they have same set of operations, and there is a bijection $f$ between operations s.t.

\begin{itemize}
\item[-] if $o_1 <_{\textit{hb}} o_2$, then $f(o_1) <_{\textit{hb}} f(o_2)$,

\item[-] $o$ and $f(o)$ have same call and return actions.
\end{itemize}

A concurrent execution $e$ is linearizable w.r.t a set $S$ of sequential executions, denoted $e \sqsubseteq S$, if there exists $s \in S$ such that $e \sqsubseteq s$. A set of concurrent executions $E$ is linearizable w.r.t $S$, denoted $E \sqsubseteq S$, if $e \sqsubseteq S$ for all $e \in H$.

The following lemma states that with the help of data-independence, it is enough to consider only data-differentiated executions when checking linearizability. This is similar to that in \cite{Abdulla:2013}, where they use the notion of data-independence in \cite{Wolper:1986}. Thus, in the remainder of the paper, we focus on characterizing linearizability for data-differentiated executions. The proof of this lemma can be found in Appendix \ref{sec:appendix in section data-independence of EPQ}.

\begin{restatable}{lemma}{DataDifferentiatedisEnoughforPQ}
\label{lemma:data differentiated is enough for PQ}
A data-independent implementation $\mathcal{I}$ is linearizable w.r.t data-independent set $S$ of sequential executions, if and only if $\mathcal{I}_{\neq}$ is linearizable with respect to $S_{\neq}$.
\end{restatable}




\subsection{Modified $\textit{check-PQ}$}
\label{subsec:recursive procedure con-check-EPQ}

Let us extend $\textit{check-PQ}$ with linearizability to deal with linearizability w.r.t $\textit{PQ}$ as follows: 

\begin{itemize}
\setlength{\itemsep}{0.5pt}
\item[-] The input $e$ is a concurrent execution. 

\item[-] $\textit{last}(e)$ is the union of $\textit{last}(u)$ for any sequential execution $u$ such that $e \sqsubseteq u$. $e \sqsubseteq \textit{MS}(R)$ with witness $x$ (resp., $o$), if $e \sqsubseteq u \in \textit{MS}(R)$ and $x$ (resp., $o$) is the witness of $u$. 
    
\item[-] $P_i(e,o)$ and $P_i(e,x)$ holds, if $e \sqsubseteq \textit{MS}(\textit{PQ}_i)$ withe witness $o$ or $x$, respectively. 
\end{itemize} 

The following lemma states that $\textit{check-PQ}$ is correct. 

\begin{restatable}{lemma}{ConCheckEPQIsCorrect}
\label{lemma:con-check-EPQ is correct}
Given a data-differentiated execution $e$, $e \sqsubseteq \textit{EPQ}$, if and only if, $\textit{check-PQ} = \textit{true}$.
\end{restatable} 

\begin {proof} (Sketch)
It is not hard to see that $\textit{check-PQ}(e) = \textit{false} \Rightarrow e \not\sqsubseteq \textit{PQ}$, since $\textit{PQ}_{\neq} \subset \textit{MS}(R)_{\neq}$. To ensure that $\textit{check-PQ}(e) = \textit{true} \Rightarrow e \sqsubseteq \textit{PQ}$, we require $\textit{PQ}$ to satisfy the following property.  

for any data-differentiated execution $e$,
\begin{itemize}
\setlength{\itemsep}{0.5pt}
\item[-] if $e \sqsubseteq \textit{MS}(R)$ ($R \in \{ \textit{PQ}_1, \textit{PQ}_2 \}$) with witness $x$, we have: $e \setminus x \sqsubseteq \textit{PQ} \Rightarrow e \sqsubseteq \textit{PQ}$.

\item[-] if $e \sqsubseteq \textit{MS}(\textit{PQ}_3)$ and $o$ is a $\textit{rm}(\textit{empty})$, we have: $e \setminus o \sqsubseteq \textit{PQ} \Rightarrow e \sqsubseteq \textit{PQ}$.
\end{itemize}

With this property, we can build a linearization of a whole execution by increasingly construct linearization of sub-execution from $\epsilon$, and this convince us that $\textit{check-PQ}(e) = \textit{true} \Rightarrow e \sqsubseteq \textit{PQ}$. 

Let us briefly explain the idea of proving this property for $\textit{PQ}_1$ with an example. Given a data-differentiated concurrent execution $e \sqsubseteq u \cdot \textit{put}(x,p) \cdot v \cdot \textit{rm}(x) \cdot w \in \textit{MS}(\textit{PQ}_1)$ with witness $x$ and assume that $e \setminus x \sqsubseteq l' \in \textit{PQ}$, as shown in \figurename~\ref{fig:concurrent execution for EPQ1} (a), we explicitly construct a sequence $l''= l''_1 \cdot \textit{put}(x,p) \cdot l''_2 \cdot \textit{rm}(x) \cdot l''_3$ and prove that $e \sqsubseteq l'' \in \textit{PQ}$. In this example, $p_1 \prec p$, $p_1 \prec p_2$, $u=\epsilon$, $w$ contains $\textit{rm}(z_2), \textit{put}(x_3,p_2), \textit{rm}(z_1)$ (emphasize by $\textit{rm}(z_2)-w$), and the remanning operations are in $v$. We also explicitly draw the linearization points according to $l'$. We construct $l''$ as follows, and this method does not rely on the locations of linearization points of $l'$ and operations of $x$. Here we implicitly mix operation $o$ (when $\textit{call}_o(m,a)$, $\textit{ret}_o(m,b)$ in $e$) with $m(a,b)$.

\begin{itemize}
\setlength{\itemsep}{0.5pt}
\item[-] Let $p$-comparable operations (resp., $p$-incomparable operations) be the set of operations with items whose priority is comparable with $p$ (resp., incomparable with $p$). It seems that we can construct $l''_1$, $l''_2$ and $l''_3$ as the projection of $l'$ into operations of $u$, $v$ and $w$, respectively. However, this is incorrect, since $\textit{PQ}_1$ has no restriction to $p$-incomparable operations operations in $u \cdot v$, and thus, there is no guarantee that the projection of $l'$ into $p$-incomparable operations operations in $u \cdot v$ being correct. In this example, such projection is $\textit{put}(z_1,p_3) \cdot \textit{rm}(z_2)$ and is incorrect.

\item[-] Let us construct sets $U'$, $V'$ and $W'$, such then make $l''_1$, $l''_2$ and $l''_3$ the projections of $l'$ into $U'$, $V'$ and $W'$, respectively. This contains two steps:

\item[-] The first step is to define $W'$. The $p$-comparable operations in $W'$ is same as that in $w$. To obtain $p$-incomparable operations in $W'$, we try to find a $p$-incomparable operation $o$ which either happens before some $p$-comparable operations in $w$, or with linearization points after $\textit{rm}(x)$. Then, we put $o$ and all $p$-incomparable operations after $o$ in $l'$ into $W'$ (emphasized by boxes in example). In this example, $o$ is $\textit{rm}(z_1)$, and $W'$ contains $\textit{put}(x_3,p)$, $\textit{rm}(z_1)$ and $\textit{rm}(z_2)$. We use boxes to emphasize they are put into $W'$.

\item[-] The second step is to define $U'$ and $V'$. $U'$ contains the following two kinds of operations: (1) Operations whose linearization points are before $\textit{ret}(\textit{put},x)$, and (2) other $\textit{put}$ operations with priority $p$. $V'$ contains the remanning operations. In this example, $U'$ contains $\textit{put}(z_1,p_3)$ and $\textit{put}(x_2,p_2)$.

\item[-] In this example, $l''_1 = \textit{put}(z_1,p_2) \cdot \textit{put}(x_2,p)$, $l''_2 = \textit{put}(z_2,p_2) \cdot \textit{rm}(x_2) \cdot \textit{put}(y_1,p_1) \cdot \textit{rm}(y_1)$, and $l''_3 = \textit{put}(x_3,p) \cdot \textit{rm}(z_1) \cdot \textit{rm}(z_2)$. In \figurename~\ref{fig:concurrent execution for EPQ1} (b), we add linarization points according to $l''$, and we can see that $l''$ holds as required.
\end{itemize}

This completes the proof of this lemma. 


\qed
\end {proof}



\begin{figure}[htbp]
  \centering
  \includegraphics[width=1 \textwidth]{figures/PIC-HIS-EPQ1-TwoHis.pdf}
%\vspace{-10pt}
  \caption{The process of obtaining linearization of $e$}
  \label{fig:concurrent execution for EPQ1}
\end{figure}








%\input{preliminaries-EPQ}

%\section{Inductive Rules of Extended Priority Queue}
\label{sec:inductive rules of extended priority queue}

A priority queue contains two method: $\textit{put}$ and $\textit{rm}$. A $\textit{put}$ method has two arguments, while the first argument is an item and the second argument is its priority. A $\textit{put}$ method is used to put an item into the priority queue with certain priority. Here we assume that the item is chosen from a specific (possibly infinite) data domain $\mathbb{D}$ and priority is chosen from a (possibly infinite) set $\mathbb{P}$. {\color {red}Moreover, we assume that there is a strict partial-order $<_\mathbb{P}$ among elements in $\mathbb{P}$.} A $\textit{rm}$ method intends to remove the item with minimal priority (with respect to $<_\mathbb{P}$) in priority queue and then returns it. It works as follows:

\begin{itemize}
\setlength{\itemsep}{0.5pt}
\item[-] If the priority queue is empty, then $\textit{rm}$ returns $\textit{empty}$.

\item[-] {\color {red}Else, $\textit{rm}$ returns a oldest element of one of minimal priorities. Formally, there are a set $S$ of items in priority queue. $S$ can be divided into several group, such that (1) items in each group have same priority, (2) the priorities of each two groups is incomparable and (3) no priority of items in $S$ can be larger than items not in $S$. Each group is the set of items of some minimal priority. Then, $\textit{rm}$ returns an item of some group, and this item must be putted earliest in this group. However, the chosen of group is arbitrary.}
\end{itemize}

We say that $\textit{put}(a,p)$ matches $\textit{rm}(b)$, if $a = b$. Our priority queue is an extension of common priority queue, where the priority is chosen from the set $\mathbb{N}$ of natural numbers, or from a set with total order. To distinguish our priority queue with common priority queue, we explicitly call our priority queue the extended priority queue.

Similar as \cite{Bouajjani:2015}, we use inductive rules to define the set of sequential executions of extended priority queue. Each rule is of the form $l_1 \cdot \ldots \cdot l_k \in \textit{EPQ} \wedge \textit{Guard}(l_1,\ldots,l_k,\textit{itm},\textit{pri}) \Rightarrow \textit{Expr}(l_1,\ldots,l_k,\textit{itm},\textit{pri}) \in \textit{EPQ}$. Here $\textit{EPQ}$ is the set of sequential executions of extended priority queue. Here $\textit{itm}$ and $\textit{pri}$ are two variables, and represent an arbitrary item and priority, respectively. $\textit{Guard}(l_1,\ldots,l_k,\textit{itm},\textit{pri})$ is a conjunction of conditions with the following notations:

\begin{itemize}
\setlength{\itemsep}{0.5pt}
\item[-] Given sequential execution $l$, $\textit{noRE}(l)$ is satisfied when each method event of $l$ is not $\textit{rm}(\textit{empty})$.

\item[-] Given sequential execution $l$, $\textit{LEI}(\textit{pri},l)$ is satisfied, if for priority of every item of $l$, $\textit{pri}$ is either larger, or equal, or incomparable with it. Here $L$ represents larger, $E$ represents equal, and $I$ represents incomparable. Similarly, we can define $\textit{LI}(\textit{pri},l)$. We use $U$ to represent items of unmatched $\textit{put}$, and $\textit{LI-U}(\textit{pri},l)$ is satisfied, if for priority of every item of unmatched $\textit{put}$ in $l$, $\textit{pri}$ is either larger, or incomparable with it. We use $M$ to represent items of matched $\textit{put}$, and define $\textit{LI-M}$ and $\textit{LEI-M}$ similarly.

    {\color {red} We use $E'$ to emphasize that equal must holds somewhere. For example, $\textit{LE'I}(\textit{pri},l)$ holds, if (1) for priority of every item of $l$, $\textit{pri}$ is either larger, or equal, or incomparable with it, and (2) $\textit{pri}$ indeed equals priority of some items of $l$. $\textit{LE'I-M}$ is similarly defined.}

\item[-] Given sequential execution $l$, its sub-sequence $l'$ and a priority $p$, $\textit{putInSeq}(l,l',p)$ is satisfied when all the $\textit{put}$ with priority $p$ of $l$ (if exists) is in $l'$.

\item[-] {\color {red}Given sequential execution $l$ and priority $p$, $\textit{matched-C}(l,p)$ is satisfied, if (1) for each item $a$ whose priority is comparable with $p$, if $\textit{put}(a,\_)$ is in $l$, then $\textit{rm}(a)$ is in $l$, and (2) for each item $a$ whose priority is comparable with $p$, if $\textit{rm}(a)$ is in $l$, then $\textit{put}(a,\_)$ is in $l$. Similarly, we can define $\textit{matched-All}(l)$, where all items in $l$, instead of items with priority comparable with some priority in $l$, is considered and matched.}
\end{itemize}

$\textit{Expr}(l_1,\ldots,l_k,\textit{itm},\textit{pri})$ is a expression $l'_1 \cdot \ldots \cdot l'_m$, where each $l'_i$ is chosen from (1) $l_j$ for some $j$, (2) method event with item $\textit{itm}$ and priority $\textit{pri}$ and (3) method event $\textit{rm}(\textit{empty})$.
%Given $l'_1,\ldots,l'_k$ and expression $e=\textit{Expr}(l_1,\ldots,l_k,\textit{itm},\textit{pri})$, we define $\llbracket e \rrbracket$ as the set of sequential executions which can be obtained from $e$ by replacing $l_i$ with $l'_i$ for each $i$, and replacing $\textit{itm}$ and $\textit{pri}$ with some values in $\mathbb{D}$ and $\mathbb{P}$, respectively.
Given a rule $R \equiv l_1 \cdot \ldots \cdot l_k \in \textit{EPQ} \wedge \textit{Guard}(l_1,\ldots,l_k,\textit{itm},\textit{pri}) \Rightarrow \textit{Expr}(l_1,\ldots,l_k,\textit{itm},\textit{pri}) \in \textit{EPQ}$ and a sequential execution $w$, if $w=l'_1 \cdot \ldots \cdot l'_k$, and $\textit{Guard}(l'_1,\ldots,l'_k,a,p)$ holds for some $a \in \mathbb{D}$ and $p \in \mathbb{P}$, then we use $w \xrightarrow{R} w'$ to denote that we can obtain $w'$ from $w$ according to rule $R$, where $w' = \textit{Expr}(l'_1,\ldots,l'_k,a,p)$. Let $\llbracket \textit{EPQ} \rrbracket$ be the set of sequential executions $w$ which can be derived from the empty word: $$\epsilon = w_0 \xrightarrow{R_1} w_1 \ldots \xrightarrow{R_k} w$$ where each $R_i$ is one rules of the extended priority queue. When clear from context, we abuse $\llbracket \textit{EPQ} \rrbracket$ by $\textit{EPQ}$.

\begin{definition}\label{def:inductive rules of priority queue}
$\textit{EPQ}$ is defined by the following rules:
\begin{itemize}
\setlength{\itemsep}{0.5pt}
\item[-] $\textit{EPQ}_0 \equiv \epsilon \in \textit{EPQ}$.

\item[-] $\textit{EPQ}_1 \equiv (u \cdot v \cdot w \in \textit{EPQ}) \wedge
(\textit{noRE}(u \cdot v \cdot w)) \wedge
(\textit{LEI}(\textit{pri}, u \cdot v \cdot w)) \wedge
(\textit{LI-U}(\textit{pri},u \cdot v \cdot w)) \wedge
(\textit{matched-C}(u \cdot v,\textit{pri}) ) \wedge
%(\textit{matched}(w) ) \wedge
(\textit{putInSeq}(u \cdot v \cdot w,u,\textit{pri}))
\Rightarrow
(u \cdot \textit{put}(\textit{itm},\textit{pri}) \cdot v \cdot \textit{rm}(\textit{itm}) \cdot w \in \textit{EPQ})$.

\item[-] $\textit{EPQ}_2 \equiv
(u \cdot v \in \textit{EPQ}) \wedge
(\textit{noRE}(u \cdot v)) \wedge
(\textit{LEI}(\textit{pri},u \cdot v)) \wedge
(\textit{putInSeq}(u \cdot v,u,\textit{pri}))
\Rightarrow
(u \cdot \textit{put}(\textit{itm},\textit{pri}) \cdot v \in \textit{EPQ})$.

\item[-] $\textit{EPQ}_3 \equiv
(u \cdot v \in \textit{EPQ}) \wedge
(\textit{matched-All}(u) )
\Rightarrow
(u \cdot \textit{rm}(\textit{empty}) \cdot v \in \textit{EPQ})$.
\end{itemize}
\end{definition}

\begin{example}\label{example:generate extended priority queue executions}
Given priorities $p_1,p_2,p_3$ with orders $p_1 <_{\mathbb{P}} p_2$ and $p_1 <_{\mathbb{P}} p_3$, one sequential execution of $\textit{EPQ}$ is generated as follows:

$\epsilon$ $\xrightarrow{\textit{EPQ}_1}$ $\textit{put}(a,p_1) \cdot \textit{rm}(a)$

$\xrightarrow{\textit{EPQ}_2}$ $\textit{put}(a,p_1) \cdot \textit{rm}(a) \cdot \textit{put}(b,p_1)$

$\xrightarrow{\textit{EPQ}_1}$ $\textit{put}(c,p_2) \cdot \textit{put}(a,p_1) \cdot \textit{rm}(a) \cdot \textit{rm}(c) \cdot \textit{put}(b,p_1)$

$\xrightarrow{\textit{EPQ}_2}$ $\textit{put}(c,p_2) \cdot \textit{put}(a,p_1) \cdot \textit{rm}(a) \cdot \textit{rm}(c) \cdot \textit{put}(d,p_2) \cdot \textit{put}(b,p_1)$

$\xrightarrow{\textit{EPQ}_1}$ $\textit{put}(c,p_2) \cdot \textit{put}(a,p_1) \cdot \textit{rm}(a) \cdot \textit{rm}(c) \cdot \textit{put}(d,p_2) \cdot \textit{put}(e,p_3) \cdot \textit{rm}(e) \cdot \textit{put}(b,p_1)$

$\xrightarrow{\textit{EPQ}_3}$ $\textit{put}(c,p_2) \cdot \textit{put}(a,p_1) \cdot \textit{rm}(a) \cdot \textit{rm}(c) \cdot \textit{rm}(\textit{empty}) \cdot \textit{put}(d,p_2) \cdot \textit{put}(e,p_3) \cdot \textit{rm}(e) \cdot \textit{put}(b,p_1)$
\end{example}

To facilitate our proof of latter sections, we need to separate $\textit{EPQ}_1$ into two cases: (1) no matched pair of $\textit{put}$ and $\textit{rm}$ in $u \cdot v \cdot w$ has priority $\textit{pri}$, (2) some matched pair of $\textit{put}$ and $\textit{rm}$ in $u \cdot v \cdot w$ has priority $\textit{pri}$.  Therefore, we separate $\textit{EPQ}_1$ into two rules:

\begin{itemize}
\setlength{\itemsep}{0.5pt}
\item[-] $\textit{EPQ}_1^{>} \equiv (u \cdot v \cdot w \in \textit{EPQ}) \wedge
(\textit{noRE}(u \cdot v \cdot w)) \wedge
(\textit{LI}(\textit{pri}, u \cdot v \cdot w)) \wedge
(\textit{LI-U}(\textit{pri},u \cdot v \cdot w)) \wedge
(\textit{matched-C}(u \cdot v,\textit{pri}) ) %\wedge
%(\textit{matched}(w) )
\Rightarrow
(u \cdot \textit{put}(\textit{itm},\textit{pri}) \cdot v \cdot \textit{rm}(\textit{itm}) \cdot w \in \textit{EPQ})$.

\item[-] $\textit{EPQ}_1^{=} \equiv (u \cdot v \cdot w \in \textit{EPQ}) \wedge
(\textit{noRE}(u \cdot v \cdot w)) \wedge
(\textit{LE'I}(\textit{pri}, u \cdot v \cdot w)) \wedge
(\textit{LI-U}(\textit{pri},u \cdot v \cdot w)) \wedge
(\textit{matched-C}(u \cdot v,\textit{pri}) ) %\wedge
%(\textit{matched}(w) )
\wedge
(\textit{putInSeq}(u \cdot v \cdot w,u,\textit{pri}))
\Rightarrow
(u \cdot \textit{put}(\textit{itm},\textit{pri}) \cdot v \cdot \textit{rm}(\textit{itm}) \cdot w \in \textit{EPQ})$.
\end{itemize}


For $\textit{EPQ}_2$, we also need to distinguish two cases: (1) no matched pair of $\textit{put}$ and $\textit{rm}$ in $u \cdot v$ has priority $\textit{pri}$, (2) some matched pair of $\textit{put}$ and $\textit{rm}$ in $u \cdot v$ has priority $\textit{pri}$. Therefore, we separate $\textit{EPQ}_2$ into two rules:

\begin{itemize}
\setlength{\itemsep}{0.5pt}
\item[-] $\textit{EPQ}_2^{>} \equiv
(u \cdot v \in \textit{EPQ}) \wedge
(\textit{noRE}(u \cdot v)) \wedge
(\textit{LEI}(\textit{pri},u \cdot v)) \wedge
(\textit{LI-M}(\textit{pri},u \cdot v)) \wedge
(\textit{putInSeq}(u \cdot v,u,\textit{pri}))
\Rightarrow
(u \cdot \textit{put}(\textit{itm},\textit{pri}) \cdot v \in \textit{EPQ})$.

\item[-] $\textit{EPQ}_2^{=} \equiv
(u \cdot v \in \textit{EPQ}) \wedge
(\textit{noRE}(u \cdot v)) \wedge
(\textit{LEI}(\textit{pri},u \cdot v)) \wedge
(\textit{LE'I-M}(\textit{pri},u \cdot v)) \wedge
(\textit{putInSeq}(u \cdot v,u,\textit{pri}))
\Rightarrow
(u \cdot \textit{put}(\textit{itm},\textit{pri}) \cdot v \in \textit{EPQ})$.
\end{itemize}

To persuade readers that our rules is indeed the rules of extended priority queue, in Appendix \ref{sec:appendix in section inductive rules of extended priority queue}, we give a semantical version definition $\textit{EPQ}_s$ of extended priority queue, and shows that the language generated by our rules equals the set of traces of $\textit{EPQ}_s$. To model the possible behaviors of extended priority queue, we model it as an labelled transition system (shortened as LTS) $\textit{LTS}_e$. Each state of $\textit{LTS}_e$ is a function from $\mathbb{P}$ into sequences over $\mathbb{D}$, and represents a snapshot of contents of extended priority queue. $\textit{EPQ}_s$ is the set of traces of $\textit{LTS}_e$. The definition of $\textit{EPQ}_s$ and the proof of the following lemma can be found in Appendix \ref{sec:appendix in section inductive rules of extended priority queue}.

\begin{restatable}{lemma}{EPQRulesAndSemantics}
\label{lemma:EPQ rules and semantics}
$\textit{EPQ} = \textit{EPQ}_s$.
\end{restatable}


Given a sequential execution $w \in \textit{EPQ}$ and let $P$ be the set of priorities in $w$. According to above example, $w$ is generated from $\epsilon$ as follows: Let $P$ be the set of priorities of $w$ and $p$ be one of minimal priority in $P$. Then we start to loop. In each round of the loop there are two possibilities: (1) If there are matched $\textit{put}$ and $\textit{rm}$ with priority $p$ in $w$: Add pairs of matched $\textit{put}$ and $\textit{rm}$ with priority $p$ by using one time of $\textit{EPQ}_1^{>}$ and then (possibly) several times of $\textit{EPQ}_1^{=}$, and then (possibly) add unmatched $\textit{put}$ with priority $p$ by using $\textit{EPQ}_2^{=}$, (2) If there are no matched $\textit{put}$ and $\textit{rm}$ with priority $p$ in $w$: Add unmatched $\textit{put}$ with priority $p$ by using $\textit{EPQ}_2^{>}$ for several times. Then, let $P = P \setminus \{ p \}$, and restart this loop, until $P = \emptyset$. Finally, add $\textit{rm}(\textit{empty})$ by using $\textit{EPQ}_3$ for several times. We can see that, the order for generating $w$ may be not fixed, since some priorities are incomparable.

Thus, given $w$, we define $\textit{last}(w)$ as the set of last possible rule to generate $w$ according to the rules of extended priority queues:

\begin{itemize}
\setlength{\itemsep}{0.5pt}
\item[-] If $w$ contains $\textit{rm}(\textit{empty})$, then $\textit{last}(w) = \{ \textit{EPQ}_3 \}$.

\item[-] Else, if items of several unmatched $\textit{put}$ and matched $\textit{put}$ have a maximal priority of $w$, then $\textit{last}(w)$ contains $\textit{EPQ}_2^{=}$.

\item[-] Else, if items of only several unmatched $\textit{put}$ have a maximal priority of $w$, then $\textit{last}(w)$ contains $\textit{EPQ}_2^{>}$.

\item[-] Else, if items of only more than one matched $\textit{put}$ have a maximal priority of $w$, then $\textit{last}(w)$ contains $\textit{EPQ}_1^{=}$.

\item[-] Else, if items of only one matched $\textit{put}$ has a maximal priority of $w$, then $\textit{last}(w)$ contains $\textit{EPQ}_1^{>}$.

\item[-] Else ($w = \epsilon$), $\textit{last}(w) = \{ \textit{EPQ}_0 \}$.
\end{itemize}

Note that $\textit{last}(w)$ may contains more than one rules. For example, given $w = \textit{put}(c,p_2) \cdot \textit{put}(a,p_1) \cdot \textit{rm}(a) \cdot \textit{rm}(c) \cdot \textit{put}(d,p_2) \cdot \textit{put}(e,p_3) \cdot \textit{rm}(e) \cdot \textit{put}(b,p_1)$ and orders $p_1 <_{\mathbb{P}} p_2$ and $p_1 <_{\mathbb{P}} p_3$, then $\textit{last}(w) = \{  \textit{EPQ}_2^{=}, \textit{EPQ}_1^{>} \}$. The notion of $\textit{last}$ can be extended into execution and histories: given a history $h$, $\textit{last}(h) = \textit{last}(u)$, where $u$ is any sequential execution such that $h \sqsubseteq u$. When $\textit{last}(e)$ contains only one rule $R$, we write $\textit{last}(e)=R$ for simplicity.


%\section{Data-Independence of Extended Priority Queue}
\label{sec:data-independence of extended priority queue}

Data-independence \cite{Wolper:1986} can be used to effectively handle unbounded data domain. In this section, we slightly modify the notion of data-independence in \cite{Wolper:1986} and propose data-differentiated sequences and data-independence for extended priority queues.

Let $\_$ denote an element, of which the value is irrelevant. A sequential execution $e$ of extended priority queue is said to be data-differentiated if, for all $d \in \mathbb{D}$, there is at most one method event $\textit{put}(d,\_)$ in $e$. Note that a data-differentiated sequential execution $e$ may contains more than one items with a same priority. The subset of data-differentiated sequential executions of a set $S$ is denoted by $S_{\neq}$. The definition extends to (sets of) executions and histories. For instance, an execution is data-differentiated if, for all $d \in \mathbb{D}$, there is at most one $\textit{cal}_{\_}(\textit{put},d,\_)$.

\begin{example}\label{example:data-differentiated}
$\textit{cal}_{o_1}(\textit{put},a,7) \cdot \textit{ret}_{o_1}(\textit{put},a) \cdot \textit{cal}_{o_2}(\textit{put},a,8) \cdot \textit{ret}_{o_2}(\textit{put},a)$ is not data-differentiated, since there are two $\textit{put}$ methods with the same item.
\end{example}

A renaming function $r$ for extended priority queue is a function from $\mathbb{D}$ to $\mathbb{D}$. Given a sequential execution (resp., execution or history) $e$, we denote by $r(e)$ the sequential execution (resp., execution or history) obtained from $e$ by replacing every item $x$ by $r(x)$. Note that here the renaming functions rename only the items and keep the priorities unchanged. The intuitive explanation is that renaming items will not influence the executions of program, while renaming priorities may influence the executions of program. 

%\vspace{-6pt}
\begin{definition}\label{def:priority-value data-independence}
A set of sequential executions (resp., executions or histories) $S$ is data-independent, if:
\begin{itemize}
\setlength{\itemsep}{0.5pt}
\item[-] for all $e \in S$, there exists $e' \in S'$, and a renaming function $r$, such that $e=r(e')$,

\item[-] for all $e \in S$ and for all renaming $r$, $r(e) \in S$.
\end{itemize}
\end{definition}

The following lemma states that, when checking that a data-independent implementation $\mathcal{I}$ is linearizable with respect to a data-independent specification, it is enough to do so for data-differentiated executions, similar as that in \cite{Abdulla:2013}, where the notion of data-independence and differentiated in \cite{Wolper:1986} is used. Thus, in the remainder of the paper, we focus on characterizing linearizability for data-differentiated executions, rather than arbitrary ones. The proof of this lemma can be found in Appendix \ref{sec:appendix in section data-independence of EPQ}.

\begin{restatable}{lemma}{DataDifferentiatedisEnoughforPQ}
\label{lemma:data differentiated is enough for PQ}
A data-independent implementation $\mathcal{I}$ is linearizable with respect to a data-independent specification $S$, if and only if $\mathcal{I}_{\neq}$ is linearizable with respect to $S_{\neq}$.
\end{restatable}


%\section{Step-by-Step Linearizability of Extended Priority Queues}
\label{sec:step-by-step linearizability of extended priority queues}

In this section we shows that, with the help of a property called step-by-step linearizability, we can partition the concurrent executions which are not linearizable with respect to $\textit{EPQ}$ into a finite number of classes. Intuitively, each such class represents a set of sequential execution that violate one rule of extended priority queue. Here step-by-step linearizability enables us to build a linearization for an execution $e$ incrementally, using linearizations of projections of $e$. Our step-by-step linearizability is inspired by the step-by-step linearizabilility of queue and stacks in \cite{Bouajjani:2015}. The proof of lemmas in this section can be found in Appendix \ref{sec:appendix in section step-by-step linearizability of extended priority queues}.

The projection $e \vert{\mathcal{D}}$ of a sequential execution $e$ into a subset $\mathcal{D} \subseteq \mathbb{D}$ of items is obtained from $e$ by erasing all method events with a data value not in $\mathcal{D}$. The set of projections of $e$ is denoted $\textit{proj}(e)$. When refer to $\textit{proj}(e)$, we implicitly assume that each $\textit{rm}(\textit{empty})$ in $e$ has a ghost argument that is unique. We write $e \setminus x$ for the projection $e \vert_{ \mathbb{D} \setminus \{ x \} }$. This extends naturally to histories and concurrent executions.

A set $S$ of sequential executions is closed under projection, if for all $\mathcal{D} \subseteq \mathbb{D}$ and $e \in S$, we have $e \vert_{ \mathcal{D} } \in S$. The following lemma states that $\textit{EPQ}$ is closed under projection, since the predicates used in rules of extended priority queue are ``closed under projection''.

\begin{restatable}{lemma}{EPQisClosedUnderProjection}
\label{lemma:EPQ is closed under projection}
$\textit{EPQ}$ is closed under projection.
\end{restatable}

A sequential execution $e$ matches a rule $R$ of extended priority queue, if $e=\textit{Expr}(l_1, \ldots, l_k,a,b)$, and $\textit{Guard}(l_1, \ldots, l_k,a,b)$ holds. Here $\textit{Guard}$ and $\textit{Expr}$ are of rule $R$, and we call $a$ (if exists) the witness of $e$. We denote by $\textit{MS}(R)$ the set of sequential executions which match $R$. Note that sequences in $\textit{MS}(R)$ only respect rule $R$ and may be not in $\textit{EPQ}$. $e$ is linearizable with respect to $\textit{MS}(R)$ with witness $x$, if $e$ is linearizable with respect to $u \in \textit{MS}(R)$ and $x$ is the witness of $u$.

\begin{example}\label{example:match set of R}
Assume that $p_1 <_{\mathbb{P}} p_2$, we can see that $e = \textit{rm}(b) \cdot \textit{put}(b,p_1) \cdot \textit{put}(a,p_2) \cdot \textit{rm}(a)$ is in $\textit{MS}(\textit{EPQ}_1^{>} )$, but it is obvious that $e \notin \textit{EPQ}$.
\end{example}

The following lemma states that for data-differentiated sequential execution, checking inclusion into $\textit{EPQ}$ is equivalent to checking inclusion into $\textit{MS}(R)$ for everyone of its projections.

\begin{restatable}{lemma}{EPQasMultiInMRpriforSequence}
\label{lemma:EPQ as multi in MRpri for sequence}
Given a data-differentiated sequential execution $e$, $e \in \textit{EPQ}$, if and only if, $\forall e' \in \textit{proj}(e)$ and $\forall R \in \textit{last}(e')$, we have $e' \in \textit{MS}(R)$.
\end{restatable}

Lemma \ref{lemma:EPQ as multi in MRpri for sequence} simplifies checking inclusion into $\textit{EPQ}$, since checking $\textit{MS}(R)$ only concerns information of one rule, while checking $\textit{EPQ}$ need to consider every method events. We want a similar lemma for checking linearizability with respect to $\textit{EPQ}$. To enable such equivalent characterization, we introduce the notion of step-by-step linearizability for extended priority queue. The projection $e \vert{O}$ of a concurrent execution $e$ into a set $O$ of operations is obtained from $e$ by erasing all call and return actions of non-$O$ operations. We write $e \setminus o$ for the projection $e \vert_{ O_h \setminus \{ o \} }$, where $O_h$ is the set of operations of $e$. This extends naturally to histories. Similarly we can define projection into method events.

\begin{definition}\label{def:step-by-step linearizability of extended priority queue}
A set $S$ of sequential executions of extended priority queue is step-by-step linearizable, if for any data-differentiated execution $e$,
\begin{itemize}
\setlength{\itemsep}{0.5pt}
\item[-] if $e$ is linearizable w.r.t. $\textit{MS}(R)$ ($R \in \{ \textit{EPQ}_1^{>}, \textit{EPQ}_1^{=}, \textit{EPQ}_2^{>},\textit{EPQ}_2^{=} \}$) with witness $x$, we have: $e \setminus x \sqsubseteq \textit{EPQ} \Rightarrow e \sqsubseteq \textit{EPQ}$.

\item[-] if $e$ is linearizable w.r.t. $\textit{MS}(\textit{EPQ}_3)$ and $o$ is a $\textit{rm}(\textit{empty})$ event, we have:

$e \setminus o \sqsubseteq \textit{EPQ} \Rightarrow e \sqsubseteq \textit{EPQ}$.
\end{itemize}
\end{definition}

Given a data-differentiated execution and its history, we can abuse notation and mix labels and method events with operations themselves, since items are unique in a data-differentiated execution. For instance, we will reference an operation labeled by $\textit{put}(p,a)$ as $\textit{put}(p,a)$. The following lemma states that $\textit{EPQ}$ is step-by-step linearizability.

\begin{restatable}{lemma}{EPQueueisStepByStepLinearizability}
\label{lemma:EPQ is step-by-step linearizability}
$\textit{EPQ}$ is step-by-step linearizability.
\end{restatable}

Let us briefly explain the idea of proving step-by-step linearizability of $\textit{EPQ}_1$ with an example, while the other two rules is much simpler to deal with. Given a data-differentiated concurrent execution $e \sqsubseteq l \in \textit{MS}(\textit{EPQ}_1)$ with witness $x$ and assume that $e \setminus x \sqsubseteq l' \in \textit{EPQ}$, we explicitly construct a sequence $l''$ and prove that $e \sqsubseteq l'' \in \textit{EPQ}$. The construction of $l''$ is as follows:

\begin{itemize}
\setlength{\itemsep}{0.5pt}
\item[-] In \figurename~\ref{fig:concurrent execution for EPQ1}, we give an example of $e$, and we explicitly draw the linearization points according to $l'$. Assume that $p_1 <_{\mathbb{P}} p_2$ and $p_1 <_{\mathbb{P}} p_3$. According to $\textit{EPQ}_1$, there exists $u$, $v$ and $w$, such that $l = u \cdot \textit{put}(x,\textit{pri}_x) \cdot v \cdot \textit{rm}(x) \cdot w$. We use $\textit{put}(z_2,p_3)-w$ to denote that $\textit{put}(z_2,p_3)$ is in $w$.

\item[-] Let $O_c$ and $O_i$ be the set of operations with priorities comparable with $\textit{pri}_x$ and incomparable with $\textit{pri}_x$ in $l$, respectively. It seems that we can make $l''= l''_1 \cdot \textit{put}(x,\textit{pri}_x) \cdot l''_2 \cdot \textit{rm}(x) \cdot l''_3$, where $l''_1$, $l''_2$ and $l''_3$ are sequences of operations chosen from $u$, $v$ and $w$, respectively. However, this is incorrect. The reason is that there is no restriction to $O_i$ operations in $u \cdot v$, and thus, there is no guarantee that the projection of $l'$ into $O_i$ operations in $u \cdot v$ being correct. In our example, we can see that such projection is $\textit{put}(z_1,p_3) \cdot \textit{rm}(z_2)$ and is incorrect.

\item[-] Let $U$, $V$ and $W$ be the set of operations in $u$, $v$ and $w$, respectively. We use a two-steps method to construct $U'$, $V'$ and $W'$, which are the new version of $U$, $V$ and $W$.

\item[-] The first step is to obtain $U' \cup V'$ and $W'$. We leave $O_c$ operations unchanged for $U \cdot V$ and $W$. We try to put as many as possible $O_i$ operations into $U' \cup V'$ (regardless whether they belong to $U \cdot V$ or $W$), until we meet a $O_i$ operation $o$ that happens before a $O_c$ operation in $W$. We move $O_i$ operations that before $o$ in $l'$ into $U' \cdot V'$, and move $o$ and $O_i$ operations that after $o$ in $l'$ into $W'$. In this example, $o$ is $\textit{put}(x_3,p_2)$ (emphasized by adding vertical dashed line), we move $\textit{put}(z_1,p_3)$ and $\textit{put}(z_2,p_3)$ into $U' \cdot V'$, and move $\textit{rm}(z_1)$ and $\textit{rm}(z_2)$ into $W'$ (emphasized by adding boxes).

\item[-] The second step is to obtain $U'$ and $V'$. $U'$ contains the following two kinds of operations: (1) Operations whose linearization points are before $\textit{ret}(\textit{put},x)$, and (2) other $\textit{put}$ operations with priority $\textit{pri}_x$. $V'$ contains the remanning operations. In this example, $U'$ contains $\textit{put}(z_1,p_3)$ and $\textit{put}(x_2,p_2)$.

\item[-] $l''_1$ is the concatenation of projection of $l'$ into the first and second kinds of operations in $U'$. $l''_2$ and $l''_3$ are projection of $l'$ into $V'$ and $W'$, respectively. In this example, $l''_1 = \textit{put}(z_1,p_3) \cdot \textit{put}(x_2,p_2)$, $l''_2 = \textit{put}(z_2,p_3) \cdot \textit{rm}(x_2) \cdot \textit{put}(y_1,p_1) \cdot \textit{rm}(y_1)$, and $l''_3 = \textit{put}(x_3,p_2) \cdot \textit{rm}(z_1) \cdot \textit{rm}(z_2)$. In \figurename~\ref{fig:concurrent execution with new linearization points for EPQ1}, we add linarization points according to $l''$, and we can see that $l''$ holds as required.
\end{itemize}

\begin{figure}[htbp]
  \centering
  \includegraphics[width=0.5 \textwidth]{figures/PIC-HIS-EPQ1.pdf}
%\vspace{-10pt}
  \caption{Concurrent execution $e$}
  \label{fig:concurrent execution for EPQ1}
\end{figure}


\begin{figure}[htbp]
  \centering
  \includegraphics[width=0.5 \textwidth]{figures/PIC-HIS-EPQ1-NEWLP.pdf}
%\vspace{-10pt}
  \caption{Concurrent execution $e$ with linearization points according to $l''$}
  \label{fig:concurrent execution with new linearization points for EPQ1}
\end{figure}


The following lemma states that for data-differentiated execution, checking linearizability with respect to $\textit{EPQ}$ is equivalent to checking linearizability with respect to $\textit{MS}(R)$ for everyone of its projections. Roughly speaking, step-by-step linearizability of extended priority queue play a important rule for proof of the $\textit{if}$ direction of Lemma \ref{lemma:EPQ as multi in MRpri for history}: It guide us how to build a linearization of a whole execution by increasingly construct linearization of sub-execution from $\epsilon$ execution.

\begin{restatable}{lemma}{EPQasMultiInMRpriforHistory}
\label{lemma:EPQ as multi in MRpri for history}
Given a data-differentiated execution $e$, $e \sqsubseteq \textit{EPQ}$, if and only if, $\forall e' \in \textit{proj}(e)$ and $\forall R \in \textit{last}(e')$, we have $e' \sqsubseteq \textit{MS}(R)$.
\end{restatable}



%!TEX root = draft.tex
\section{Reducing Linearizability of Priority Queues to Reachability}
\label{sec:co-regular of extended priority queues}

We show that the set of executions for which $\textit{Check-PQ-Conc-NonRec}$ fails on some projection can be described using register automata, modulo a value renaming. Renaming values (which is complete under data independence) allows to simplify the reasoning about projections. W.l.o.g., we assume that all the operations which are not in the projection failing this test use the same distinguished value $\top$, different from those in the projection. Then, it is enough to find an automata characterization of the executions $e$ for which $\textit{Check-PQ-Conc-NonRec}(e)$ is $\mathsf{false}$, i.e., for which
$\Gamma(e) := \mathsf{Has\text{-}}\Gamma(e) \Rightarrow \exists \alpha.\ \Gamma\mathsf{\text{-}Conc}(e,\alpha)$ is false, for some $\Gamma\in \{\mathsf{EmptyRemove}$, $\mathsf{UnmatchedMaxPriority}$, $\mathsf{MatchedMaxPriority}\}$.
Intuitively, $\Gamma(e)$ states that $e$ is linearizable w.r.t. the set of sequential executions described by $\Gamma\mathsf{\text{-}Seq}$ (provided that $\mathsf{Has\text{-}}\Gamma(e)$ holds). Therefore, by an abuse of terminology, an execution $e$ satisfying $\Gamma(e)$ is called \emph{linearizable w.r.t. $\Gamma$}, or \emph{$\Gamma$-linearizable}.
Extending the automaton characterizing non $\Gamma$-linearizable executions with self-loops that allow any operation with argument $\top$ results in an automaton satisfying the following property called \emph{$\Gamma$-completeness}.

\begin{definition}
For $\Gamma\in \{\mathsf{EmptyRemove}$, $\mathsf{UnmatchedMaxPriority}$, $\mathsf{MatchedMaxPriority}\}$, an automaton $A$ is called \emph{$\Gamma$-complete} when for each data-independent implementation $\mathcal{I}$:

$A \cap \mathcal{I} \neq \emptyset$ iff there exists $e \in \mathcal{I}$ and $e' \in \textit{proj}(e)$ such that $e'$ is not $\Gamma$-linearizable.
\end{definition}




We can show that for any $\Gamma\in \{\mathsf{EmptyRemove}$,$\mathsf{UnmatchedMaxPriority}$,$\mathsf{MatchedMaxPriority}\}$ there exists a $\Gamma$-complete automaton. For lack of space, we only consider the case $\Gamma=\mathsf{MatchedMaxPriority}$ in
Section~\ref{ssec:aut}.
When defining $\Gamma$-complete automata, we assume that every implementation $\mathcal{I}$ behaves correctly, i.e., as a FIFO queue, when only values with the same priority are observed. More precisely, we assume that for every execution $e\in\mathcal{I}$ and every priority $p\in\mathbb{P}$, the projection of $e$ to values with priority $p$ is linearizable (w.r.t. $\seqPQ$). This property can be checked separately using register automata similar to the automata in~\cite{DBLP:conf/icalp/BouajjaniEEH15} describing FIFO queue violations (see Appendix~\ref{sec:appendix lemma and register automata for FIFO of single-priority executions} for more details). This assumption excludes some obvious violations, such as an $\textit{rm}(a)$ operation happening before a $\textit{put}(a,p)$ operation, for some $p$.

For $\Gamma\in \{\mathsf{UnmatchedMaxPriority}, \mathsf{MatchedMaxPriority}\}$, we consider $\Gamma$-complete automata recognizing executions which contain only one maximal priority. This is w.l.o.g. because any data-differentiated execution for which $\Gamma(e)$ is false has such a projection.
Formally, given a data-differentiated execution $e$ and $p$ a maximal priority in $e$, $e\vert_{\preceq p}$ is the projection of $e$ to the set of values with priorities smaller or equal to $p$. Then,

\begin{lemma}
\label{lemma:pri execution is enough}
For $\Gamma\in \{\mathsf{UnmatchedMaxPriority}, \mathsf{MatchedMaxPriority}\}$, a data-differentiated execution $e$
 is $\Gamma$-linearizable iff $e\vert_{\preceq p}$ is $\Gamma$-linearizable for some maximal priority $p$ in $e$.
\end{lemma}
\begin {proof} (Sketch)
For the ``only-if'' direction, let $e$ be a data-differentiated execution linearizable w.r.t. $l = u \cdot \textit{put}(x,p) \cdot v \cdot \textit{rm}(x) \cdot w$ s.t. $\mathsf{MatchedMaxPriority}\mathsf{\text{-}Seq}(l,x)$ holds. Since the predicate $\mathsf{MatchedMaxPriority}\mathsf{\text{-}Seq}(l,x)$ imposes no restriction on the operations in $u$, $v$, and $w$ with priorities incomparable to $p$, erasing all these operations results in a sequential execution which still satisfies this predicate. Similarly, for $\Gamma=\mathsf{UnmatchedMaxPriority}$.

The ``if'' direction follows from the fact that if the projection of an execution to a set of operations $O_1$ has a linearization $l_1$ and the projection of the same execution to the remaining set of operations has a linearization $l_2$, then the execution has a linearization which is defined as an interleaving of $l_1$ and $l_2$ (see Appendix~\ref{sec:appendix proof of Lemma pri execution is enough} for more details).

Thus, let $e$ be an execution such that $e\vert_{\preceq p}$ is linearizable w.r.t. $l = u \cdot \textit{put}(x,p) \cdot v \cdot \textit{rm}(x) \cdot w$ where $\mathsf{MatchedMaxPriority}\mathsf{\text{-}Seq}(l,x)$ holds. By the property above, we know that $e$ has a linearization $l' = u' \cdot \textit{put}(x,p) \cdot v' \cdot \textit{rm}(x) \cdot w'$, such that the projection of $l'$ to values of priority comparable to $p$ is $l$.
Since $\mathsf{MatchedMaxPriority}\mathsf{\text{-}Seq}(l,x)$ doesn't constrain the values of priority incomparable to $p$, we obtain that $\mathsf{MatchedMaxPriority}\mathsf{\text{-}Seq}(l',\alpha)$ also holds.
\end {proof}

We shows in Appendix \ref{subsec:appendix co-regular of EPQ2Lar} and Appendix \ref{subsec:appendix co-regular of EPQ2Equal} that it is safe to ignore constructing $\mathsf{UnmatchedMaxPriority}$-complete register automata, and we give the $\mathsf{EmptyRemove}$-complete register automata in Appendix \ref{subsec:co-regular of EPQ3}. The existence of $\Gamma$-complete automata enable an effective reduction of checking linearizability of concurrent priority queue implementations to state reachability.
Section~\ref{subsec:combine step-by-step linearizability and co-regular} discusses decidability results implied by this reduction.

\begin{theorem}
\label{lemma:reduce EPQ into state reachability}
Let $\mathcal{I}$ be a data-independent implementation. Then, there is a $\Gamma$-complete automaton $A(\Gamma)$ for each $\Gamma$. Moreover,
$\mathcal{I} \sqsubseteq \seqPQ$ iff $\mathcal{I} \cap A(\Gamma) = \emptyset$ for all $\Gamma$.
\end{theorem}



\subsection{A $\mathsf{MatchedMaxPriority}$-complete automaton}\label{ssec:aut}


A differentiated execution $e$ is not $\mathsf{MatchedMaxPriority}$-linearizable when all the $\textit{put}$ operations in $e$ using the maximal priority $p$ are matched, and $e$ is not linearizable w.r.t. the set of sequential executions satisfying $\mathsf{MatchedMaxPriority\text{-}Seq}(e,x)$ for each value $x$ of priority $p$. We consider two cases depending on whether $e$ contains exactly one value with priority $p$ or at least two values. We denote by $\mathsf{MatchedMaxPriority}^{>}$ the strengthening of $\mathsf{MatchedMaxPriority}$ with the condition that all the values other than $x$ have a priority strictly smaller than $p$ (corresponding to the first case), and by $\mathsf{MatchedMaxPriority}^{=}$ the strengthening of the same formula with the negation of this condition (corresponding to the second case).
We use particular instances of register automata~\cite{DBLP:journals/tcs/KaminskiF94,DBLP:conf/icalp/Cerans94,DBLP:conf/stacs/SegoufinT11} whose states include only two registers, one for storing a priority guessed at the initial state, and one for storing the priority of the current action in the execution. The transitions can check equality or the order relation $\prec$ between the values stored in the two registers. Instead of formalizing the full class of register automata, we consider a simpler class which suffices our needs. Thus, we consider a class of labeled transition systems whose states consist of a finite control part and a register $r$ interpreted to elements of $\mathbb{P}$. The transition labels are:
\begin{itemize}
	\item $r=*$ for storing an arbitrary value to $r$,
	\item $\textit{call}(\textit{rm},a)$ and $\textit{ret}(\textit{rm},a)$ for reading call/return actions of a remove,
	\item $\textit{call}(\textit{put},d,g)$ where $g\in\{=r,\prec r,true\}$ is a guard, for reading a call action $\textit{call}(\textit{put},d,p)$ and checking if $p$ is either equal to or smaller than the value stored in $r$, or arbitrary,
	\item $\textit{ret}(\textit{put},d,true)$ for reading a return action $\textit{ret}(\textit{put},d,p)$ for any $p$.
\end{itemize}
The set of sequences (executions) accepted by such a transition system is defined as usual.





\subsubsection{A $\mathsf{MatchedMaxPriority}^>$-complete automaton}
\label{subsec:co-regular of EPQ1Lar}

\begin{figure}[t]
  \centering
  \includegraphics[width=0.4 \textwidth]{figures/PIC-HIS-INTRO-GAP-EPQ1L.pdf}
  \caption{An execution that is not $\mathsf{MatchedMaxPriority}^{>}$-linearizable. We represent each operation as a time interval whose left, resp., right, bound corresponds to the call, resp., return action.}
  \label{fig:introduce gap for EPQ1Lar}
\end{figure}






Figure~\ref{fig:introduce gap for EPQ1Lar} contains a typical example of an execution $e$ which is not $\mathsf{MatchedMaxPriority}^>$-linearizable,
where $p_1 \prec p_4$, $p_2 \prec p_4$, and $p_3 \prec p_4$.
Intuitively, this is a violation because during the whole execution of $\textit{rm}(b)$, the priority queue stores a smaller priority value (which should be removed before $b$). To be more precise, we define \emph{the interval of a value $x$} as the time interval from the return of a put $\textit{ret}(\textit{put},x,p)$ to the call of the matching remove $\textit{call}(rm,x)$, or to the end of the execution if such a call action doesn't exist. This represents the time interval in which a value is guaranteed to be stored into the priority queue. Concretely, for a standard indexing of actions in an execution, a time interval is a closed interval between the indexes of two actions in the execution.
In \figurename~\ref{fig:introduce gap for EPQ1Lar}, the interval of each value of priority smaller than $p_4$ is pictured as a dashed line. There is no sequence $l$ s.t. $e \sqsubseteq l$ and $\mathsf{MatchedMaxPriority}\mathsf{\text{-}Seq}(l,b)$ hold, since each time point from $\textit{call}(\textit{rm},b)$ to $\textit{ret}(\textit{rm},b)$ is included in the interval of a smaller priority value,
and $\textit{rm}(b)$ can't take effect in the interval of a smaller priority value.
To formalize this scenario we use the notion of \emph{left-right constraint} defined below.



\begin{definition}\label{def:left-right constraint for matched put and rm operations}
Let $e$ be a data-differentiated execution which contains only one maximal priority $p$, and only one value $x$ of priority $p$ (and no $\textit{rm}(\textit{empty})$ operations).
The \emph{left-right constraint of $x$} is the graph $G$ where:
\begin{itemize}
\item the nodes are the values occurring in $e$,
\item there is an edge from $d_1$ to $x$, if $\textit{put}(d_1,\_) <_{\textit{hb}} \textit{put}(x,p)$ or $\textit{put}(d_1,\_) <_{\textit{hb}} \textit{rm}(x)$,
\item there is an edge from $x$ to $d_1$, if $\textit{rm}(x)<_{\textit{hb}}\textit{rm}(d_1)$ or $\textit{rm}(d_1)$ does not exists,
\item there is an edge from $d_1$ to $d_2$, if $\textit{put}(d_1,\_) <_{\textit{hb}} \textit{rm}(d_2,\_)$.
\end{itemize}
\end{definition}

The execution in \figurename~\ref{fig:introduce gap for EPQ1Lar} is not $\mathsf{MatchedMaxPriority}^>$-linearizable because the left-right constraint of the maximal priority value $b$ contains the cycle $f \rightarrow d \rightarrow c \rightarrow b \rightarrow f$. The following lemma states that the presence of such a cycle is equivalent to \emph{non} $\mathsf{MatchedMaxPriority}^>$-linearizability, and it is proved in Appendix \ref{sec:appendix proof and definition in section co-regular of EPQ1Lar}: 

\begin{lemma}
\label{lemma:Lin Equals Constraint for EPQ1Lar}
Let $e$ be a data-differentiated execution such that
$\mathsf{Has\text{-}MatchedMaxPriority}(e)$ holds, $p$ is the maximal priority in $e$, and $\textit{put}(x,p)$ and $\textit{rm}(x)$ are only operations with arguments of priority $p$ in $e$.
Then, $e$ is $\mathsf{MatchedMaxPriority}$-linearizable iff the left-right constraint of $x$ contains no cycle going through $x$.
\end{lemma}

When the left-right constraint contains a cycle $d_1 \rightarrow \ldots \rightarrow d_m \rightarrow x \rightarrow d_1$, for some $d_1$,$\ldots$,$d_n\in \mathbb{D}$, we say that $x$ is \emph{covered} by $d_1,\ldots,d_m$. The shape of an execution witnessing such a cycle (i.e., the alternation between call/return actions) can be identified using our class of automata, the only complication being the unbounded number of values $d_1$,$\ldots$,$d_n$. However, by data independence, whenever an implementation contains such an execution it also contains an execution where all the values $d_1$,$\ldots$,$d_n$ are renamed to the same value $a$, and $x$ is renamed to $b$. Therefore, our automata can be defined over a fixed set of values $a$, $b$, and $\top$ (recall that $\top$ is used for operations outside of the non-linearizable projection).

To define a $\mathsf{MatchedMaxPriority}^>$-complete automaton, we need to consider all the possible orders between the call/return actions of the $\textit{put}$/$\textit{rm}$ operations that add and respectively, remove the value $b$. The case where the put happens-before the remove (as in \figurename~\ref{fig:introduce gap for EPQ1Lar}) is pictured in \figurename~\ref{fig:automata APQ1Lar-1 in paper}. This automaton captures the three possible ways of ordering the first action $\textit{ret}(\textit{put},a,\_)$ w.r.t. the actions with value $b$, which are pictured in \figurename~\ref{fig:executions APQ1Lar-1 in paper}(a) (this action cannot occur after $\textit{call}(\textit{rm},b,\_)$ since $b$ must be covered by the $a$-s). The paths corresponding to these three possible orders are: $q_1 \rightarrow q_2 \rightarrow q_3 \ldots \rightarrow q_7$, $q_1 \rightarrow q_2 \rightarrow q_3 \ldots \rightarrow q_{10}$, and $q_1 \rightarrow q_9 \rightarrow q_{10} \ldots \rightarrow q_7$. \figurename~\ref{fig:executions APQ1Lar-1 in paper} lists the four possible orderings of the call/return actions of adding and removing $b$, and also possible orders of the first $\textit{ret}(\textit{put},a,\_)$ w.r.t the actions with value $b$. Each such ordering corresponds to an automaton similar to the one in \figurename~\ref{fig:automata APQ1Lar-1 in paper}, their union defining a $\mathsf{MatchedMaxPriority}^>$-complete automaton. In Appendix \ref{sec:appendix proof and definition in section co-regular of EPQ1Lar}, three register automata is constructed according to the cases of \figurename~\ref{fig:executions APQ1Lar-1 in paper} (b), (c) and (d), respectively.




\begin{figure}[t]
  \centering
  \includegraphics[width=.8\textwidth]{figures/PIC_HIS_PQ1Lar-fouCase.pdf}
  \caption{Orderings to be considered when defining a $\mathsf{MatchedMaxPriority}^>$-complete automaton.}
  \label{fig:executions APQ1Lar-1 in paper}
\end{figure}

\begin{figure}[t]
  \centering
  \includegraphics[width=.9\textwidth]{figures/PIC_AUTO_PQ1Lar-pprr.pdf}
  \caption{A register automaton capturing the scenario in Figure~\ref{fig:executions APQ1Lar-1 in paper}(a). We use the following notations: $A_1 = A \cup \{ \textit{ret}(\textit{rm},a) \}$, $A_2 = A \cup \{ \textit{call}(\textit{put},a,=r) \}$, $A_3 = A_2 \cup \{ \textit{ret}(\textit{rm},a) \}$, where $A = \{ \textit{call}(\textit{put},\top,\textit{true}),\textit{ret}(\textit{put},\top,\textit{true}), \textit{call}(\textit{rm},\top)$, $\textit{ret}(\textit{rm},\top),\textit{call}(\textit{rm},\textit{empty}),\textit{ret}(\textit{rm},\textit{empty}) \}$.}
  \label{fig:automata APQ1Lar-1 in paper}
\end{figure}





\subsubsection{A $\mathsf{MatchedMaxPriority}^=$-complete automaton}
\label{subsec:co-regular of EPQ1Equal}

When an execution contains at least two values of maximal priority, the acyclicity of the left-right constraints (for all the maximal priority values) is not enough to conclude that the execution is $\mathsf{MatchedMaxPriority}$-linearizable.
Intuitively, there may exist a value $a$ which is added before another value $b$ such that all the possible linearization points of $\textit{rm}(b)$ are disabled by the position of $\textit{rm}(a)$ in the happens-before. We give an example of such an execution $e$ in \figurename~\ref{fig:introduce pb order}, where $p_1\prec p_4$. This execution  is not linearizable w.r.t. $\mathsf{MatchedMaxPriority}$ (or $\mathsf{MatchedMaxPriority}^{=}$) even if
neither $a$ nor $b$ are covered by values with smaller priority.
Since $\textit{put}(a,p_4) <_{\textit{hb}} \textit{put}(b,p_4)$ and values of the same priority are removed in FIFO order, $\textit{rm}(a)$ should be linearized before $\textit{rm}(b)$ (i.e., this execution should be linearizable w.r.t. a sequence where $\textit{rm}(a)$ occurs before $\textit{rm}(b)$).
Since $\textit{rm}(b)$ cannot take effect during the interval of a smaller priority value, it could be only linearized in one of the two time intervals pictured with dotted lines in \figurename~\ref{fig:introduce pb order}. However, each of those time intervals ends before $\textit{call}(\textit{rm},a)$, and thus $\textit{rm}(a)$ cannot be linearized before $\textit{rm}(b)$.

\begin{figure}[t]
  \centering
  \includegraphics[width=0.5 \textwidth]{figures/PIC-HIS-INTRO-PB-ORDER-EPQ.pdf}
  \caption{An execution that is not $\mathsf{MatchedMaxPriority}^=$-linearizable.}
  \label{fig:introduce pb order}
\end{figure}

To recognize the scenarios in \figurename~\ref{fig:introduce pb order}, we introduce an order $<_{\textit{pb}}$ between values which intuitively, can be thought of as ``a value $a$ is put before another value $b$''.
More precisely, given a data-differentiated execution $e$ and two values $a$ and $b$ of maximal priority, $a <_{\textit{pb}} b$ if one of the following holds: (1) $\textit{put}(a,\_) <_{\textit{hb}} \textit{put}(b,\_)$, (2) $\textit{rm}(a) <_{\textit{hb}} \textit{rm}(b)$, or (3) $\textit{rm}(a) <_{\textit{hb}} \textit{put}(b,\_)$. Sometimes we use $a <_{\textit{pb}}^A b$, $a <_{\textit{pb}}^B b$, and $a <_{\textit{pb}}^C b$ to explicitly distinguish between these three cases. Let $<_{\textit{pb}}^*$ be the transitive closure of $<_{\textit{pb}}$.

To define the time intervals in which a remove like $\textit{rm}(b)$ in \figurename~\ref{fig:introduce pb order}, can be linearized (outside of intervals of smaller priority values) we use the notion of gap-point. As before, defining time intervals relies on an indexing of actions in an execution, starting with 0.

\begin{definition}\label{def:gap-point for matched put and rm operations}
Let $e$ be a data-differentiated execution with only one maximal priority $p$, and $\textit{put}(x,p)$ and $\textit{rm}(x)$ two operations in $e$. An index $i\in [0,|e|-1]$ is a \emph{gap-point of $x$} if $i$ is greater than or equal to the index of both $\textit{call}(\textit{put},x,p)$ and $\textit{call}(\textit{rm},x)$, smaller than the index of $\textit{ret}(\textit{rm},x)$, and not included in the interval of a value with priority smaller than $p$.
\end{definition}

The case of \figurename~\ref{fig:introduce pb order} can be formally described as follows: $a <_{\textit{pb}}^* b$ while the right-most gap-point of $b$ is before $\textit{call}(\textit{rm},a)$ or $\textit{call}(\textit{put},a,p_4)$. The following lemma states that these conditions are enough to characterize non-linearizability w.r.t. $\mathsf{MatchedMaxPriority}^{=}$. It is proved in Appendix~\ref{sec:appendix proof and definition in section co-regular of EPQ1Equal}. 

\begin{lemma}
\label{lemma:EPQ1Equal as pb order and gap-point}
Let $e$ be a data-differentiated execution with only one maximal priority $p$ such that $\mathsf{Has\text{-}MatchedMaxPriority}(e)$ holds.
Then, $e$ is not $\mathsf{MatchedMaxPriority}^{=}$-linearizable iff $e$ contains two values $x$ and $y$ of maximal priority $p$ such that $y <_{\textit{pb}}^* x$, and the rightmost gap-point of $x$ is strictly smaller than the index of $\textit{call}(\textit{put},y,p)$ or $\textit{call}(\textit{rm},y)$.
\end{lemma}


The following shows that the number of values needed to witness that $y <_{\textit{pb}}^* x$, for some $x$ and $y$, is bounded.

\begin{lemma}
\label{lemma:ob order has bounded length}
Let $e$ be a data-differentiated execution such that $a <_{\textit{pb}} a_1 <_{\textit{pb}} \ldots <_{\textit{pb}} a_m <_{\textit{pb}} b$ holds for some set of values $a$, $a_1$,$\ldots$,$a_m$, $b$. Then, one of the following holds:
\begin{itemize}
\item[-] $a <_{\textit{pb}}^A b$, $a <_{\textit{pb}}^B b$, or $a <_{\textit{pb}}^C b$,

\item[-] $a <_{\textit{pb}}^A a_i <_{\textit{pb}}^B b$ or $a <_{\textit{pb}}^B a_i <_{\textit{pb}}^A b$, for some $i$.
\end{itemize}
\end{lemma}

\begin{figure}[t]
  \centering
  \includegraphics[width=0.8 \textwidth]{figures/PIC-HIS-FiveEnumerations.pdf}
  \caption{Orderings to be considered when defining a $\mathsf{MatchedMaxPriority}^=$-complete automaton. We omit the operations which can be ordered arbitrarily, e.g., $\textit{put}(b)$ in the cases (d) and (e).}
  \label{fig:five enumerations}
\end{figure}

To characterize violations to $\mathsf{MatchedMaxPriority}^{=}$-linearizability, one has to consider all the possible orders between call/return actions of the operations on values $a$, $b$, and $a_i$ in Lemma \ref{lemma:ob order has bounded length}, and the right-most gap point of $b$. Excluding the inconsistent cases, we are left with the five orders in \figurename~\ref{fig:five enumerations}, where $o$ denotes the rightmost gap-point of $b$ (see Lemma \ref{lemma:five enumeration is enough for EPQ1Equal} in Appendix~\ref{sec:appendix proof and definition in section co-regular of EPQ1Equal}).
For each case, we define an automaton recognizing the induced set of violations. The register automaton for the case in \figurename~\ref{fig:five enumerations}(a) is shown in \figurename~\ref{fig:an enumeration and its witness automaton}. In this case, Lemma~\ref{lemma:EPQ1Equal as pb order and gap-point} is equivalent to the fact that intuitively, the time interval from $\textit{call}(\textit{rm},a)$ to $\textit{ret}(\textit{rm},b)$ is covered by lower priority values (thus, there is no gap-point of $b$ which occurs after $\textit{call}(\textit{rm},a)$). By data-independence, these lower priority values can be renamed to a fixed value $c$. In Appendix~\ref{sec:appendix proof and definition in section co-regular of EPQ1Equal}, we give the detailed construction for register automata for all five orders in \figurename~\ref{fig:five enumerations}. 


\begin{figure}[t]
  \centering
  \includegraphics[width=0.65 \textwidth]{figures/PIC-WitnessAutomata-For1.pdf}
  \caption{A register automaton for the case in Figure~\ref{fig:five enumerations}(a), where $A_1 = A \cup \{ \textit{call}(\textit{put},c,<r)\}$, $A_2 = A_1 \cup \{ \textit{ret}(\textit{put},b,=r) \}$, $A_3 = A_2 \cup \{ \textit{ret}(\textit{rm},c) \}$, $A_4 = A \cup \{ \textit{ret}(\textit{put},b,=r), \textit{ret}(\textit{rm},c) \}$, where $A = \{ \textit{call}(\textit{put},\top,\textit{true})$,$\textit{ret}(\textit{put},\top,\textit{true})$,$\textit{call}(\textit{rm},\top)$,$\textit{ret}(\textit{rm},\top)$,$\textit{call}(\textit{rm},\textit{empty})$,$\textit{ret}(\textit{rm},\textit{empty})\}$.}
  \label{fig:an enumeration and its witness automaton}
\end{figure}



\subsection{Decidability Result}
\label{subsec:combine step-by-step linearizability and co-regular}

We describe a class $\mathcal{C}$ of data-independent implementations for which linearizability w.r.t. $\seqPQ$ is decidable. The implementations in $\mathcal{C}$ allow an unbounded number of values but a bounded number of priorities. Each method has a finite set of local variables storing Boolean values or values from $\mathbb{D}$. Methods communicate through a finite number of shared variables interpreted also as Booleans or values from $\mathbb{D}$. To ensure data independence, values in $\mathbb{D}$ may be assigned, but never used in Boolean expressions (e.g., of if-then-else statements). This class captures typical implementations, or finite-state abstractions thereof, e.g., obtained via predicate abstraction. The $\Gamma$-complete automata we define use a fixed set $D=\{a,b,c,a_1,\top\}$ of values ($a_1$ is needed to deal with the second item in Lemma~\ref{lemma:ob order has bounded length}). Therefore, for any $\Gamma$, $\mathcal{C}\cap A(\Gamma)\neq\emptyset$ iff $\mathcal{C}_D\cap A(\Gamma)\neq\emptyset$, where $\mathcal{C}_D$ is the subset of $\mathcal{C}$ that uses only values in $D$.

The set of executions $\mathcal{C}_D$ can be represented by a Vector Addition System with States (VASS). Since the set of values and priorities is bounded, each method invocation can be represented by a finite-state automaton (see~\cite{conf/esop/BouajjaniEEH13}). For a fixed set of priorities $P\subseteq \mathbb{P}$, the register automata $A(\Gamma)$ can be transformed to finite-state automata (the number of valuations of the registers is bounded). Thus, checking linearizability of an implementation in $\mathcal{C}$ is PSPACE when the number of threads is bounded, and EXPSPACE, otherwise. Moreover, reachability in VASSs can be reduced to checking linearizability of such an implementation. Essentially, given an instance of the VASS reachability problem, one can define a priority queue implementation where the $\textit{put}$ methods behave correctly and additionally, they include the code of the VASS simulation defined in~\cite{conf/esop/BouajjaniEEH13}, and the $\textit{rm}$ methods behave correctly, except for the moment where the target state is reached, in which case they trigger a linearizability violation by returning an arbitrary value. 

Based on above discussion, we have the following complexity result. The detailed proof can be found in Appendix \ref{subsec:appendix proof and definition in subsection decidability result}.

%\begin{restatable}{theorem}{ComplexityOfPriorityQueue}
%\label{theorem:complexity of priority queue}
%Verifying whether an implementation in $\mathcal{C}$ is linearizable w.r.t. $\seqPQ$ is PSPACE-complete for a fixed number of %threads, and EXPSPACE-complete otherwise.
%\end{theorem}



\begin{theorem}
\label{theorem:complexity of priority queue}
Verifying whether an implementation in $\mathcal{C}$ is linearizable w.r.t. $\seqPQ$ is PSPACE-complete for a fixed number of threads, and EXPSPACE-complete otherwise.
\end{theorem}










%!TEX root = draft.tex
\section{Related work}\label{sec:related}

The theoretical limits of checking linearizability have been investigated in previous works.
Checking linearizability of a single execution w.r.t. an arbitrary ADT is NP-complete~\cite{journals/siamcomp/GibbonsK97} while checking linearizability of all the executions
of a finite-state implementation w.r.t. an arbitrary ADT
specification (given as a regular language) is EXPSPACE-complete when the number of program
threads is bounded~\cite{journals/iandc/AlurMP00,netys-lin}, and
undecidable otherwise~\cite{conf/esop/BouajjaniEEH13}.

Existing automated methods for proving linearizability of a concurrent object
 are also based on reductions to safety
verification, e.g.,~\cite{conf/tacas/AbdullaHHJR13, conf/concur/HenzingerSV13,
conf/cav/Vafeiadis10}. The approach in~\cite{conf/cav/Vafeiadis10} considers
implementations where
operations' \emph{linearization points}
are
manually specified.
Essentially, this approach instruments the
implementation with ghost variables simulating the ADT specification at
linearization points. This approach is incomplete since not all implementations
have fixed linearization points. Aspect-oriented
proofs~\cite{conf/concur/HenzingerSV13} reduce linearizability to the
verification of four simpler safety properties. This approach has only
been applied to queues, and has not produced a fully automated
and complete proof technique. The work in~\cite{Dodds:2015:SCT:2676726.2676963} proves
linearizability of stack implementations with an automated proof assistant.
Their approach does not lead to full automation however, e.g.,~by reduction to
safety verification.

Our previous work~\cite{DBLP:conf/icalp/BouajjaniEEH15}
shows that checking linearizability of finite-state implementations of concurrent queues and stacks is decidable.
Roughly, we follow the same schema: the recursive procedure in Section~\ref{ssec:seq_exec} is similar to the inductive rules in~\cite{DBLP:conf/icalp/BouajjaniEEH15}, and its extension to concurrent executions in Section~\ref{ssec:conc_exec} corresponds to the notion of step-by-step linearizability in~\cite{DBLP:conf/icalp/BouajjaniEEH15}. Although similar in nature, defining these procedures and establishing their correctness require proof techniques which are specific to the priority queue semantics. The order in which values are removed from a priority queue is encoded in their priorities which come from an unbounded domain, and not in the happens-before order as in the case of stacks and queues. Therefore, the results we introduce in this paper cannot be inferred from those in~\cite{DBLP:conf/icalp/BouajjaniEEH15}. At a technical level, characterizing the priority queue violations requires a more expressive class of automata (with registers) than the finite-state automata in~\cite{DBLP:conf/icalp/BouajjaniEEH15}.



%\section{Relate Other Data Structures with Extended Priority Queue}
\label{sec:relate other data structures with extended priority queue}

In this section, we shows how to reduce queue and stack executions into priority queue executions in polynomial time.


\subsection{Relate Queue with Extended Priority Queue}
\label{subsec:relate queue with extended priority queue}

Given a queue execution $e_q$, we can obtain an execution $\textit{QtoEPQ}(e_q)$ of extended priority queue by (1) transforming $\textit{enq}(a)$ and $\textit{deq}(a)$ into $\textit{put}(a,p)$ and $\textit{rm}(a)$ for each $a \in \mathbb{D}$, and (2) transforming $\textit{deq}(\textit{empty})$ into $\textit{rm}(\textit{empty})$. Note that in $\textit{QtoEPQ}(e_q)$, we assign all items with a same priority $p \in \mathbb{P}$. Since we have already guarantee that each single-priority execution of priority queue has FIFO property, it is easy to see that $e_q$ is linearizability w.r.t queue, if and only if $\textit{QtoEPQ}(e_q)$ is linearizable w.r.t $\textit{EPQ}$, as stated by the following lemma:

\begin{restatable}{lemma}{RelateQueuewithEPQ}
\label{lemma:relate queue with extended priority queue}
Given an execution $e_q$ of queue, $e_q$ is linearizable w.r.t queue, if and only if $\textit{QtoEPQ}(e_q) \sqsubseteq \textit{EPQ}$.
\end{restatable}


\subsection{Relate Multi-Set with Extended Priority Queue}
\label{subsec:relate multiSet with extended priority queue}

A multi-set contains two method: $\textit{insert}$ and $\textit{delete}$. A $\textit{insert}$ method has one argument, and is used to insert an item into the multi-set. We also assume that the item is chosen from a specific (possibly infinite) data domain $\mathbb{D}$. A $\textit{delete}$ method intends to remove an arbitrary item in multi-set and then returns it. If the multi-set is empty, $\textit{delete}$ returns $\textit{empty}$. Note that when $\textit{delete}$ returns an item, there is no restriction (such as FIFO, LIFO or priorities) for how to choose this item in multi-set.

Let $\textit{MSet}$ be the set of sequential executions of multi-set. In Appendix \ref{subsec:appendix proof and definition in section relate multiset with extended priority queue}, we give inductive rules of $\textit{MSet}$, and prove that these rules are step-by-step linearizability and co-regular. We also give witness automata for $\textit{MSet}$. Here we use the notion of inductive rules, step-by-step linearizability and co-regular in \cite{Bouajjani:2015}.

Since multi-set does not add any constraint for selecting item when deleting items, we simulate such behavior by giving each item an unique priority where such priorities are pair-wise incomparable. To make it simple, we just ignore items that are putted more than one times. Such ignorance is correct due to data-independence. Formally, given a multi-set execution $e_m$, we can obtain an execution $\textit{MStoEPQ}(e_m)$ of extended priority queue by (1) erasing operations of items which are putted more than once in $e_m$, (2) transforming $\textit{insert}(a)$ and $\textit{delete}(a)$ into $\textit{put}(a,p_a)$ and $\textit{rm}(a)$ for each $a \in \mathbb{D}$, and (3) transforming $\textit{delete}(\textit{empty})$ into $\textit{rm}(\textit{empty})$. Here we need to ensure that for each $a \neq b$, $p_a$ and $p_b$ must be incomparable. Given a set $S$ of executions of multi-set, let $\textit{MStoEPQ}(S)$ be the set generated by applying $\textit{MStoEPQ}$ to each sequence in $S$. Let $\textit{Auts}_{\textit{EPQ}}$ and $\textit{Auts}_{\textit{MS}}$ be the set of witness automata for extended priority queue and for multi-set, respectively. The following lemma stats that our witness automata for extended priority queue are enough for checking violations w.r.t $\textit{MSet}$.

\begin{restatable}{lemma}{RelateMultiSetwithEPQ}
\label{lemma:relate multi set with extended priority queue}
Given an data-independent implementation $\mathcal{I}_m$ of multi-set, $\mathcal{I}_m \cap \textit{Auts}_{\textit{MS}} \neq \emptyset$, if and only if $\textit{MStoEPQ}(\mathcal{I}_m) \cap \textit{Auts}_{\textit{EPQ}} \neq \emptyset$.
\end{restatable}


\subsection{Relate Stack with Extended Priority Queue}
\label{subsec:relate stack with extended priority queue}

A stack contains two method: $\textit{push}$ and $\textit{pop}$. A $\textit{push}$ method has one argument, and is used to insert an item into the stack. A $\textit{pop}$ method intends to remove an item that is newest in stack and then returns it. If the stack is empty, $\textit{pop}$ returns $\textit{empty}$.

Given an execution $e_s$ of stack, we can obtain an execution $\textit{StoEPQ}(e_s)$ of extended priority queue by transforming $\textit{push}(a)$ and $\textit{pop}(a)$ into $\textit{put}(a,p_a)$ and $\textit{rm}(a)$, and transforming $\textit{pop}(\textit{empty})$ into $\textit{rm}(\textit{empty})$. Here let us show the process of assigning priorities for items. Given $e_s$, the process of assign priorities is as follows: 

\begin{itemize}
\setlength{\itemsep}{0.5pt}
\item[-] Let $e'$ be the projection of $e_s$ into $\textit{push}$. For each $i$, associate the $\textit{i-th}$ element of $e'$ with $e$. 

\item[-] For each $\textit{push}_o(a)$ in $e$, we assign priority $(i,j)$ to it. Here $i$ and $j$ are the number associated with the call and return action of $\textit{push}_o(a)$. 
\end{itemize}  

Then the order $<_{\mathbb{P}}$ is defined as follows: $(i_1,i_2) <_{\mathbb{P}} (j_1,j_2)$, if $j_1 > i_2$, or we can say, if the $\textit{put}$ of the former happens before the $\textit{put}$ of the latter. In \figurename~\ref{fig:execution with intervals}, we associate numbers with the call and return actions of $\textit{put}$, and then assign priorities to items. We can see that the priority of $c$ is smaller than the priority of $a$ according to $<_{\mathbb{P}}$. 

\begin{figure}[htbp]
  \centering
  \includegraphics[width=0.6 \textwidth]{figures/PIC-HIS-INTRO-Interval.pdf}
%\vspace{-10pt}
  \caption{Execution with intervals}
  \label{fig:execution with intervals}
\end{figure}

















%\FloatBarrier
  \bibliography{violin}
\bibliographystyle{plainurl}
%\bibliography{biblio_cat.bib}


%\begin{thebibliography}{50}
%
%\bibitem{Bouajjani:2015}
%Bouajjani, A., Emmi, M., Enea, C., Hamza, J.:
%\newblock On reducing linearizability to state reachability.
%\newblock In: Halld{\'{o}}rsson, M.M. et al. (eds.) ICALP 2015, Part II, pp. 95--107. Springer (2015)
%
%\bibitem{Bouajjani:2015POPL}
%Bouajjani, A., Emmi, M., Enea, C., Hamza, J.:
%\newblock Tractable Refinement Checking for Concurrent Objects.
%\newblock In: {\em POPL'15}. pp.651--662. (2015)
%
%\bibitem{Wolper:1986}
%Wolper, P.:
%\newblock Expressing interesting properties of programs in propositional
%  temporal logic.
%\newblock In {\em POPL'86}. pp.184--193. (1986)
%
%\bibitem{Henzinger:2013}
%Henzinger A. T., Sezgin A., Vafeiadis, V.:
%\newblock Aspect-Oriented Linearizability Proofs.
%\newblock In {\em CONCUR'13}. pp.242--256. (2013)
%
%
%\bibitem{Herlihy:1990}
%Herlihy, M., Wing, M, J. A.:
%\newblock Linearizability: {A} Correctness Condition for Concurrent Objects.
%\newblock In {\em {ACM} Trans. Program. Lang. Syst.}. pp.463--492. (1990)
%
%
%\bibitem{Abdulla:2013}
%Abdulla, P. A., Haziza, F., Holšªk, L., Jonsson, B., Rezine, A.:
%\newblock An integrated specification and verification technique for highly concurrent data
%structures.
%\newblock In {\em TACAS'13}. pp.324--338. (2013)
%
%
%
%\end{thebibliography}
\newpage

\appendix

\section{Proof in Section \ref{sec:inductive rules of extended priority queue}}
\label{sec:appendix in section inductive rules of extended priority queue}

A $\textit{labelled transition system}$ ($LTS$) is a tuple $\mathcal{A}=(Q,\Sigma,\rightarrow,q_0)$, where $Q$ is a set of states, $\Sigma$ is an alphabet of transition labels, $\rightarrow\subseteq Q\times\Sigma\times Q$ is a transition relation and $q_0$ is the initial state.

Let us model extended priority queue as an LTS $\textit{LTS}_e = (Q,\Sigma,\rightarrow,q_0)$ as follows:

\begin{itemize}
\setlength{\itemsep}{0.5pt}
\item[-] Each state of $Q$ is a function from $\mathbb{P}$ into a finite sequence over $\mathbb{D}$.

\item[-] The initial state $q_0$ is a function that maps each element in $\mathbb{P}$ into $\epsilon$.

\item[-] $\Sigma = \{ \textit{put}(a,p),\textit{rm}(a),\textit{rm}(\textit{empty}) \vert a \in \mathbb{D}, p \in \mathbb{P} \}$.

\item[-] The transition relation $\rightarrow$ is defined as follows:

    \begin{itemize}
    \setlength{\itemsep}{0.5pt}
    \item[-] $q_1 \xrightarrow{\textit{put}(a,p)} q_2$, if $q_1$ maps $p$ into some finite sequence $l$, and $q_2$ is the same as $q_1$, except for $p$, where it maps $p$ into $a \cdot l$.

    \item[-] $q_1 \xrightarrow{\textit{rm}(a)} q_2$, if $q_1$ maps $p$ into $l \cdot a$ for some finite sequence $l$, and $q_2$ is the same as $q_1$, except for $p$, where it maps $p$ into $l$. We also require that for each priority $p'$ such that $p' <_{\mathbb{P}} p$, $q_1$ and $q_2$ map $p'$ into $\epsilon$.

    \item[-] $q_1 \xrightarrow{\textit{rm}(\textit{empty})} q_2$, if $q_1 = q_2$, and they maps each element in $\mathbb{P}$ into $\epsilon$.
    \end{itemize}
\end{itemize}

A path of an LTS is a finite transition sequence $q_0\xrightarrow{\beta_1}q_1\overset{\beta_2}{\longrightarrow}\ldots\overset{\beta_k}{\longrightarrow}q_k$ for $k\geq 0$, where $q_0$ is the initial state of the LTS. A trace of an LTS is a finite sequence $\beta_1 \cdot \beta_2 \cdot \ldots \cdot \beta_k$, where $k \geq 0$ if there exists a path $q_0\overset{\beta_1}{\longrightarrow}q_1\overset{\beta_2}{\longrightarrow}\ldots\overset{\beta_k}{\longrightarrow}q_k$ of the LTS. Let $\textit{EPQ}_s$ be the set of traces of $\textit{LTS}_e$. The following lemma states that the language generated by our rules equals the set of traces of $\textit{EPQ}_s$.


\EPQRulesAndSemantics*

\begin {proof}

To prove that $\textit{EPQ} \subseteq \textit{EPQ}_s$, we prove that each sequence in $\textit{EPQ}$ is also in $\textit{EPQ}_s$ by induction:

\begin{itemize}
\setlength{\itemsep}{0.5pt}
\item[-] It is obvious that $\epsilon \in \textit{EPQ}_s$.

\item[-] If $l_1 \in \textit{EPQ}_s$ and $l_1 \xrightarrow{\textit{EPQ}_1} l_2$. Then we need to prove that $l_2 \in \textit{EPQ}_s$. Let $l_1 = u \cdot v \cdot w$, such that $\textit{Guard}(u,v,w,\textit{item},\textit{pri})$ of $\textit{EPQ}_1$ holds, and $l_2 = u \cdot \textit{put}(\textit{itm},\textit{pri}) \cdot v \cdot \textit{rm}(\textit{itm})$.

    Assume that $u = \alpha_1 \cdot \ldots \cdot \alpha_i$, $v = \alpha_{\textit{i+1}} \cdot \ldots \cdot \alpha_j$ and $w = \alpha_{\textit{j+1}} \cdot \ldots \cdot \alpha_m$. Assume that $q_0 \xrightarrow{\alpha_1} q_1 \ldots \xrightarrow{\alpha_i} q_i \xrightarrow{\alpha_{\textit{i+1}}} q_{\textit{i+1}} \ldots  \xrightarrow{\alpha_j} q_j \xrightarrow{\alpha_{\textit{j+1}}} q_{\textit{j+1}} \ldots \xrightarrow{\alpha_m} q_m$ is the path of $l_1$ on $\textit{EPQ}_s$. For each $i \leq k \leq j$, let $q'_k$ be the same as $q_k$, except that $q'_k$ maps $\textit{pri}$ into $\textit{item} \cdot l_k$ and $q_k$ maps $\textit{pri}$ into $l_k$ for some finite sequence $l_k$.

    We already know that $q_0 \xrightarrow{\alpha_1} q_1 \ldots \xrightarrow{\alpha_i} q_i$, and it is obvious that $q_i \xrightarrow{\textit{put}(\textit{itm},\textit{pri})} q'_i$. Since (1) all $\textit{put}$ with priority $\textit{pri}$ is in $u$, and (2) in $u \cdot v$, only items with priority either incomparable, or less, or equal than $\textit{pri}$ is removed, we can see that it is safe to add to each $q_k$ ($1 \leq k \leq j$) with a newest $\textit{itm}$ with priority $\textit{pri}$. Or we can say, $q'_i \xrightarrow{\alpha_{\textit{i+1}}} q'_{\textit{i+1}} \ldots \xrightarrow{\alpha_j} q'_j$ are transitions of $\textit{LTS}_e$. Since $\textit{matched-C}(u \cdot v)$ holds, we can see that $q_j$ maps each priority that is smaller than $\textit{pri}$ into $\epsilon$ and maps $\textit{pri}$ into $\epsilon$, and $q'_j$ maps each priority that is smaller than $\textit{pri}$ into $\epsilon$ and maps $\textit{pri}$ into $\textit{itm}$. Then, we can see that $q'_j \xrightarrow{\textit{rm}(\textit{itm})} q_j$. We already know that that $q_j \xrightarrow{\alpha_{\textit{j+1}}} q_{\textit{j+1}} \ldots \xrightarrow{\alpha_m} q_m$. Therefore, we can see that $l_2 = u \cdot \textit{put}(\textit{itm},\textit{pri}) \cdot v \cdot \textit{rm}(\textit{itm}) \in \textit{EPQ}_s$.

\item[-] If $l_1 \in \textit{EPQ}_s$ and $l_1 \xrightarrow{\textit{EPQ}_2} l_2$. Then we need to prove that $l_2 \in \textit{EPQ}_s$. Let $l_1 = u \cdot v$, such that $\textit{Guard}(u,v,\textit{item},\textit{pri})$ of $\textit{EPQ}_2$ holds, and $l_2 = u \cdot \textit{put}(\textit{itm},\textit{pri}) \cdot v$.

    Assume that $u = \alpha_1 \cdot \ldots \cdot \alpha_i$ and $v = \alpha_{\textit{i+1}} \cdot \ldots \cdot \alpha_m$. Assume that $q_0 \xrightarrow{\alpha_1} q_1 \ldots \xrightarrow{\alpha_i} q_i \xrightarrow{\alpha_{\textit{i+1}}} q_{\textit{i+1}} \ldots  \xrightarrow{\alpha_m} q_m$ is the path of $l_1$ on $\textit{EPQ}_s$. For each $i \leq k \leq m$, let $q'_k$ be the same as $q_k$, except that $q'_k$ maps $\textit{pri}$ into $\textit{item} \cdot l_k$ and $q_k$ maps $\textit{pri}$ into $l_k$ for some finite sequence $l_k$.

    We already know that $q_0 \xrightarrow{\alpha_1} q_1 \ldots \xrightarrow{\alpha_i} q_i$, and it is obvious that$q_i \xrightarrow{\textit{put}(\textit{itm},\textit{pri})} q'_i$. Since (1) all $\textit{put}$ with priority $\textit{pri}$ is in $u$, (2) in $u \cdot v$, only items with priority either incomparable, or less, or equal than $\textit{pri}$ is removed, we can see that it is safe to add to each $q_k$ ($1 \leq k \leq m$) with a newest $\textit{itm}$ with priority $\textit{pri}$. Or we can say, $q'_i \xrightarrow{\alpha_{\textit{i+1}}} q'_{\textit{i+1}} \ldots \xrightarrow{\alpha_m} q'_m$ are transitions of $\textit{LTS}_e$. Therefore, we can see that $l_2 = u \cdot \textit{put}(\textit{itm},\textit{pri}) \cdot v \in \textit{EPQ}_s$.

\item[-] If $l_1 \in \textit{EPQ}_s$ and $l_1 \xrightarrow{\textit{EPQ}_3} l_2$. Then we need to prove that $l_2 \in \textit{EPQ}_s$. Let $l_1 = u \cdot v$, such that $\textit{Guard}(u,v)$ of $\textit{EPQ}_3$ holds, and $l_2 = u \cdot \textit{rm}(\textit{empty}) \cdot v$.

    Assume that $u = \alpha_1 \cdot \ldots \cdot \alpha_i$ and $v = \alpha_{\textit{i+1}} \cdot \ldots \cdot \alpha_m$. Assume that $q_0 \xrightarrow{\alpha_1} q_1 \ldots \xrightarrow{\alpha_i} q_i \xrightarrow{\alpha_{\textit{i+1}}} q_{\textit{i+1}} \ldots  \xrightarrow{\alpha_m} q_m$ is the path of $l_1$ on $\textit{EPQ}_s$.

    We already know that $q_0 \xrightarrow{\alpha_1} q_1 \ldots \xrightarrow{\alpha_i} q_i$. Since $\textit{matched-All}(u)$ holds, we can see that $q_i$ maps each element in $\mathbb{P}$ into $\epsilon$, and then $q_i \xrightarrow{\textit{rm}(\textit{empty})} q_i$. We already know that $q_i \xrightarrow{\alpha_{\textit{i+1}}} q_{\textit{i+1}} \ldots \xrightarrow{\alpha_m} q_m$. Therefore, we can see that $l_2 = u \cdot \textit{rm}(\textit{empty}) \cdot v \in \textit{EPQ}_s$.
\end{itemize}

To prove that $\textit{EPQ}_s \subseteq \textit{EPQ}$, we show that given $l_2 \in \textit{EPQ}_s$ and $l_2 \neq \epsilon$, how to construct a sequence $l_1$, such that $l_1 \xrightarrow{R} l_2$ for some rule $R$, and $l_1 \in \textit{EPQ}_s$. Based on this, we can decompose a sequence of $\textit{EPQ}_s$ into $\epsilon$, while the reverse process is the reason of why this sequence is in $\textit{EPQ}$. Note that from a $l_2$ we may construct more than one $l_1$, and this does not influence the correctness of our proof.

\begin{itemize}
\setlength{\itemsep}{0.5pt}
\item[-] If $l_2$ contains $\textit{rm}(\textit{empty})$: Assume that $l_2 = u \cdot \textit{rm}(\textit{empty}) \cdot v$. It is easy to see that $\textit{matched-All}(u)$ holds. Let $l_1 = u \cdot v$. It is easy to see that $l_1 \xrightarrow{\textit{EPQ}_3} l_2$, and $l_1$ satisfy the guard of $\textit{EPQ}_3$.

    Assume that $u = \alpha_1 \cdot \ldots \cdot \alpha_i$ and $v = \alpha_{\textit{i+1}} \cdot \ldots \cdot \alpha_m$. Since We already know that $q_0 \xrightarrow{\alpha_1} q_1 \ldots \xrightarrow{\alpha_i} q_i \xrightarrow{\textit{rm}(\textit{empty})} q'_i \xrightarrow{\alpha_{\textit{i+1}}} q_{\textit{i+1}} \ldots  \xrightarrow{\alpha_m} q_m$ is transitions of $\textit{LTS}_e$. Since $\textit{matched-All}(u)$ holds, we can see that $q_i = q'_i$, and they map each element in $\mathbb{P}$ into $\epsilon$. Then we can see that $q_0 \xrightarrow{\alpha_1} q_1 \ldots \xrightarrow{\alpha_i} q_i \xrightarrow{\alpha_{\textit{i+1}}} q_{\textit{i+1}} \ldots  \xrightarrow{\alpha_m} q_m$ is transitions of $\textit{LTS}_e$, and $l_1 \in \textit{EPQ}_s$.

\item[-] If $l_2$ does not contain $\textit{rm}(\textit{empty})$, $\textit{pri}$ is one of maximal priority of $l_2$, and items in $l_2$ with priority $\textit{pri}$ are unmatched $\textit{put}$ and (possibly) matched $\textit{put}$:

    Assume that $l_2 = u \cdot \textit{put}(\textit{itm},\textit{pri}) \cdot v$, such that all $\textit{put}$ with priority $\textit{pri}$ of $u \cdot v$ is in $u$. Let $l_1 = u \cdot v$. According to construction of $\textit{LTS}_e$, we can see that $l_1$ satisfies the guard of $\textit{EPQ}_2$, and $l_1 \xrightarrow{\textit{EPQ}_2} l_2$.

    Assume that $u = \alpha_1 \cdot \ldots \cdot \alpha_i$ and $v = \alpha_{\textit{i+1}} \cdot \ldots \cdot \alpha_m$. We already know that $\textit{pa} = q_0 \xrightarrow{\alpha_1} q_1 \ldots \xrightarrow{\alpha_i} q_i \xrightarrow{\textit{put}(\textit{itm},\textit{pri})} q_{\textit{i+}} \xrightarrow{\alpha_{\textit{i+1}}} q_{\textit{i+1}} \ldots \xrightarrow{\alpha_m} q_m$ are transitions of $\textit{LTS}_e$. For each $\textit{i+1} \leq k \leq m$, let $q'_k$ be the same as $q_k$, except that $q_k$ maps $\textit{pri}$ into some $\textit{itm} \cdot l_k$ for some finite sequence $l_k$, and $q'_k$ maps $\textit{pri}$ into $l_k$. Since (1) all $\textit{put}$ with priority $\textit{pri}$ of $u \cdot v$ is in $u$ and (2) $\textit{pri}$ is one of maximal priority of $l_2$, it is safe to remove $\textit{itm}$ without influence other transitions of $\textit{pa}$. Or we can say, $q_0 \xrightarrow{\alpha_1} q_1 \ldots \xrightarrow{\alpha_i} q_i \xrightarrow{\alpha_{\textit{i+1}}} q'_{\textit{i+1}} \ldots \xrightarrow{\alpha_m} q'_m$ are transitions of $\textit{LTS}_e$. Therefore, $l_1 \in \textit{EPQ}_s$.


\item[-] If $l_2$ does not contain $\textit{rm}(\textit{empty})$, $\textit{pri}$ is one of maximal priority of $l_2$, and items in $l_2$ with priority $\textit{pri}$ are matched $\textit{put}$:

    Assume that $l_2 = u \cdot \textit{put}(\textit{itm},\textit{pri}) \cdot v \cdot \textit{rm}(\textit{itm}) \cdot w$, such that all $\textit{put}$ with priority $\textit{pri}$ of $u \cdot v \cdot w$ is in $u$. Let $l_1 = u \cdot v \cdot w$. According to construction of $\textit{LTS}_e$, we can see that $l_1$ satisfies the guard of $\textit{EPQ}_1$. We can also see that $l_1 \xrightarrow{\textit{EPQ}_1} l_2$.

    Assume that $u = \alpha_1 \cdot \ldots \cdot \alpha_i$, $v = \alpha_{\textit{i+1}} \cdot \ldots \cdot \alpha_j$ and $w = \alpha_{\textit{j+1}} \cdot \ldots \cdot \alpha_m$. We already know that $q_0 \xrightarrow{\alpha_1} q_1 \ldots \xrightarrow{\alpha_i} q_i \xrightarrow{\textit{put}(\textit{itm},\textit{pri})} q_{\textit{i+}} \xrightarrow{\alpha_{\textit{i+1}}} q_{\textit{i+1}} \ldots \xrightarrow{\alpha_j} q_j \xrightarrow{\textit{rm}(\textit{itm})} q_{\textit{j+}} \xrightarrow{\alpha_{\textit{j+1}}} q_{\textit{j+1}} \ldots \xrightarrow{\alpha_m} q_m$. For each $\textit{i+1} \leq k \leq j$, let $q'_k$ be the same as $q_k$, except that $q_k$ maps $\textit{pri}$ into $\textit{itm} \cdot l_k$ for some finite sequence $l_k$, and $q'_k$ maps $\textit{pri}$ into $l_k$. Since (1) $\textit{pri}$ is one of maximal priority in $l_2$, (2) $\textit{itm}$ is the newest item with priority $\textit{pri}$ in $l_2$, and (3) $\textit{itm}$ is not removed until $\textit{rm}(\textit{itm})$, we know that whether we keep $\textit{itm}$ or remove it will not influence transitions from $q_{\textit{i+1}}$ to $q_j$. Then we can see that $q_0 \xrightarrow{\alpha_1} q_1 \ldots \xrightarrow{\alpha_i} q_i \xrightarrow{\alpha_{\textit{i+1}}} q'_{\textit{i+1}} \ldots \xrightarrow{\alpha_j} q'_j \xrightarrow{\alpha_{\textit{j+1}}} q_{\textit{j+1}} \ldots \xrightarrow{\alpha_m} q_m$ are transitions of $\textit{LTS}_e$. Therefore, $l_1 \in \textit{EPQ}_s$.
\end{itemize}

This completes the proof of this lemma. \qed
\end {proof}




\section{Proof in Section \ref{sec:data-independence of extended priority queue}}
\label{sec:appendix in section data-independence of EPQ}


\DataDifferentiatedisEnoughforPQ*

\begin {proof}

To prove the $\textit{only if}$ direction, given a data-differentiated execution $e \in \mathcal{I}_{\neq}$. By assumption, it is linearizable with respect to a sequential execution $l \in S$, and the bijection between the operations of $e$ and the method events of $l$ ensures that $l$ is differentiated and belongs to $S_{\neq}$.

To prove the $\textit{if}$ direction, given an execution $e \in \mathcal{I}$. By data independence of $\mathcal{I}$, we know that there exists $e' \in \mathcal{I}_{\neq}$ and a renaming function $r$, such that $r(e') = e$. By assumption, $e'$ is linearizable with respect to a sequential execution $l' \in S_{\neq}$. Let $l=r(l')$. By data independence of $S$ it is easy to see that $l \in S$, and it is easy to see that $e \sqsubseteq l$  using the same bijection used for $e' \sqsubseteq l'$. \qed
\end {proof}





\section{Proofs in Section \ref{sec:step-by-step linearizability of extended priority queues}}
\label{sec:appendix in section step-by-step linearizability of extended priority queues}


\subsection{Proof of Lemma \ref{lemma:EPQ is closed under projection}}

\EPQisClosedUnderProjection*

\begin {proof}

This is obvious, since for each conditions in the $\textit{Guard}$ part of the rules of priority queue, if a sequence of sequential executions satisfy it, then its sub-sequence also satisfy it. For example, if $\textit{noRE}(l)$ holds for some $l = u \cdot v \cdot w$ and let $D_l$ be the set of items of $l$, then for each subset $D' \subseteq D_l$, it is obvious that $\textit{rm}(\textit{empty}) \notin l \vert_{ D' }$ and $\textit{noRE}(l')$ holds. Similar cases hold for other predicates of the four rules of $\textit{EPQ}$, such as $\textit{LEI}$, $\textit{LI-U}$, $\textit{matched-C}$, $\textit{putInSeq}(l,l_1,\textit{pri})$ and $\textit{matched-All}$. This completes the proof of this lemma. \qed
\end {proof}


\subsection{Proof of Lemma \ref{lemma:EPQ as multi in MRpri for sequence}}

\EPQasMultiInMRpriforSequence*

\begin {proof}

The \textit{only if} direction is obvious and can be similarly proved as the $\textit{EPQ}_s \subseteq \textit{EPQ}$ direction of Lemma \ref{lemma:EPQ rules and semantics}.

To prove the $\textit{if}$ direction, we proceed as follows: From $e1=e$, we generate a sequence $e_2$ as follows:

\begin{itemize}
\setlength{\itemsep}{0.5pt}
\item[-] If $\textit{PQ}_3 \in \textit{last}(e_1)$: Then we can see that $e_1$ contains at least one $\textit{rm}(\textit{empty})$. $e_2$ is generated from $e_1$ by erasing one $\textit{rm}(\textit{empty})$.

\item[-] Else, if $\textit{PQ}_2^{=} \in \textit{last}(e_1)$: Then we can see that one of the maximal priority of $e_1$ is unmatched $\textit{put}$ and (possibly) matched $\textit{put}$. Assume the set of the items of these unmatched $\textit{put}$ is $S$. $e_2$ is generated from $e_1$ by erasing one unmatched $\textit{put}$ which use the item last putted in $S$.

\item[-] Else, if $\textit{PQ}_2^{>} \in \textit{last}(e_1)$: Then we can see that one of the maximal priority in $e_1$ is unmatched $\textit{put}$. Assume the set of the items of these unmatched $\textit{put}$ is $S$. $e_2$ is generated from $e_1$ by erasing one unmatched $\textit{put}$ which use the item last putted in $S$.

\item[-] Else, if $\textit{PQ}_1^{=} \in \textit{last}(e_1)$: Then we can see that one of the maximal priority in $e_1$ is of more than one pair of matched $\textit{put}$. Assume the set of the items of these matched $\textit{put}$ is $S$. $e_2$ is generated from $e_1$ by erasing matched $\textit{put}$ and $\textit{rm}$ of the item which is last putted in $S$.

\item[-] Else, if $\textit{PQ}_1^{>} \in \textit{last}(e_1)$: Then we can see that one of the maximal priority in $e_1$ is of one pair of matched $\textit{put}$. $e_2$ is generated from $e_1$ by erasing this pair of matched $\textit{put}$ and $\textit{rm}$.
\end{itemize}

Similarly, for each $i > 1$, we obtain $e_{\textit{i+1}}$ from $e_i$, until we obtain $e_m = \epsilon$ for some $m$. It is obvious that $e_m \in \textit{EPQ}$. For $e_{\textit{m-1}}$, since

\begin{itemize}
\setlength{\itemsep}{0.5pt}
\item[-] $e_m \in \textit{EPQ}$,

\item[-] By assumption, we know that $e_{\textit{m-1}} \in \textit{MS}(R_{\textit{m-1}})$, where $R_{\textit{m-1}} \in \textit{last}(e_{\textit{m-1}})$. This implies that the guard of $R_{\textit{m-1}}$ is satisfied.
\end{itemize}

Therefore, we know that $e_{\textit{m-1}} \in \textit{EPQ}$. Similarly, we can prove that $e_{\textit{m-2}},\ldots,e_1 = e \in \textit{EPQ}$. \qed
\end {proof}



\subsection{Proof of Lemma \ref{lemma:EPQ is step-by-step linearizability}}


The prove that $\textit{EPQ}$ is step-by-step linearizability, we investigate each rules individually.

Given a data-differentiated execution and its history, we can abuse notation and mix labels and method events with operations themselves, since items are unique in a data-differentiated execution. For instance, we will reference an operation labeled by $\textit{put}(p,a)$ as $\textit{put}(p,a)$.


Given an operation $o$ with call action $\textit{cal}_o (\textit{put},a,p)$ and return actions $\textit{ret}_o (\textit{put})$, its method event is $\textit{put}(a,p)$. Given an operation $o$ with call action $\textit{cal}_o (\textit{rm})$ and return actions $\textit{ret}_o (\textit{rm},a)$, its method event is $\textit{rm}(a)$.

Given a data-differentiated execution $e$ and its history $h$, we can obtain a sequence $h'$ from $h$ by adding $\textit{put}(a,p)$ (resp., $\textit{rm}(a)$, $\textit{rm}(\textit{empty})$) between each pair of $\textit{cal}(\textit{put},a,p)$ and $\textit{ret}(\textit{put},a)$ (resp., $\textit{cal}(\textit{rm},a)$ and $\textit{ret}(\textit{rm},a)$, $\textit{cal}(\textit{rm},\textit{empty})$ and $\textit{ret}(\textit{rm},\textit{empty})$). The projection of $h'$ into method events is called linearization of $e$ and $h$, and each method event we add in $h'$ can be considered as a linearization point of the corresponding method event. We call such $h'$ an execution with linearization points of $e$.


\begin{restatable}{lemma}{EPQ1isStepByStepLinearizability}
\label{lemma:EPQ1 is step-by-step linearizability}
If a data-differentiated concurrent execution $e$ is linearizable w.r.t. $\textit{MS}(\textit{EPQ}_1)$ with witness $x$, then $e \setminus x \sqsubseteq \textit{EPQ} \Rightarrow e \sqsubseteq \textit{EPQ}$.
\end{restatable}

\begin {proof}
Let $h$ be the data-differentiated history of $e$, and $l$ be an sequential execution such that $h \sqsubseteq l$ and $l$ matches $\textit{EPQ}_1$ with witness $x$. Let the priority of $x$ be $\textit{pri}_x$, and let $h'=h \setminus x$ and assume that $h' \sqsubseteq l' \in \textit{EPQ}$. Let $e_{\textit{lp}}$ be an execution with linearization points of $e$ and the linearization points is added according to $l'$. Or we can say, $e_{\textit{lp}}$ is generated from $e$ by instrumenting linearization points, and the projection of $e_{\textit{lp}}$ into method event is $l'$.

According to $\textit{MS}(\textit{EPQ}_1)$, there exist sequences $u$, $v$, and $w$, such that $l=u \cdot \textit{put}(x,\textit{pri}_x) \cdot v \cdot \textit{rm}(x) \cdot w$ and $u$, $v$, $w$, $x$ and $\textit{pri}_x$ satisfy the guard of $\textit{EPQ}_1$. Let $l'_v$ be the shortest prefix of $l'$ that contains all method event of $u \cdot v$.

Let $U$, $V$ and $W$ be the set of operations of $u$, $v$ and $w$, respectively. Let us change $U$, $V$ and $W$ by loops as follows: In the first loop, we start from the first $W$-element of $l'_v$, and let it be $o_h$,

\begin{itemize}
\setlength{\itemsep}{0.5pt}
\item[-] Case $1$: If in $e_{\textit{lp}}$, the linearization point of $o_h$ is before $\textit{ret}(\textit{rm},x)$, and no element in $W$ happens before $o_h$. Then, we erase $o_h$ from $W$ and put it into $U \cdot V$.

\item[-] Case $2$: Else, if in $e_{\textit{lp}}$, the linearization point of $o_h$ is before $\textit{ret}(\textit{rm},x)$, and there exists $o_w \in W$, such that $o_w <_{\textit{hb}} o_h$. Then, in $l'_v$, we erase all such operations from $U \cdot V$ and put them into $W$, and then stop the process of changing $U$, $V$ and $W$. These operations should satisfy (1) their linearization point is after $o_h$ in $e_{\textit{lp}}$, and (2) the priority of their item is incomparable with $\textit{pri}_x$.

\item[-] Case $3$: Else, if in $e_{\textit{lp}}$, the linearization point of $o_h$ is after $\textit{ret}(\textit{rm},x)$. Then, in $l'_v$, we erase all such operations from $U \cdot V$ and put them into $W$, and then stop the process of changing $U$, $V$ and $W$. These operations should satisfy (1) their linearization point is after $o_h$ in $e_{\textit{lp}}$, and (2) the priority of their item is incomparable with $\textit{pri}_x$.
\end{itemize}

Our process proceed, until either all element in $l'_v$ are in new $U$ or new $V$, or case $2$ or case $3$ holds and this process terminates. Let $U'$, $V'$ and $W'$ be the new version of $U$, $V$ and $W$ after the process terminates, respectively. Let $O_+$ be the set of operations that are moved into $U' \cup V'$ in the process, and let $O_-$ be the set of operations that are moved into $W'$ in the process.

Let $l'_{\textit{u'v'}}$ be the projection of $l'$ into $U' \cup V'$, let $O_x$ be the set of $\textit{put}(\_,\textit{pri}_x)$ while the item is not $x$ in $h$. Let $l''_a$ be the longest prefix of $l'_{\textit{u'v'}}$, where linearization of each operation of $l''_a$ is before $\textit{ret}(\textit{put},b)$ in $e_{\textit{lp}}$. Let $l''_d$ be the projection of $l'$ into operations of $O_x$ that are not in $l''_a$. Let $l''_1 = l''_a \cdot l''_d$. Let $l''_2$ be the projection of $l'$ into operations of $l'_{\textit{u'v'}}$ that are not in $l''_1$. Let $l''_3$ be the projection of $l'$ into operations of $l'_{\textit{u'v'}}$ that are not in $l''_1$ and $l''_2$. Let $l'' = l''_1 \cdot \textit{put}(x,\textit{pri}_x) \cdot l''_2 \cdot \textit{rm}(x) \cdot l''_3$.

To prove $h \sqsubseteq l''$, we define a graph $G$ whose nodes are the operations of $h$ and there is an edge from operation $o_1$ to $o_2$, if one of the following case holds

\begin{itemize}
\setlength{\itemsep}{0.5pt}
\item[-] $o_1$ happens-before $o_2$ in h,

\item[-] the method event corresponding to $o_1$ in $l''$ is before the one corresponding to $o_2$.
\end{itemize}

Assume there is a cycle in $G$. According the the property of interval order and the fact that the order of $l''$ is total, we know that there must exists $o_1$ and $o_2$, such that $o_1$ happens-before $o_2$ in $h$, but the corresponding method events are in the opposite order in $l''$. Then, we consider all possible case of $o_1$ and $o_2$ as follows: Let $O_a$ and $O_d$ be the set of operations in $l''_a$ and $l''_d$, respectively.

\begin{itemize}
\setlength{\itemsep}{0.5pt}
\item[-] If $o_2 \in l''_1 \wedge o_1 \in l''_1$:
    \begin{itemize}
    \setlength{\itemsep}{0.5pt}
    \item[-] If $o_1,o_2 \in O_a$ or $o_1,o_2 \in O_d$: Then $l'$ contradicts with happen before relation of $h$.

    \item[-] If $o_2 \in O_a \wedge o_1 \in O_d$: Then the order of linearization points of $e_{\textit{lp}}$ contradicts with happen before relation of $h$.
    \end{itemize}

\item[-] If $o_2 \in l''_1 \wedge o_1 = \textit{put}(x,\textit{pri}_x)$:
    \begin{itemize}
    \setlength{\itemsep}{0.5pt}
    \item[-] If $o_1 \in O_a$: This is impossible, since the linearization point of operations in $O_a$ is before $\textit{ret}(\textit{put},b)$ in $e_{\textit{lp}}$.

    \item[-] If $o_1 \in O_d$: Then $l$ contradicts with happen before relation of $h$.
    \end{itemize}

\item[-] If $o_2 \in l''_1 \wedge o_1 \in l''_2$:
    \begin{itemize}
    \setlength{\itemsep}{0.5pt}
    \item[-] If $o_2 \in O_a$: This violates the order of linearization point in $e_{\textit{lp}}$.

    \item[-] If $o_2 \in O_d$: According to $l$, we can see that $\textit{put}(x,\textit{pri}_x)$ does not happen before any operation in $O_x$. Then we can see that the linearization point of $o_1$ is before $\textit{ret}(\textit{put},b)$ and $o_1 \in O_a$. This violates that $o_1 \in l''_2$.
    \end{itemize}

\item[-] If $o_2 \in l''_1 \wedge o_1 = \textit{rm}(x)$:
    \begin{itemize}
    \setlength{\itemsep}{0.5pt}
    \item[-] If $o_2 \in U \cup V$: Then $l$ contradicts with happen before relation of $h$.

    \item[-] If $o_2 \in O_+$: This is impossible, since the linearization point of operations in $O_+$ is before $\textit{ret}(\textit{rm},b)$ in $e_{\textit{lp}}$.
    \end{itemize}

\item[-] If $o_2 \in l''_1 \wedge o_1 \in l''_3$:
    \begin{itemize}
    \setlength{\itemsep}{0.5pt}
    \item[-] If $o_1 \in W \wedge o_2 \in U \cup V$: Then $l$ contradicts with happen before relation of $h$.

    \item[-] If $o_1 \in W \wedge o_2 \in O_+$: This is impossible, since according to construction of $O_+$, we can see that $o_2 \in O_-$.

    \item[-] If $o_1 \in O_- \wedge o_2 \in U \cup V$:
        \begin{itemize}
        \setlength{\itemsep}{0.5pt}
        \item[-] If the reason of $o_1 \in O_-$ is case $2$: Let $o_h$ be as in case $2$. Then there exists $o_w \in W$, and in $e_{\textit{lp}}$, $\textit{ret}(o_w)$ is before $\textit{cal}(o_h)$, the linearization point of $o_h$ is before the linearization point of $o_1$, and $\textit{ret}(o_1)$ is before $\textit{cal}(o_2)$. Therefore, we can see that $o_w <_{\textit{hb}} o_2$, and then $l$ contradicts with happen before relation of $h$.

        \item[-] If the reason of $o_1 \in O_-$ is case $3$: Let $o_h$ be as in case $3$. Then in $e_{\textit{lp}}$, $\textit{ret}(\textit{rm},x)$ is before the linearization point of $o_h$, the linearization point of $o_h$ is before the linearization point of $o_1$, and $\textit{ret}(o_1)$ is before $\textit{cal}(o_2)$. Therefore, we can see that $\textit{rm}(x) <_{\textit{hb}} o_2$, and then $l$ contradicts with happen before relation of $h$.
        \end{itemize}
    \item[-] If $o_1 \in O_- \wedge O_2 \in O_+$: This is impossible, since in $e_{\textit{lp}}$, the linearization points of operations in $O_+$ is before the linearization points of operations in $O_-$.
    \end{itemize}

\item[-] If $o_2 = \textit{put}(x,\textit{pri}_x) \wedge o_1 \in l''_2$: This is impossible, since in $e_{\textit{lp}}$, the linearization points of operations in $l''_2$ is after $\textit{ret}(\textit{put},x)$.

\item[-] If $o_2 = \textit{put}(x,\textit{pri}_x) \wedge o_1 = \textit{rm}(x)$: Then $l$ contradicts with happen before relation of $h$.

\item[-] If $o_2 = \textit{put}(x,\textit{pri}_x) \wedge o_1 \in l''_3$:
    \begin{itemize}
    \setlength{\itemsep}{0.5pt}
    \item[-] If $o_1 \in W$: Then $l$ contradicts with happen before relation of $h$.

    \item[-] If $o_1 \in O_-$:
         \begin{itemize}
         \setlength{\itemsep}{0.5pt}
         \item[-] If the reason of $o_1 \in O_-$ is case $2$: Let $o_h$ be as in case $2$. Then there exists $o_w \in W$, and in $e_{\textit{lp}}$, $\textit{ret}(o_w)$ is before $\textit{cal}(o_h)$, the linearization point of $o_h$ is before the linearization point of $o_1$, and $\textit{ret}(o_1)$ is before $\textit{cal}(\textit{put},x,\textit{pri}_x)$. Therefore, we can see that $o_w <_{\textit{hb}} \textit{put}(x,\textit{pri}_x)$, and then $l$ contradicts with happen before relation of $h$.

         \item[-] If the reason of $o_1 \in O_-$ is case $3$: Let $o_h$ be as in case $3$. Then in $e_{\textit{lp}}$, $\textit{ret}(\textit{rm},x)$ is before the linearization point of $o_h$, the linearization point of $o_h$ is before the linearization point of $o_1$, and $\textit{ret}(o_1)$ is before $\textit{cal}(\textit{put},x,\textit{pri}_x)$. Therefore, we can see that $\textit{rm}(x) <_{\textit{hb}} \textit{put}(x,\textit{pri}_x)$, and then $l$ contradicts with happen before relation of $h$.
         \end{itemize}
    \end{itemize}

\item[-] If $o_2 \in l''_2 \wedge o_1 \in l''_2$: Then $l'$ contradicts with happen before relation of $h$.

\item[-] If $o_2 \in l''_2 \wedge o_1 = \textit{rm}(x)$: We can prove this similarly as the case of $o_2 \in l''_1 \wedge o_1 = \textit{rm}(x)$.

\item[-] If $o_2 \in l''_2 \wedge o_1 \in l''_3$: We can prove this similarly as the case of $o_2 \in l''_1 \wedge o_1 \in l''_3$.

\item[-] If $o_2 = \textit{rm}(x) \wedge o_1 \in l''_3$:
    \begin{itemize}
    \setlength{\itemsep}{0.5pt}
    \item[-] If $o_1 \in W$: Then $l$ contradicts with happen before relation of $h$.

    \item[-] If $o_1 \in O_-$:
         \begin{itemize}
         \setlength{\itemsep}{0.5pt}
         \item[-] If the reason of $o_1 \in O_-$ is case $2$: Let $o_h$ be as in case $2$. Then there exists $o_w \in W$, and in $e_{\textit{lp}}$, $\textit{ret}(o_w)$ is before $\textit{cal}(o_h)$, the linearization point of $o_h$ is before the linearization point of $o_1$.

             Since $l$ is consistent with the happen before order of $h$, we can see that $\textit{cal}(\textit{rm},x)$ is before $\textit{ret}(o_w)$. Therefore, we can see that the linearization point of $o_1$ is after $\textit{cal}(\textit{rm},x)$, and then it is impossible that $o_1 <_{\textit{hb}} \textit{rm}(x)$.

         \item[-] If the reason of $o_1 \in O_-$ is case $3$: Let $o_h$ be as in case $3$. Then in $e_{\textit{lp}}$, $\textit{ret}(\textit{rm},x)$ is before the linearization point of $o_h$, and the linearization point of $o_h$ is before the linearization point of $o_1$. Therefore, we can see that the linearization point of $o_1$ is after $\textit{ret}(\textit{rm},x)$, and then it is impossible that $o_1 <_{\textit{hb}} \textit{rm}(x)$.
         \end{itemize}
    \end{itemize}

\item[-] If $o_2 \in l''_3 \wedge o_1 \in l''_3$: Then $l'$ contradicts with happen before relation of $h$.
\end{itemize}

Therefore, we know that $G$ is acyclic, and then we know that $h \sqsubseteq l''$.

It remains to prove that $l'' \in \textit{EPQ}$. Let $O_c$ be the set of operations in $h$, whose items have priority that are comparable with $\textit{pri}_x$, and let $O_i$ be the set of operations in $h$, whose items have priority that are incomparable with $\textit{pri}_x$. Then we obtain that $l'' \in \textit{EPQ}$ as follows:

\begin{itemize}
\setlength{\itemsep}{0.5pt}
\item[-] Since $l' \in \textit{EPQ}$ and $l'_v$ is a prefix of $l'$, it is obvious that $l'_v \in \textit{EPQ}$.

\item[-] $l'_{\textit{u'v'}}$ can be obtained from $l'_v$ as follows: From some time point $t$, discard all the $O_i$ operations after $t$ in $l'_v$. Since $\textit{pri}_x$ is one of maximal priority in $h$, we can see that $\forall p_i \in O_i$ and $\forall p_c \in O_c$, either $p_c <_{\mathbb{P}} p_i$, or $p_c$ is incomparable with $p_i$. Therefore, this process does not influence correctness of $l'_{\textit{u'v'}}$, and then $l'_{\textit{u'v'}} \in \textit{EPQ}$.

\item[-] $l''_1 \cdot l''_2$ can be obtained from $l'_{\textit{u'v'}}$ as follows: Execute until reaching some time point $t$, then first execute all $O_x$ operations after $t$, and then execute remanning operations. Since $O_x$ only contains $\textit{put}(\_,\textit{pri}_x)$ and $\textit{pri}_x$ is one of maximal priority in $h$, we can see that $l''_1 \cdot l''_2 \in \textit{EPQ}$.

\item[-] Let $l''_f$ be the projection of $l'$ into $O_-$, and let $l''_g$ be the projection of $l'$ into operations not in $l''_1$, $l''_2$ or $l''_f$. It is easy to see that $l''_3 = l''_f \cdot l''_g$.

\item[-] $l'_{\textit{u'v'}} \cdot l''_f$ can be obtained from $l'_v$ as follows: (1) Execute until reaching some time point $t$, (2) from $t$, execute only $O_c$ operations until the end of $l'_v$, and (3) execute $O_i$ operations after $t$ in $l'_v$. Since $\textit{pri}_x$ is one of maximal priority in $h$, we can see that $l'_{\textit{u'v'}} \cdot l''_f \in \textit{EPQ}$.

\item[-] It is easy to see that $l' = l'_v \cdot l''_g \in \textit{EPQ}$. We already know that $l'_{\textit{u'v'}} \cdot l''_f \in \textit{EPQ}$, $l'_v \in \textit{EPQ}$. It is not hard to see that the content of extended priority queue after executing $l'_{\textit{u'v'}} \cdot l''_f$ is the same as that after executing $l'_v$. Therefore, it is easy to prove that $l'_{\textit{u'v'}} \cdot l''_f \cdot l''_g \in \textit{EPQ}$.

\item[-] We already know that $l''_1 \cdot l''_2\in \textit{EPQ}$ and $l'_{\textit{u'v'}} \in \textit{EPQ}$, and it is not hard to see that the the content of extended priority queue after executing $l''_1 \cdot l''_2$ is the same as that after executing $l'_{\textit{u'v'}}$. Since $l'_{\textit{u'v'}} \cdot l''_f \cdot l''_g \in \textit{EPQ}$, it is easy to see that $l''_1 \cdot l''_2 \cdot l''_f \cdot l''_g \in \textit{EPQ}$. Or we can say, $l''_1 \cdot \l''_2 \cdot l''_3 \in \textit{EPQ}$. Since $u$, $v$, $w$, $x$ and $\textit{pri}_x$ satisfy the guard of $\textit{EPQ}_1$, it is easy to see that $l'' \in \textit{EPQ}$.
\end{itemize}

Therefore, we prove that $h \sqsubseteq l'' \in \textit{EPQ}$. This completes the proof of this lemma. \qed
\end {proof}


\begin{restatable}{lemma}{EPQ2isStepByStepLinearizability}
\label{lemma:EPQ2 is step-by-step linearizability}
If a data-differentiated concurrent execution $e$ is linearizable w.r.t. $\textit{MS}(\textit{EPQ}_2)$ with witness $x$, then $e \setminus x \sqsubseteq \textit{EPQ} \Rightarrow e \sqsubseteq \textit{EPQ}$.
\end{restatable}

\begin {proof}

Let $h$ be the data-differentiated history of $e$, and $l$ be an sequential execution such that $h \sqsubseteq l$ and $l$ matches $\textit{EPQ}_2$ with witness $x$. Let the priority of $x$ be $\textit{pri}_x$, and let $h'=h \setminus x$ and assume that $h' \sqsubseteq l' \in \textit{EPQ}$. Let $e_{\textit{lp}}$ be an execution with linearization points of $e$ and the linearization points is added according to $l'$. Or we can say, $e_{\textit{lp}}$ is generated from $e$ by instrumenting linearization points, and the projection of $e_{\textit{lp}}$ into method event is $l'$.

According to $\textit{MS}(\textit{EPQ}_2)$, there exist sequences $u$ and $v$, such that $l=u \cdot \textit{put}(x,\textit{pri}_x) \cdot v$ and $u$, $v$, $x$ and $\textit{pri}_x$ satisfy the guard of $\textit{EPQ}_2$.

Let $l''_a$ be the longest prefix of $l'$ such that linearization point of each operation of $l''_a$ is before $\textit{ret}(\textit{put},x)$ in $e_{\textit{lp}}$. Let $O_x$ be the set of $\textit{put}(\_,\textit{pri}_x)$ while the item is not $x$ in $h$. Let $l''_s$ be the projection of $l'$ into operations of $O_x$ that are not in $l''_a$. Let $l''_1 = l''_a \cdot l''_s$. Let $l''_2$ be the projection of $l'$ into operations of $l'$ that are not in $l''_1$. Let $l'' = l''_1 \cdot \textit{put}(x,\textit{pri}_x) \cdot l''_2$.

To prove $h \sqsubseteq l''$, we define graph $G$ as in Lemma \ref{lemma:EPQ1 is step-by-step linearizability}. Assume that there is a cycle in $G$, then there must exists $o_1$ and $o_2$, such that $o_1$ happens-before $o_2$ in $h$, but the corresponding method events are in the opposite order in $l''$. Then, we consider all possible case of $o_1$ and $o_2$ as follows: Let $O_a$ and $O_s$ be the set of operations in $l''_a$ and $l''_s$, respectively.

\begin{itemize}
\setlength{\itemsep}{0.5pt}
\item[-] If $o_2 \in l''_1 \wedge o_1 \in l''_1$:
    \begin{itemize}
    \setlength{\itemsep}{0.5pt}
    \item[-] If $o_1,o_2 \in O_a$ or $o_1,o_2 \in O_s$: Then $l'$ contradicts with happen before relation of $h$.

    \item[-] If $o_2 \in O_a \wedge o_1 \in O_s$: It is not hard to see that $\textit{put}(x,\textit{pri}_x) <_{\textit{hb}} o_2$. But then it is impossible that in $e_{\textit{lp}}$, the linearization point of $o_2$ be located before $\textit{ret}(\textit{put},x)$.
    \end{itemize}

\item[-] If $o_2 \in l''_1 \wedge o_1 = \textit{put}(x,\textit{pri}_x)$:
    \begin{itemize}
    \setlength{\itemsep}{0.5pt}
    \item[-] If $o_2 \in O_a$: This is impossible, since in $e_{\textit{lp}}$, the linearization point of $o_1$ is before $\textit{ret}(\textit{put},x)$.

    \item[-] If $o_2 \in O_s$: This is impossible, since such $o_1 = \textit{put}(\_,\textit{pri}_x)$, and $l$ is consistent with happen before relation of $h$.
    \end{itemize}

\item[-] If $o_2 \in l''_1 \wedge o_1 \in l''_2$:
    \begin{itemize}
    \setlength{\itemsep}{0.5pt}
    \item[-] If $o_2 \in O_a$: This is impossible, since in $e_{\textit{lp}}$, the linearization point of operation in $l''_a$ is before the linearization point of operations in $l''_2$.

    \item[-] If $o_2 \in O_s$: Since no $\textit{put}(\_,\textit{pri}_x)$ happens before $\textit{put}(x,\textit{pri}_x)$ in $h$, $\textit{cal}(o_2)$ is before $\textit{ret}(\textit{put},x)$. Since $o_1 <_{\textit{hb}} o_2$, we can see that $\textit{ret}(o_1)$ is before $\textit{cal}(o_2)$, and then $\textit{ret}(o_1)$ is before $\textit{ret}(\textit{put},x)$. Then the linearization point of $o_1$ can only be before $\textit{ret}(\textit{put},x)$, and $o_1 \in l''_a$, which contradicts that $o_1 \in l''_2$.
    \end{itemize}

\item[-] If $o_2 = \textit{put}(x,\textit{pri}_x) \wedge o_1 \in l''_2$: Then since the linearization point of $o_1$ can only be before $\textit{ret}(\textit{put},x)$, we can see that $o_1 \in l''_a$, which contradicts that $o_1 \in l''_2$.

\item[-] If $o_2 \in l''_2 \wedge o_1 \in l''_2$: Then $l'$ contradicts with happen before relation of $h$.
\end{itemize}

Therefore, we know that $G$ is acyclic, and then we know that $h \sqsubseteq l''$.

It remains to prove that $l'' \in \textit{EPQ}$. $l''_a \cdot l''_s \cdot l''_2$ can be obtained from $l'$ as follows: Execute until reaching some time point $t$, then first execute all $O_x$ operations after $t$, and then execute remanning operations. Since $O_x$ only contains $\textit{put}(x,\textit{pri}_x)$ and $\textit{pri}_x$ is one of maximal priority in $h$, we can see that $l''_1 \cdot l''_2 = l''_a \cdot l''_s \cdot l''_2 \in \textit{EPQ}$. Since $u$, $v$, $x$ and $\textit{pri}_x$ satisfy the guard of $\textit{EPQ}_2$, it is easy to see that $l'' \in \textit{EPQ}$.

Therefore, we prove that $h \sqsubseteq l'' \in \textit{EPQ}$. This completes the proof of this lemma.\qed
\end {proof}


\begin{restatable}{lemma}{EPQ3isStepByStepLinearizability}
\label{lemma:EPQ3 is step-by-step linearizability}
If a data-differentiated concurrent execution $e$ is linearizable w.r.t. $\textit{MS}(\textit{EPQ}_3)$ and $o$ is a $\textit{rm}(\textit{empty})$ event, then $e \setminus o \sqsubseteq \textit{PQueue} \Rightarrow e \sqsubseteq \textit{PQueue}$.
\end{restatable}

\begin {proof}
Let $h$ be the data-differentiated history of $e$, $l$ be an sequential execution such that $h \sqsubseteq l$, $l$ matches $\textit{EPQ}_3$ and $o$ is a $\textit{rm}$ method event in $h$. Let $h'=h \setminus o$ and assume that $h' \sqsubseteq l' \in \textit{EPQ}$.

According to $\textit{EPQ}_3$, there exist sequences $u$ and $v$, such that $l=u \cdot \textit{rm}(\textit{empty}) \cdot v$, where all the $\textit{put}$ operations and $\textit{rm}$ in $u$ are matched.

Let $E_L$ be the set of method events in $u$ and $E_R$ be the set of method events in $v$. Let $l'_L = l' \vert_{E_L}$ and $l'_R = l' \vert_{E_R}$. Let sequence $l'' = l'_L \cdot o \cdot L'_R$. Since priority queue is closed under projection (Lemma \ref{lemma:EPQ is closed under projection}) and all the $\textit{put}$ operations and $\textit{rm}$ in $u$ are matched, we know that $l'_L \in \textit{EPQ}$ and the the priority queue is empty after executing $l'_L$. Then we know that $l'_L \cdot \textit{rm}(\textit{empty}) \in \textit{EPQ}$. Since $l'_R$ is obtained from $l'$ by discarding pairs of matched $\textit{put}$ and $\textit{rm}$ operations, it is easy to see that $L'_R \in \textit{EPQ}$, and then we know that $l'' = l'_L \cdot o \cdot L'_R \in \textit{EPQ}$.

It remains to prove that $h \sqsubseteq l''$. To prove $h \sqsubseteq l''$, we define graph $G$ as in Lemma \ref{lemma:EPQ1 is step-by-step linearizability}. Assume that there is a cycle in $G$, then there must exists $o_1$ and $o_2$, such that $o_1$ happens-before $o_2$ in $h$, but the corresponding method events are in the opposite order in $l''$. Then, we consider all possible case of $o_1$ and $o_2$ as follows:

\begin{itemize}
\setlength{\itemsep}{0.5pt}
\item[-] $o_1,o_2 \in l'_L$, or $o_1,o_2 \in l'_R$: Then $l'$ contradicts with happen before relation of $h$.

\item[-] If $o_1=o \wedge o_2 \in l'_L$, or $o_1 \in l'_R \wedge o_2 \in l'_L$, or $o_1 \in l'_R \wedge o_2 = o$, then $l$ contradicts with happen before relation of $h$.
\end{itemize}

Therefore, we know that $G$ is acyclic, and then we know that $h \sqsubseteq \textit{EPQ}$. \qed
\end {proof}

The following lemma states that $\textit{EPQ}$ is step-by-step linearizability, it is a direct consequence of Lemma \ref{lemma:EPQ1 is step-by-step linearizability}, Lemma \ref{lemma:EPQ2 is step-by-step linearizability} and Lemma \ref{lemma:EPQ3 is step-by-step linearizability}.


\EPQueueisStepByStepLinearizability*

\begin {proof}
This is a direct consequence of Lemma \ref{lemma:EPQ1 is step-by-step linearizability}, Lemma \ref{lemma:EPQ2 is step-by-step linearizability} and Lemma \ref{lemma:EPQ3 is step-by-step linearizability}. \qed
\end {proof}


\subsection{Proof of Lemma \ref{lemma:EPQ as multi in MRpri for history}}

\EPQasMultiInMRpriforHistory*

\begin {proof}

To prove the $\textit{only if}$ direction, assume that $e \sqsubseteq l \in \textit{EPQ}$. Given $e' = e \vert_{D}$ and $l' = l \vert_{D}$, it is easy to see that $e' \sqsubseteq l'$, and by Lemma \ref{lemma:EPQ is closed under projection}, we can see that $l' \in \textit{EPQ}$. Then by Lemma \ref{lemma:EPQ as multi in MRpri for sequence} we know that for each $R \in \textit{last}(l')$, we have $l' \in \textit{MS}(R)$.

To prove the $\textit{if}$ direction, given $e_1 = e$, we generate sequence $e_2$ from $e_1$ as follows: Since $e_1 \in \textit{proj}(e)$, we know that for each $R_1 \in \textit{last}(e_1)$, we have $e_1 \sqsubseteq \textit{MS}(R_1)$. We choose an arbitrary $R_1$ in $\textit{last}(e_1)$,

\begin{itemize}
\setlength{\itemsep}{0.5pt}
\item[-] If $R_1 = \textit{EPQ}_3$: $e_2$ is generated from $e_1$ by erasing call and return of one $\textit{rm}(\textit{empty})$ operation.

\item[-] Else, if $R_1 = \textit{EPQ}_2^{=}, \textit{EPQ}_2^{>}, \textit{EPQ}_1^{=}, \textit{EPQ}_1^{>}$, and the witness is $x$: $e_2$ is generated from $e_1$ by erasing call and return of method event of item $x$.
\end{itemize}

Similarly, for each $i > 1$, we obtain $e_{\textit{i+1}}$ from $e_i$, until we obtain $e_m = \epsilon$ for some $m$. It is obvious that $e_m \sqsubseteq \textit{EPQ}$. For $e_{\textit{m-1}}$, since

\begin{itemize}
\setlength{\itemsep}{0.5pt}
\item[-] $e_m \sqsubseteq \textit{EPQ}$,

\item[-] If $\textit{last}(e_{\textit{m-1}}) = R_{\textit{m-1}} \in \{ \textit{EPQ}_1^{>}, \textit{EPQ}_1^{=}, \textit{EPQ}_2^{>}, \textit{EPQ}_2^{=} \}$ and $e_{\textit{m-1}}$ matches $R_{\textit{m-1}}$ with witness $x$: We already know that $e_{\textit{m-1}} \sqsubseteq \textit{MS}(R_{\textit{m-1}})$, $e_m = e_{\textit{m-1}} \setminus x \sqsubseteq \textit{EPQ}$, and by step-by-step linearizability of $\textit{EPQ}$ (Lemma \ref{lemma:EPQ is step-by-step linearizability}), we can see that $e_{\textit{m-1}} \sqsubseteq \textit{EPQ}$.


\item[-] If $\textit{last}(e_{\textit{m-1}}) = R_{\textit{m-1}} = \textit{EPQ}_3$ and $o$ is a $\textit{rm}(\textit{empty})$ in $e_{\textit{m-1}}$ that is removed in the process of constructing $e_m$: We already know that $e_{\textit{m-1}} \sqsubseteq \textit{MS}(R_{\textit{m-1}})$, $e_m = e_{\textit{m-1}} \setminus o \sqsubseteq \textit{EPQ}$, and by step-by-step linearizability of $\textit{EPQ}$ (Lemma \ref{lemma:EPQ is step-by-step linearizability}), we can see that $e_{\textit{m-1}} \sqsubseteq \textit{EPQ}$.
\end{itemize}

Therefore, we know that $e_{\textit{m-1}} \in \textit{EPQ}$. Similarly, we can prove that $e_{\textit{m-2}},\ldots,e_1 = e \in \textit{EPQ}$. \qed
\end {proof}


\section{Proofs and Definitions in Section \ref{sec:co-regular of extended priority queues}}
\label{sec:appendix proof and definition in section co-regular of extended priority queues}


\subsection{Proofs and Definitions in Subsection \ref{subsec:definition of co-regular}}
\label{sec:appendix proof and definition in section definition of co-regular}

\cite{Bouajjani:2015} states that, given a differentiated queue execution $e$ without $\textit{deq}(\textit{empty})$, $e$ is not linearizable with respect to queue, if one of the following cases holds for some $a,b$: (1) $\textit{deq}(b) <_{hb} \textit{enq}(b)$, (2) there are are no $\textit{enq}(b)$ and at least one $\textit{deq}(b)$, (3) there are are one $\textit{enq}(b)$ and more than one $\textit{deq}(b)$, and (4) $\textit{enq}(a) <_{\textit{hb}} \textit{enq}(b)$, and $\textit{deq}(b) <_{\textit{hb}} \textit{deq}(a)$, or $\textit{deq}(a)$ does not exists.

For each such case, we construct a witness automata. We generate witness automata $\mathcal{A}_{\textit{SinPri}}^1$ for the first case, and it is shown in \figurename~\ref{fig:automata for FIFO-1 in appendix}. Here $c_1 = \textit{cal}(\textit{put},a,\textit{anyPri})$, $\textit{ret}(\textit{put},a), \textit{cal}(\textit{rm},a),\textit{ret}(\textit{rm},a),\textit{cal}(\textit{rm},b),\textit{cal}(\textit{rm},\textit{empty}),\textit{ret}(\textit{rm},\textit{empty})$, $c_2 = c_1 + \textit{ret}(\textit{rm},b)$, $c_3 = c_2 + \textit{ret}(\textit{put},b)$.


\begin{figure}[htbp]
  \centering
  \includegraphics[width=0.6 \textwidth]{figures/PIC_AUTO_FIFO_1.pdf}
%\vspace{-10pt}
  \caption{Automaton $\mathcal{A}_{\textit{SinPri}}^1$}
  \label{fig:automata for FIFO-1 in appendix}
\end{figure}


We generate witness automata $\mathcal{A}_{\textit{SinPri}}^2$ for the second case, and it is shown in \figurename~\ref{fig:automata for FIFO-2}. Here $c_1 = \textit{cal}(\textit{put},a,\textit{anyPri}),\textit{ret}(\textit{put},a), \textit{cal}(\textit{rm},a),\textit{ret}(\textit{rm},a),\textit{cal}(\textit{rm},\textit{empty}),\textit{ret}(\textit{rm},\textit{empty})$, $c_2 = c_1 + \textit{cal}(\textit{rm},b) + \textit{ret}(\textit{rm},b)$.


\begin{figure}[htbp]
  \centering
  \includegraphics[width=0.3 \textwidth]{figures/PIC_AUTO_FIFO_2.pdf}
%\vspace{-10pt}
  \caption{Automaton $\mathcal{A}_{\textit{SinPri}}^2$}
  \label{fig:automata for FIFO-2}
\end{figure}

We generate witness automata $\mathcal{A}_{\textit{SinPri}}^3$ for the third case, and it is shown in \figurename~\ref{fig:automata for FIFO-3}. Here $c_1 = \textit{cal}(\textit{put},a,\textit{anyPri}),\textit{ret}(\textit{put},a), \textit{cal}(\textit{rm},a),\textit{ret}(\textit{rm},a),\textit{cal}(\textit{rm},\textit{empty}),\textit{ret}(\textit{rm},\textit{empty})$, $c_2 = c_1 + \textit{ret}(\textit{put},b)$, $c_3 = c_2 + \textit{ret}(\textit{rm},b)$, $c_4 = c_3 + \textit{cal}(\textit{rm},b)$, $c_5 = c_1 + \textit{ret}(\textit{rm},b)$, $c_6 = c_5 + \textit{cal}(\textit{rm},b)$.

\begin{figure}[htbp]
  \centering
  \includegraphics[width=0.7 \textwidth]{figures/PIC_AUTO_FIFO_3.pdf}
%\vspace{-10pt}
  \caption{Automaton $\mathcal{A}_{\textit{SinPri}}^3$}
  \label{fig:automata for FIFO-3}
\end{figure}

We generate witness automata $\mathcal{A}_{\textit{SinPri}}^4$ for the forth case, and it is shown in \figurename~\ref{fig:automata for FIFO-4}. Here $c_1 = c + \textit{cal}(\textit{rm},b)$, and $c_2 = c + \textit{ret}(\textit{put},b) + \textit{cal}(\textit{rm},a) + \textit{ret}(\textit{rm},a)$, where $c = \textit{cal}(\textit{put},d,\textit{anyPri}),\textit{ret}(\textit{put},d), \textit{cal}(\textit{rm},d),\textit{ret}(\textit{rm},d),\textit{cal}(\textit{rm},\textit{empty}),\textit{ret}(\textit{rm},\textit{empty})$.

\begin{figure}[htbp]
  \centering
  \includegraphics[width=0.9 \textwidth]{figures/PIC_AUTO_FIFO_4.pdf}
%\vspace{-10pt}
  \caption{Automaton $\mathcal{A}_{\textit{SinPri}}^4$}
  \label{fig:automata for FIFO-4}
\end{figure}

Let $\textit{Auts}_{\textit{sinPri}} = \{ \mathcal{A}_{\textit{SinPri}}^1, \mathcal{A}_{\textit{SinPri}}^2, \mathcal{A}_{\textit{SinPri}}^3, \mathcal{A}_{\textit{SinPri}}^4 \}$. Let us prove Lemma \ref{lemma:automata for extended priority queue with single priority}.

\AutoForEPQwithSignlePri*

\begin {proof}

\cite{Bouajjani:2015} states that, given a differentiated queue execution $e$ without $\textit{deq}(\textit{empty})$, $e$ is not linearizable with respect to queue, if one of the following cases holds for some $v_a,v_b$: (1) $\textit{deq}(v_b) <_{hb} \textit{enq}(v_b)$, (2) there are are no $\textit{enq}(v_b)$ and at least one $\textit{deq}(v_b)$, (3) there are are one $\textit{enq}(v_b)$ and more than one $\textit{deq}(v_b)$, and (4) $\textit{enq}(v_a) <_{\textit{hb}} \textit{enq}(v_b)$, and $\textit{deq}(v_b) <_{\textit{hb}} \textit{deq}(v_a)$, or $\textit{deq}(v_a)$ does not exists.

Let us prove the $\textit{only if}$ direction. Assume that there exists execution $e_0 \in \mathcal{I}$ and $e_0$ is accepted by an automaton in $\textit{Auts}_{\textit{sinPri}}$. By data-independence, we can see that there exists a data-differentiated $e \in \mathcal{I}$ and renaming function, such that $e_0=r(e)$. Let $e'$ be obtained from $e$ by first removing $\textit{rm}(\textit{empty})$, and then,

\begin{itemize}
\setlength{\itemsep}{0.5pt}
\item[-] If $e_0$ is accepted by $\mathcal{A}_{\textit{SinPri}}^1$, $\mathcal{A}_{\textit{SinPri}}^2$ or $\mathcal{A}_{\textit{SinPri}}^3$: Then remove all items that are not renamed into $b$ by $r$.

\item[-] If $e_0$ is accepted by $\mathcal{A}_{\textit{SinPri}}^4$: Then remove all items that are not renamed into $a$ or $b$ by $r$.
\end{itemize}

It is obvious that $e' \in \textit{proj}(e)$. It is easy to see that $\textit{transToQueue}(e')$ satisfies one of above conditions, and then $\textit{transToQueue}(e')$ is not linearizable w.r.t queue.

Let us prove the $\textit{if}$ direction. Assume that exists $e \in \mathcal{I}_{\neq}$, $e' \in \textit{proj}(e)$, such that $e'$ is single-priority  without $\textit{rm}(\textit{empty})$, and $\textit{transToQueue}(e')$ does not linearizable to queue. Then we construct a renaming function $r$ as follows:

\begin{itemize}
\setlength{\itemsep}{0.5pt}
\item[-] If this is because case $1$, case $2$ or case $3$: $r$ maps $v_b$ into $b$ and maps all other values into $a$.

\item[-] If this is because case $4$: $r$ maps $v_a$ and $v_b$ into $a$ and $b$, respectively, and maps all other values into $d$.
\end{itemize}

Then it is not hard to see that $r(e) \in \mathcal{I}$ and it is accepted by some automaton in $\textit{Auts}_{\textit{sinPri}}$. This completes the proof of this lemma. \qed
\end {proof}




\subsection{Proofs and Definitions in Subsection \ref{subsec:co-regular of EPQ1Lar}}
\label{sec:appendix proof and definition in section co-regular of EPQ1Lar}

The following lemma states that, from linearization of sub-histories, we can merge them and obtain a linearization (regardless of whether it belongs to sequential specification) of the whole history.

\begin{restatable}{lemma}{MergeTwoLinearization}
\label{lemma:merge two linearization}

Given a history $h$, operation sets $S_1$, $S_2$ and sequences $l_1$ and $l_2$. Let $h_1 = h \vert{S_1}$ and $h_2 = h \vert{S_2}$. Assume that $h_1 \sqsubseteq l_1$, $h_2 \sqsubseteq l_2$, and $S_1 \cup S_2$ contains all operations of $h$. Then, there exists a sequence $l$, such that $h \sqsubseteq l$, $l \vert{S_1} = l_1$ and $l \vert{S_2} = l_2$.
\end{restatable}

\begin {proof}

Given a history $h$ and a operation $o \in S_2$, let $\textit{MB}(o) = \{ o' \vert o' \in S_1$ and $o'$ happens before $o$ in $h \}$, let $\textit{SBI}(o) = \textit{min}\{ i \vert l_1[0,i]$ contains all elements of $\textit{MB}(o) \}$.

Let $l = s_1 \cdot l_2[1] \cdot \ldots \cdot l_2[n] \cdot s_{\textit{n+1}}$ be generated as follows, where $n = \vert l_2 \vert$:

\begin{itemize}
\setlength{\itemsep}{0.5pt}
\item[-] $s_1 = l_1[0, \textit{SBI}(l_2[1])]$,

\item[-] If $s_1 \cdot l_2[1] \cdot \ldots \cdot l_2[i]$ already contains $l_1(\textit{SBI}(l_2[\textit{i+1}]))$, then $s_{\textit{i+1}} = \epsilon$. Otherwise, $s_{\textit{i+1}}$ is a subsequence of $l_1$, which starts from the next of last elements of $s_1 \cdot l_2[1] \cdot \ldots \cdot l_2[i]$ in $l_1$ and ends in $l_1(\textit{SBI}(l_2[\textit{i+1}]))$.
\end{itemize}

It is obvious that $l \vert{S_1} = l_1$ and $l \vert{S_2} = l_2$, and it remains to prove that $h \sqsubseteq l$. We prove this by contradiction. Assume that $h$ is not linearizable with respect to $l$. Then there must be two operations of $h$, such that $o_1 <_{hb} o_2$ in $h$ but $o_2$ before $0_1$ in $l$. Since $l \vert{S_1} = l_1$, $l \vert{S_2} = l_2$, and $h_1 \sqsubseteq l_1$, $h_2 \sqsubseteq l_2$, it is easy to see that it is impossible that $o_1,o_2 \in S_1$ or $o_1,o_2 \in S_2$. There are only two possibilities:

\begin{itemize}
\setlength{\itemsep}{0.5pt}
\item[-] $o_1 \in S_1 \wedge o_2 \in S_2$. Then we can see that $o_2=l_2[i]$ and $o_1 \in s_j$ for some $i < j$. Since $o_1 <_{hb} o_2$, we know that $o_1 \in \textit{SBI}(o_2)$. By the construction of $l$, we know that $o_1$ must be in $s_k$ for some $k \leq i$, contradicts that $o_1 \in s_j$ with $i < j$.

\item[-] $o_1 \in S_2 \wedge o_2 \in S_1$. Then we can see that $o_2 \in s_i$ and $o_1 = l_2[j]$ for some $i \leq j$. It is easy to see that this leads to contradiction when $i = j$. For the case of $i \neq j$, we need to satisfy the following requirements: (1) $o_1$ ($l_2[j]$) does not happen before $l_2[i]$, (2) $l_2[i]$ does not happen before $o_2$, (3) $o_2$ is either overlap or happens before $o' \in \textit{MB}(l_2[i])$, and (4) $o' <_{hb} l_2[i]$. By enumeration we can see that it is impossible that above four conditions be satisfied while $o_1 <_{hb} o_2$.
\end{itemize}

This completes the proof of this lemma. \qed
\end {proof}

With Lemma \ref{lemma:merge two linearization}, we can now prove Lemma \ref{lemma:pri execution is enough}.

\priExecutionIsEnough*
\begin {proof}

We deal with the case of $R = \textit{EPQ}_1^{>}$ , and other cases can be similarly dealt with.

To prove the $\textit{only if}$ direction, given $e \sqsubseteq \textit{MS}(\textit{EPQ}_1^{>})$ and such $\textit{pri}$ and $x$. Since $e \sqsubseteq \textit{MS}(\textit{EPQ}_1^{>})$ with witness $x$, we know that $e \sqsubseteq u \cdot \textit{put}(x,\textit{pri}) \cdot v \cdot \textit{rm}(x) \cdot w$, where $u$, $v$, $w$, $x$ and $\textit{pri}$ satisfy the guard of $\textit{EPQ}_1^{>}$. Let $u'$, $v'$ and $w'$ be obtained from $u$, $v$ and $w$ by erasing all items with priority incomparable with $\textit{pri}$, respectively. It is not hard to see that $u'$, $v'$, $w'$, $x$ and $\textit{pri}$ satisfy the guard of $\textit{EPQ}_1^{>}$, and then $e \sqsubseteq l = u' \cdot \textit{put}(x,\textit{pri}) \cdot v' \cdot \textit{rm}(x) \cdot w' \in \textit{MS}(\textit{EPQ}_1^{>})$.

To prove the $\textit{if}$ direction, given $e' = \textit{pri-Exec}(e)$ and such $x$ and $\textit{pri}$. Since $e' \sqsubseteq \textit{MS}(\textit{EPQ}_1^{>})$ with witness $x$, we know that $e' \sqsubseteq l_1 = u \cdot \textit{put}(x,\textit{pri}) \cdot v \cdot \textit{rm}(x) \cdot w$, where $u$, $v$, $w$, $x$ and $\textit{pri}$ satisfy the guard of $\textit{EPQ}_1^{>}$. Let $O_c$ be the set of operations in $e$ that have priorities comparable with $\textit{pri}$, and Let $O_i$ be the set of operations in $e$ that have priorities incomparable with $\textit{pri}$. It is obvious that $l_1$ is the linearization of $e \vert_{O_c}$. By Lemma \ref{lemma:merge two linearization}, there exists sequence $l$, such that $e \sqsubseteq l$, and $l \vert_{O_c} = l_1$. Then $l = u' \cdot \textit{put}(x,\textit{pri}) \cdot v' \cdot \textit{rm}(x) \cdot w'$, where $u' \vert_{O_c} = u$, $v' \vert_{O_c} = v$ and $w' \vert_{O_c} = w$. Since $\textit{pri}$ is one of maximal priorities in $e$, and the predicates of guards of $\textit{EPQ}_1^{>}$ does not restrict $O_i$, it is easy to see that $l \in \textit{MS}(\textit{EPQ}_1^{>})$ and then $e \sqsubseteq l$ with witness $x$. \qed
\end {proof}


We can see that $\textit{UVSet}_i(e,x) \cap \textit{UVSet}_j(e,x) = \emptyset$ for any $i \neq j$.

The following lemma states that $\textit{UVSet}(e,x)$ contains only matched $\textit{put}$ and $\textit{rm}$.

\begin{restatable}{lemma}{UVSetHasMatchedPutandRm}
\label{lemma:UVSet has matched put and rm}
Given a data-differentiated $\textit{pri}_x$-execution $e$ with $\textit{last}(e) = \textit{EPQ}_1^{>}$. Let $\textit{put}(x,\textit{pri}_x)$ and $\textit{rm}(x)$ be method events of $e$ with maximal priority. Let $G$ be the graph representing the left-right constraint of $\textit{put}(x,\textit{pri}_x)$ and $\textit{rm}(x)$. Assume that $G$ has no cycle going through $x$. Then, $\textit{UVSet}(e,x)$ contains only matched $\textit{put}$ and $\textit{rm}$.
\end{restatable}
\begin {proof}

We prove this lemma by contradiction. Assume that there exists a value, such that $\textit{UVSet}(e,x)$ contains only its $\textit{put}$ and does not contain its $\textit{rm}$. Then we can see that there exists $d_1,\ldots,d_j$. Intuitively, $d_1,\ldots,d_j$ are elements in $\textit{UVSet}_1(e,x), \ldots, \textit{UVSet}_i(e,x)$, respectively. $\textit{UVSet}(e,x)$ contains $\textit{put}(d_j,\_)$ and does not contain $\textit{rm}(d_j)$. And each $d_i$ is the reason of $d_{\textit{i+1}} \in \textit{UVSet}_{\textit{i+1}}(e,x)$. Formally, we require that

\begin{itemize}
\setlength{\itemsep}{0.5pt}
\item[-] For each $1 \leq i \leq j$, method events of $d_i$ belongs to $\textit{UVSet}_i(e,x)$.

\item[-] For each $i \neq j$, $\textit{put}(d_i,\_),\textit{rm}(d_i) \in \textit{UVSet}_i(e,x)$. $\textit{put}(d_j,\_) \in \textit{UVSet}_j(e,x)$, and $e$ does not contain $\textit{rm}(d_j)$.

\item[-] An operation of $d_1$ happens before an operations of $x$. For each $1 < i \leq j$, an operation of $d_i$ happens an operation of $d_{\textit{i-1}}$.

\item[-] For each $k$ and $\textit{ind}$, if $k > \textit{ind+1}$, then no operation of $d_k$ happens before operation of $d_{\textit{ind}}$.
\end{itemize}

According to the definition of $\textit{UVSet}(e,x)$, it is easy to see that such $d_1,\ldots,d_j$ exists. Let us prove the following fact:

\noindent {\bf $\textit{fact}_1$}: Given $1 \leq i < j$, it can not be the case that $\textit{put}(d_i,\_)$ and $\textit{rm}(d_i)$ overlap.

Proof of $\textit{fact}_1$: We prove $\textit{fact}_1$ by contradiction. Assume that for some $i \neq j$, $\textit{put}(d_i,\_)$ and $\textit{rm}(d_i)$ overlap. Since $\textit{put}(d_i,\_), \textit{rm}(d_i) \in \textit{UVSet}_i(h,x)$, we know that an operation $o_i$ of $d_i$ happens before operation $o_{\textit{i-1}}$ of $d_{\textit{i-1}}$. Moreover, since $\textit{put}(d_i,\_)$ and $\textit{rm}(d_i)$ overlap, it is not hard to see that the call action of $\textit{put}(d_i,\_)$ and the call action of $\textit{rm}(d_i)$ is before the call action of $o_{\textit{i-1}}$. Since method events of $d_{\textit{i+1}}$ is in $\textit{UVSet}_{\textit{i+1}}(e,x)$, we know that an operation $o'_{\textit{i+1}}$ of $d_{\textit{i+1}}$ happens before operation $o'_i$ of $d_i$. Then, it is not hard to see that $o'_{\textit{i+1}}$ also happens before $o_{\textit{i-1}}$, which contradicts that for each $k > \textit{ind+1}$, no operation of $d_k$ happens before operation of $d_{\textit{ind}}$.

We already know that an operation of $d_1$ happens before an operation of $x$. By $\textit{fact}_1$, we can ensure that $\textit{put}(d_1,\_)$ happens before an operation of $x$, and then $d_1 \rightarrow x$ in $G$. For each $1 < i \leq j$, we know that an operation $o_i$ of $d_i$ happens before an operation $o_{\textit{i-1}}$ of $d_{\textit{i-1}}$. By $\textit{fact}_1$, we can ensure that $o_i=\textit{put}(d_i,\_)$ and $o_{\textit{i-1}}=\textit{rm}(d_{\textit{i-1}})$, and then $d_i \rightarrow d_{\textit{i-1}}$ in $G$. Since $h$ contains $\textit{put}(d_j,\_)$ and does not contain $\textit{rm}(d_j)$, we know that $x \rightarrow d_j$ in $G$. Then $G$ has a cycle going through $x$, contradicts that $G$ has no cycle going through $x$. \qed
\end {proof}


The following lemma states that $\textit{UVSet}(e,x)$ does not happen before $\textit{rm}(x)$ when the left-right constraint has no cycle going through $x$.

\begin{restatable}{lemma}{RmxDoesNotHappenBeforeUVSetForEPQ1Lar}
\label{lemma:Rmx does not happen before UVSet for EPQ1Lar}

Given a data-differentiated $\textit{pri}_x$-execution $e$ with $\textit{last}(e) = \textit{EPQ}_1^{>}$. Let $\textit{put}(x,\textit{pri}_x)$ and $\textit{rm}(x)$ be method events of $e$ with maximal priority. Let $G$ be the graph representing the left-right constraint of $\textit{put}(x,\textit{pri}_x)$ and $\textit{rm}(x)$. Assume that $G$ has no cycle going through $x$. Then, $\textit{rm}(x)$ does not happen before any operation in $\textit{UVSet}(e,x)$.
\end{restatable}

\begin {proof}

We prove this lemma by induction, and prove that $\textit{rm}(x)$ does not happen before any operation in $\textit{UVSet}_1(e,x)$, in $\textit{UVSet}_2(e,x)$, $\ldots$. Note that, by Lemma \ref{lemma:UVSet has matched put and rm}, $\textit{UVSet}(e,x)$ contains only matched $\textit{put}$ and $\textit{rm}$, and it is easy to see that for each $i$, $\textit{UVSet}_i(e,x)$ contains only matched $\textit{put}$ and $\textit{rm}$.

\noindent (1) Let us prove that $\textit{rm}(x)$ does not happen before any operation in $\textit{UVSet}_1(e,x)$ by contradiction. Assume that $\textit{rm}(x) <_{hb} o$, where $o \in \textit{UVSet}_1(e,x)$ is an operation of item $d$. %(according to the definition of $\textit{UVSet}_1(h,x)$, the priority of $d$ does not equals $\textit{pri}_x$).

We use a triple $(t_1,t_2,t_3)$ to represent related information. $t_1,t_2,t_3$ are chosen from $\{ \textit{put},\textit{rm} \}$. $t_1$ represents whether $o$ is a $\textit{put}$ method event or a $\textit{rm}$ method event. $t_2$ and $t_3$ is used for the reason of $o \in \textit{UVSet}_1(e,x)$: $o \in \textit{UVSet}_1(e,x)$, since an operation (of kind $t_2$) of $d$ happens before an operation (of kind $t_3$) of $x$. Let us consider all the possible cases of $(t_1,t_2,t_3)$:

\begin{itemize}
\setlength{\itemsep}{0.5pt}
\item[-] $(\textit{put},\textit{put},\textit{put})$: Then $\textit{rm}(x) <_{hb} \textit{put}(d,\_) <_{hb} \textit{put}(x,\textit{pri}_x)$, contradicts that $\textit{rm}(x)$ does not happen before $\textit{put}(x,\textit{pri}_x)$.

\item[-] $(\textit{put},\textit{put},\textit{rm})$: Then $\textit{rm}(x) <_{hb} \textit{put}(d,\_) <_{hb} \textit{rm}(x)$, contradicts that $\textit{rm}(x)$ does not happen before $\textit{rm}(x)$.

\item[-] $(\textit{put},\textit{rm},\textit{put})$: Then $( \textit{rm}(x) <_{hb} \textit{put}(d,\_) ) \wedge ( \textit{rm}(d) <_{hb} \textit{put}(x,\textit{pri}_x) )$. By interval order, we know that $( \textit{rm}(x) <_{hb} \textit{put}(x,\textit{pri}_x) ) \vee ( \textit{rm}(d) <_{hb} \textit{put}(d,\_) )$, which is impossible.

\item[-] $(\textit{put},\textit{rm},\textit{rm})$: Then $( \textit{rm}(x) <_{hb} \textit{put}(d,\_) ) \wedge ( \textit{rm}(d) <_{hb} \textit{rm}(x) )$. We can see that $\textit{rm}(d) <_{hb} \textit{rm}(x) <_{hb} \textit{put}(d,\_)$, which contradicts that $\textit{rm}(d)$ does not happen before $\textit{put}(d,\_)$.

\item[-] $(\textit{rm},\textit{put},\textit{put})$: Then $( \textit{rm}(x) <_{hb} \textit{rm}(d) ) \wedge ( \textit{put}(d,\_) <_{hb} \textit{put}(x,\textit{pri}_x) )$. We can see that $x$ and $d$ has circle in $G$, contradicts that $G$ has no cycle going through $x$.

\item[-] $(\textit{rm},\textit{put},\textit{rm})$: Then $( \textit{rm}(x) <_{hb} \textit{rm}(d) ) \wedge ( \textit{put}(d,\_) <_{hb} \textit{rm}(x) )$. We can see that $x$ and $d$ has circle in $G$, contradicts that $G$ has no cycle going through $x$.

\item[-] $(\textit{rm},\textit{rm},\textit{put})$: Then $\textit{rm}(x) <_{hb} \textit{rm}(d) <_{hb} \textit{put}(x,\textit{pri}_x)$, contradicts that $\textit{rm}(x)$ does not happen before $\textit{put}(x,\textit{pri}_x)$.

\item[-] $(\textit{rm},\textit{rm},\textit{rm})$: Then $\textit{rm}(x) <_{hb} \textit{rm}(d) <_{hb} \textit{rm}(x)$, contradicts that $\textit{rm}(x)$ does not happen before $\textit{rm}(x)$.
\end{itemize}

This completes the proof for $\textit{UVSet}_1(e,x)$.

\noindent (2) Assume we already prove that for some $j \geq 1$, $\textit{rm}(x)$ does not happen before any operation in $\textit{UVSet}_1(e,x) \cup \ldots \cup \textit{UVSet}_j(e,x)$. Let us prove that $\textit{rm}(x)$ does not happen before any operation in $\textit{UVSet}_{\textit{j+1}}(e,x)$ by contradiction. Assume that $\textit{rm}(x) <_{hb} o$, where $o \in \textit{UVSet}_{\textit{j+1}}(e,x)$ is an operation of item $d_{\textit{j+1}}$. We use a triple $(t_1,t_2,t_3)$ to represent related information. $t_1,t_2,t_3$ are chosen from $\{ \textit{put},\textit{rm} \}$. $t_1$ represents whether $o$ is a $\textit{put}$ method event or a $\textit{rm}$ method event. $t_2$ and $t_3$ is used for the reason of $o \in \textit{UVSet}_{\textit{j+1}}(e,x)$: $o \in \textit{UVSet}_{\textit{j+1}}(e,x)$, since an operation (of kind $t_2$) of $d_{\textit{j+1}}$ happens before an operation (of kind $t_3$) of $d_j$, where $\textit{put}(d_j,\_), \textit{rm}(d_j) \in \textit{UVSet}_j(e,x)$. Let us consider all the possible cases of $(t_1,t_2,t_3)$:

\begin{itemize}
\setlength{\itemsep}{0.5pt}
\item[-] $(\textit{put},\textit{put},\textit{put})$: Then $\textit{rm}(x) <_{hb} \textit{put}(d_{\textit{j+1}},\_) <_{hb} \textit{put}(d_j,\_)$. We can see that $( \textit{rm}(x) <_{hb} \textit{put}(d_j,\_) ) \wedge ( \textit{put}(d_j,\_) \in \textit{UVSet}_j(e,x) )$, which contradicts that $\textit{rm}(x)$ does not happen before any operation in $\textit{UVSet}_1(e,x) \cup \ldots \cup \textit{UVSet}_j(e,x)$.

\item[-] $(\textit{put},\textit{put},\textit{rm})$: Then $\textit{rm}(x) <_{hb} \textit{put}(d_{\textit{j+1}},\_) <_{hb} \textit{rm}(d_j,\_)$. We can see that $( \textit{rm}(x) <_{hb} \textit{rm}(d_j,\_) ) \wedge ( \textit{rm}(d_j) \in \textit{UVSet}_j(e,x) )$, which contradicts that $\textit{rm}(x)$ does not happen before any operation in $\textit{UVSet}_1(e,x) \cup \ldots \cup \textit{UVSet}_j(e,x)$.

\item[-] $(\textit{put},\textit{rm},\textit{put})$: Then $( \textit{rm}(x) <_{hb} \textit{put}(d_{\textit{j+1}},\_) ) \wedge ( \textit{rm}(d_{\textit{j+1}}) <_{hb} \textit{put}(d_j,\_) )$. By interval order, we know that $( \textit{rm}(x) <_{hb} \textit{put}(d_j,\_) ) \vee ( \textit{rm}(d_{\textit{j+1}}) <_{hb} \textit{put}(d_{\textit{j+1}},\_) )$, which is impossible.

\item[-] $(\textit{put},\textit{rm},\textit{rm})$: Then $( \textit{rm}(x) <_{hb} \textit{put}(d_{\textit{j+1}},\_) ) \wedge ( \textit{rm}(d_{\textit{j+1}}) <_{hb} \textit{rm}(d_j) )$. By interval order, we know that $( \textit{rm}(x) <_{hb} \textit{rm}(d_j) ) \vee ( \textit{rm}(d_{\textit{j+1}}) <_{hb} \textit{put}(d_{\textit{j+1}},\_) )$, which is impossible.

\item[-] $(\textit{rm},\textit{put},\textit{put})$: Then $( \textit{rm}(x) <_{hb} \textit{rm}(d_{\textit{j+1}}) ) \wedge ( \textit{put}(d_{\textit{j+1}},\_) <_{hb} \textit{put}(d_j,\_) )$. Let us consider the reason of $\textit{put}(d_j,\_), \textit{rm}(d_j) \in \textit{UVSet}_j(e,x)$:
    \begin{itemize}
    \setlength{\itemsep}{0.5pt}
    \item[-] If $( j > 1 ) \wedge ( \textit{put}(d_j,\_) <_{hb} o'' )$, where $o''$ is an operation of item $d_{\textit{j-1}}$ and $\textit{put}(d_{\textit{j-1}},\_), \textit{rm}(d_{\textit{j-1}}) \in \textit{UVSet}_{\textit{j-1}}(e,x)$: Then since $( \textit{put}(d_{\textit{j+1}},\_) <_{hb} \textit{put}(d_j,\_) ) \wedge ( \textit{put}(d_j,\_) <_{hb} o'' )$, we can see that $\textit{put}(d_{\textit{j+1}},\_) <_{hb} o''$, and then operations of $d_{\textit{j+1}}$ is in $\textit{UVSet}_j(e,x)$, contradicts that operations of $d_{\textit{j+1}}$ is in $\textit{UVSet}_{\textit{j+1}}(e,x)$.

    \item[-] If $( j = 1 ) \wedge ( \textit{put}(d_j,\_) <_{hb} o'' )$, where $o''$ is an operation of $x$: Similar to above case.

    \item[-] If $( j > 1 ) \wedge ( \textit{rm}(d_j) <_{hb} o'' )$, where $o''$ is an operation of item $d_{\textit{j-1}}$ and $\textit{put}(d_{\textit{j-1}},\_), \textit{rm}(d_{\textit{j-1}}) \in \textit{UVSet}_{\textit{j-1}}(e,x)$: Then since $( \textit{put}(d_{\textit{j+1}},\_) <_{hb} \textit{put}(d_j,\_) ) \wedge ( \textit{rm}(d_j) <_{hb} o'' )$, we can see that $( \textit{put}(d_{\textit{j+1}},\_) <_{hb} o'' ) \vee ( \textit{rm}(d_j) <_{hb} \textit{put}(d_j,\_) )$, which is impossible.

    \item[-] If $( j > 1 ) \wedge ( \textit{rm}(d_j) <_{hb} o'' )$, where $o''$ is an operation of $x$: Similar to above case.
    \end{itemize}

\item[-] $(\textit{rm},\textit{put},\textit{rm})$: Let $T_{\textit{ind}}$ be the set of sentences $\{ \textit{rm}(x) <_{hb} \textit{rm}(d_{\textit{j+1}}), \textit{put}(d_{\textit{j+1}},\_) <_{hb} \textit{rm}(d_j),\ldots, \textit{put}(d_{\textit{ind+1}},\_) <_{hb} \textit{rm}(d_{\textit{ind}}) \}$. Here each $d_i$ is a item of some operation in $\textit{UVSet}_i(e,x)$. Let us prove that from $T_j$ we can obtain contradiction by induction:

    {\bf Base case $1$}: From $T_1$ we can obtain contradiction.

    Let us prove base case $1$:

    \begin{itemize}
    \setlength{\itemsep}{0.5pt}
    \item[-] If $\textit{put}(d_1,\_)$ happens $o$, and $o$ is an operation of $x$. Then there is a cycle $x \rightarrow d_{\textit{j+1}} \rightarrow \ldots \rightarrow d_1 \rightarrow x$ in $G$, contradicts that $G$ has no cycle going through $x$.

    \item[-] If $\textit{rm}(d_1)$ happens before $o$, and $o$ is an operation of $x$. Then since $\textit{put}(d_2,\_) <_{hb} \textit{rm}(d_1)$ and $\textit{rm}(d_1) <_{hb} o$, we can see that $\textit{put}(d_2,\_) <_{hb} o$, and then $\textit{put}(d_2,\_) \in \textit{UVSet}_1(e,x)$, contradicts that $\textit{put}(d_2,\_) \in \textit{UVSet}_2(e,x)$.
    \end{itemize}

    {\bf Base case $2$}: From $T_2$ we can obtain contradiction.

    Let us prove base case $2$: If $\textit{rm}(d_2) <_{hb} o$, and $o$ is an operation of $d_1$, then since $( \textit{put}(d_3,\_) <_{hb} \textit{rm}(d_2) ) \wedge ( \textit{rm}(d_2) <_{hb} o )$, we know that $\textit{put}(d_3,\_) <_{hb} o$. This implies that $\textit{put}(d_3,\_) \in \textit{UVSet}_2(e,x)$, contradicts that $\textit{rm}(d_3,\_) \in \textit{UVSet}_3(e,x)$. Therefore, it is only possible that $\textit{put}(d_2,\_)$ happens before an operation of $d_1$.

    \begin{itemize}
    \setlength{\itemsep}{0.5pt}
    \item[-] If $\textit{put}(d_2,\_) <_{hb} \textit{put}(d_1,\_)$ and $\textit{put}(d_1,\_)$ happens before operations of $x$, then we know that $\textit{put}(d_2,\_)$ happens before operation of $x$, which is impossible.

    \item[-] If $\textit{put}(d_2,\_) <_{hb} \textit{put}(d_1,\_)$ and $\textit{rm}(d_1)$ happens before operations of $x$, then by interval order, we know that $\textit{put}(d_2,\_)$ happens before operation of $x$, or $\textit{rm}(d_1) <_{hb} \textit{put}(d_1,\_)$, which is impossible.

    \item[-] If $\textit{put}(d_2,\_) <_{hb} \textit{rm}(d_1)$ and $\textit{put}(d_1,\_)$ happens before operations of $x$, then $x \rightarrow d_{\textit{j+1}} \rightarrow \ldots \rightarrow d_1 \rightarrow x$ in $G$, contradicts that $G$ has no cycle going through $x$.

    \item[-] If $\textit{put}(d_2,\_) <_{hb} \textit{rm}(d_1)$ and $\textit{rm}(d_1)$ happens before operations of $x$, then we know that $\textit{put}(d_2,\_)$ happens before operation of $x$, which is impossible.
    \end{itemize}

    {\bf induction step}: Given $\textit{ind} \geq 3$, if from $T_{\textit{ind-1}}$ we can obtain contradiction, then from $T_{\textit{ind}}$ we can also contain contradiction.


    Prove of the induction step: Similarly as base case $2$, we can prove that it is only possible that $\textit{put}(d_{\textit{ind}},\_)$ happens before operations of $d_{\textit{ind-1}}$.

    \begin{itemize}
    \setlength{\itemsep}{0.5pt}
    \item[-] If $\textit{put}(d_{\textit{ind}},\_) <_{hb} \textit{put}(d_{\textit{ind-1}},\_)$ and $\textit{put}(d_{\textit{ind-1}},\_)$ happens before operations of $d_{\textit{ind-2}}$, then we know that $\textit{put}(d_{\textit{ind}})$ happens before operation of $d_{\textit{ind-2}}$, which is impossible.

    \item[-] If $\textit{put}(d_{\textit{ind}},\_) <_{hb} \textit{put}(d_{\textit{ind-1}},\_)$ and $\textit{rm}(d_{\textit{ind-1}})$ happens before operations of $d_{\textit{ind-2}}$, then by interval order, we know that $\textit{put}(d_{\textit{ind}},\_)$ happens before operation of $d_{\textit{ind-2}}$, or $\textit{rm}(d_{\textit{ind-1}}) <_{hb} \textit{put}(d_{\textit{ind-1}},\_)$, which is impossible.

    \item[-] If $\textit{put}(d_{\textit{ind}},\_) <_{hb} \textit{rm}(d_{\textit{ind-1}})$, then we obtain $T_{\textit{ind-1}}$, which already contain contradiction.
    \end{itemize}

    By base case $1$, base case $2$ and the induction step, it is easy to see that for each $i$, $T_i$ contains contradiction. Therefore, $T_j$, the case of $(\textit{rm},\textit{put},\textit{rm})$, contains contradiction.

\item[-] $(\textit{rm},\textit{rm},\textit{put})$: Then $( \textit{rm}(x) <_{hb} \textit{rm}(d_{\textit{j+1}}) ) \wedge ( \textit{rm}(d_{\textit{j+1}}) <_{hb} \textit{put}(d_j,\_) )$. We can see that $( \textit{rm}(x) <_{hb} \textit{put}(d_j,\_) ) \wedge ( \textit{put}(d_j,\_) \in \textit{UVSet}_j(e,x) )$, which contradicts that $\textit{rm}(x)$ does not happen before any operation in $\textit{UVSet}_1(e,x) \cup \ldots \cup \textit{UVSet}_j(e,x)$.

\item[-] $(\textit{rm},\textit{rm},\textit{rm})$: Then $( \textit{rm}(x) <_{hb} \textit{rm}(d_{\textit{j+1}}) ) \wedge ( \textit{rm}(d_{\textit{j+1}}) <_{hb} \textit{rm}(d_j) )$. We can see that $( \textit{rm}(x) <_{hb} \textit{rm}(d_j) ) \wedge ( \textit{rm}(d_j) \in \textit{UVSet}_j(e,x) )$, which contradicts that $\textit{rm}(x)$ does not happen before any operation in $\textit{UVSet}_1(e,x) \cup \ldots \cup \textit{UVSet}_j(e,x)$.
\end{itemize}

This completes the proof for $\textit{UVSet}_{\textit{j+1}}(e,x)$. Therefore, $\textit{rm}(x)$ does not happen before any operation in $\textit{UVSet}(e,x) = \textit{UVSet}_1(e,x) \cup \textit{UVSet}_2(e,x) \cup \ldots$. \qed
\end {proof}

With Lemma \ref{lemma:UVSet has matched put and rm} and Lemma \ref{lemma:Rmx does not happen before UVSet for EPQ1Lar}, we can now prove Lemma \ref{lemma:Lin Equals Constraint for EPQ1Lar}.

\LinEqualsConstraintforEPQOneLar*

\begin {proof}

To prove the $\textit{only if}$ direction, assume that $e \sqsubseteq \textit{MS}(\textit{EPQ}_1^{>})$. Let $u$, $v$ and $w$ be the sequences of method events in $\textit{EPQ}_1^{>}$, and let $U$, $V$ and $W$ be the set of method events of $u$, $v$ and $w$, respectively. Assume by contradiction that, there is a cycle $d_1 \rightarrow d_2 \rightarrow \ldots \rightarrow d_m \rightarrow x \rightarrow d_1$ in $G$. It is obvious that the priority of each $d_i$ is smaller than $\textit{pri}_x$. Then our proof proceeds as follows:

According to the definition of left-right constraint, there are two possibilities. The first possibility is that, $\textit{rm}(x)$ happens before $\textit{rm}(d_1)$. It is obvious that $\textit{rm}(d_1) \in W$, and then since $U \cup V$ contains matched $\textit{put}$ and $\textit{rm}$, we can see that $\textit{put}(d_1),\textit{rm}(d_1) \in W$. Then,

\begin{itemize}
\setlength{\itemsep}{0.5pt}
\item[-] Since $d_1 \rightarrow d_2$, by definition of $G$, we know that $\textit{put}(d_1)$ happens before $\textit{rm}(d_2)$. Since $\textit{put}(d_1) \in W$ and $U \cup V$ contains matched $\textit{put}$ and $\textit{rm}$, we know that $\textit{put}(d_2),\textit{rm}(d_2) \in W$. Similarly, for each $1 \leq i \leq m$, we know that $\textit{put}(d_i),\textit{rm}(d_i) \in W$.

\item[-] Since $d_m \rightarrow x$,
    \begin{itemize}
    \setlength{\itemsep}{0.5pt}
    \item[-] if $\textit{put}(d_m)$ happens before $\textit{put}(x)$, then we can see that $\textit{put}(d_m) \in U$, which contradicts that $\textit{put}(d_m) \in W$.

    \item[-] if $\textit{put}(d_m)$ happens before $\textit{rm}(x)$, then we can see that $\textit{put}(d_m) \in U \cup V$, which contradicts that $\textit{put}(d_m) \in W$.
    \end{itemize}
\end{itemize}

The second possibility is that, $e$ contains one $\textit{put}(d_1,\_)$ and no $\textit{rm}(d_1)$. Note that for each $j > 1$, $e$ contains $\textit{put}(d_j,\_)$ and $\textit{rm}(d_j)$. Since $d_m \rightarrow x$, is is obvious that $\textit{put}(d_m) \in U \cup V$. Since $U \cup V$ contains matched $\textit{put}$ and $\textit{rm}$, we know that $\textit{put}(d_m),\textit{rm}(d_m) \in U \cup V$. Then, since $d_{\textit{m-1}} \rightarrow d_m$, by definition of $G$, we know that $\textit{put}(d_{\textit{m-1}})$ happens before $\textit{rm}(d_m)$. Since $\textit{rm}(d_m) \in U \cup V$ and $U \cup V$ contains matched $\textit{put}$ and $\textit{rm}$, we know that $\textit{put}(d_{\textit{m-1}}),\textit{rm}(d_{\textit{m-1}}) \in U \cup V$. Similarly, for each $1 < i \leq m$, we know that $\textit{put}(d_i),\textit{rm}(d_i) \in U \cup V$, and also $\textit{put}(d_1)\in U \cup V$. However, there is one $\textit{put}(d_1,\_)$ and no $\textit{rm}(d_1)$ in $e$, contradicts that $U \cup V$ contains matched $\textit{put}$ and $\textit{rm}$.

This completes the proof of the $\textit{only if}$ direction.

To prove the $\textit{if}$ direction, assume that $G$ has no cycle going through $x$. Let $E_u$ be the set of operations that happen before $\textit{put}(x)$ in $e$. It is easy to see that $E_u \subseteq \textit{UVSet}(e,x)$. Let $E_v = \textit{UVSet}(e,x) \setminus E_u$. Let $E_e$ be the set of operations of $e$, and let $E_w = E_e \setminus \textit{UVSet}(e,x)$.

By Lemma \ref{lemma:UVSet has matched put and rm}, we can see that $E_u \cup E_v$ contains matched $\textit{put}$ and $\textit{rm}$ operations. It remains to prove that for $E_u$, $\{ \textit{put}(x,\textit{pri}_x) \}$, $E_v$, $\{ \textit{rm}(x) \}$, $E_w$, no elements of the latter set happens before elements of the former set. We prove this by showing that all the following cases are impossible:

\begin{itemize}
\setlength{\itemsep}{0.5pt}
\item[-] Case $1$: Some operation $o_w \in E_w$ happens before $\textit{rm}(x)$. Then we know that $o_w \in \textit{UVSet}(e,x) = E_u \cup E_v$, which contradicts that $o_w \in E_w$.

\item[-] Case $2$: Some operation $o_w \in E_w$ happens before some operation $o_{\textit{uv}} \in E_u \cup E_v$. Then we know that $o_w \in \textit{UVSet}(e,x) = E_u \cup E_v$, which contradicts that $o_w \in E_w$.

\item[-] Case $3$: Some operation $o_w \in E_w$ happens before $\textit{put}(x)$. Then we know that $o_w \in \textit{UVSet}(e,x) = E_u \cup E_v$, which contradicts that $o_w \in E_w$.

\item[-] Case $4$: $\textit{rm}(x)$ happens before some $o_{\textit{uv}} \in \textit{UVSet}(e,x) = E_u \cup E_v$. By Lemma \ref{lemma:Rmx does not happen before UVSet for EPQ1Lar} we know that this is impossible.

\item[-] Case $5$: $\textit{rm}(x)$ happens before $\textit{put}(x)$. This contradicts that each single-priority projection satisfy the FIFO property.

\item[-] Case $6$: Some operation $o_v \in E_v$ happens before $\textit{put}(x)$. Then we know that $o_v \in E_u$, which contradicts that $o_v \in E_v$.

\item[-] Case $7$: Some operation $o_v \in E_v$ happens before some operation $o_u \in E_u$. Then we know that $o_v \in E_u$, which contradicts that $o_v \in E_v$.

\item[-] Case $8$: $\textit{put}(x)$ happens before some operation $o_u \in E_u$. This is impossible.
\end{itemize}

This completes the proof of the $\textit{if}$ direction.

\qed
\end {proof}


Let us begin to represent witness automata that is used for capture the existence of a data-differentiated execution $e$, $e$ has a $\_$-projection $e'$, $\textit{last}(e') = \textit{PQ}_1^{>}$, and there exists a cycle going through the item with maximal priority in $e'$. By data-independence, we can obtain $e_r$ from $e$ by renaming function, which maps such item to be $b$, maps items that cover it to be $a$, and maps other items into $d$. There are four possible enumeration of call and return actions of $\textit{put}(b)$ and $\textit{rm}(b)$. For each of them, we generate a witness automaton.

For the case when $e_r \vert_{b} = \textit{cal}(\textit{put},b,p) \cdot \textit{ret}(\textit{put}) \cdot \textit{cal}(\textit{rm}) \cdot \textit{ret}(\textit{rm},b)$, we generate witness automaton $\mathcal{A}_{\textit{l-lar}}^1$, as shown in \figurename~\ref{fig:automata APQ1Lar-1}. Here $c_1 = c + \textit{ret}(\textit{rm},a)$, $c_2 = c + \textit{cal}(\textit{put},a,\textit{les}_p)$, $c_3 = c_2 + \textit{ret}(\textit{rm},a)$, where $c = \textit{cal}(\textit{put},d,\textit{anyPri}),\textit{ret}(\textit{put},d), \textit{cal}(\textit{rm},d)$, $\textit{ret}(\textit{rm},d),\textit{cal}(\textit{rm},\textit{empty}),\textit{ret}(\textit{rm},\textit{empty})$. The differentiated branch in $\mathcal{A}_{\textit{l-lar}}^1$ comes from the positions of the first $\textit{ret}(\textit{put},a)$.

\begin{figure}[htbp]
  \centering
  \includegraphics[width=1 \textwidth]{figures/PIC_AUTO_PQ1Lar-pprr.pdf}
%\vspace{-10pt}
  \caption{Automaton $\mathcal{A}_{\textit{l-lar}}^1$}
  \label{fig:automata APQ1Lar-1}
\end{figure}

$\mathcal{A}_{\textit{l-lar}}^1$ is used to recognize conditions in \figurename~\ref{fig:his for APQ1Lar-1}. Here for simplicity, we only draw operation of $b$, and the first $\textit{ret}(\textit{put},a)$.


\begin{figure}[htbp]
  \centering
  \includegraphics[width=1 \textwidth]{figures/PIC_HIS_PQ1Lar-pprr.pdf}
%\vspace{-10pt}
  \caption{Conditions recognized by $\mathcal{A}_{\textit{l-lar}}^1$}
  \label{fig:his for APQ1Lar-1}
\end{figure}


For the case when $e_r \vert_{b} = \textit{cal}(\textit{put},b,p) \cdot \textit{cal}(\textit{rm}) \cdot \textit{ret}(\textit{put}) \cdot \textit{ret}(\textit{rm},b)$, we generate witness automaton $\mathcal{A}_{\textit{l-lar}}^2$, as shown in \figurename~\ref{fig:automata APQ1Lar-2}. Here $c_1,c_2,c_3$ is the same as that in $\mathcal{A}_{\textit{l-lar}}^1$. The differentiated branch in $\mathcal{A}_{\textit{l-lar}}^2$ comes from the positions of the first $\textit{ret}(\textit{put},a)$.

\begin{figure}[htbp]
  \centering
  \includegraphics[width=1 \textwidth]{figures/PIC_AUTO_PQ1Lar-prpr.pdf}
%\vspace{-10pt}
  \caption{Automaton $\mathcal{A}_{\textit{l-lar}}^2$}
  \label{fig:automata APQ1Lar-2}
\end{figure}


$\mathcal{A}_{\textit{l-lar}}^2$ is used to recognize conditions in \figurename~\ref{fig:his for APQ1Lar-2}. Here for simplicity, we only draw operation of $b$, and the first $\textit{ret}(\textit{put},a)$.


\begin{figure}[htbp]
  \centering
  \includegraphics[width=0.7 \textwidth]{figures/PIC_HIS_PQ1Lar-prpr.pdf}
%\vspace{-10pt}
  \caption{Conditions recognized by $\mathcal{A}_{\textit{l-lar}}^2$}
  \label{fig:his for APQ1Lar-2}
\end{figure}

For the case when $e_r \vert_{b} = \textit{cal}(\textit{rm}) \cdot \textit{cal}(\textit{put},b,p) \cdot \textit{ret}(\textit{put}) \cdot \textit{ret}(\textit{rm},b)$, we generate witness automaton $\mathcal{A}_{\textit{l-lar}}^3$, as shown in \figurename~\ref{fig:automata APQ1Lar-3}. Here $c_1,c_2,c_3$ is the same as that in $\mathcal{A}_{\textit{l-lar}}^1$. The differentiated branch in $\mathcal{A}_{\textit{l-lar}}^3$ comes from the positions of the first $\textit{ret}(\textit{put},a)$.

\begin{figure}[htbp]
  \centering
  \includegraphics[width=1 \textwidth]{figures/PIC_AUTO_PQ1Lar-rppr.pdf}
%\vspace{-10pt}
  \caption{Automaton $\mathcal{A}_{\textit{l-lar}}^3$}
  \label{fig:automata APQ1Lar-3}
\end{figure}


$\mathcal{A}_{\textit{l-lar}}^3$ is used to recognize conditions in \figurename~\ref{fig:his for APQ1Lar-3}. Here for simplicity, we only draw operation of $b$, and the first $\textit{ret}(\textit{put},a)$.


\begin{figure}[htbp]
  \centering
  \includegraphics[width=0.7 \textwidth]{figures/PIC_HIS_PQ1Lar-rppr.pdf}
%\vspace{-10pt}
  \caption{Conditions recognized by $\mathcal{A}_{\textit{l-lar}}^3$}
  \label{fig:his for APQ1Lar-3}
\end{figure}


For the case when $e_r \vert_{b} = \textit{cal}(\textit{rm}) \cdot \textit{cal}(\textit{put},b,p) \cdot \textit{ret}(\textit{rm},b) \cdot \textit{ret}(\textit{put})$, we generate witness automaton $\mathcal{A}_{\textit{l-lar}}^4$, as shown in \figurename~\ref{fig:automata APQ1Lar-4}. Here $c_1,c_2,c_3$ is the same as that in $\mathcal{A}_{\textit{l-lar}}^1$, and $c_4 = c_1 + \textit{ret}(\textit{put},b)$. The differentiated branch in $\mathcal{A}_{\textit{l-lar}}^4$ comes from the positions of the first $\textit{ret}(\textit{put},a)$.

\begin{figure}[htbp]
  \centering
  \includegraphics[width=0.9 \textwidth]{figures/PIC_AUTO_PQ1Lar-rprp.pdf}
%\vspace{-10pt}
  \caption{Automaton $\mathcal{A}_{\textit{l-lar}}^4$}
  \label{fig:automata APQ1Lar-4}
\end{figure}


$\mathcal{A}_{\textit{l-lar}}^4$ is used to recognize conditions in \figurename~\ref{fig:his for APQ1Lar-4}. Here for simplicity, we only draw operation of $b$, and the first $\textit{ret}(\textit{put},a)$.


\begin{figure}[htbp]
  \centering
  \includegraphics[width=0.7 \textwidth]{figures/PIC_HIS_PQ1Lar-rprp.pdf}
%\vspace{-10pt}
  \caption{Conditions recognized by $\mathcal{A}_{\textit{l-lar}}^4$}
  \label{fig:his for APQ1Lar-4}
\end{figure}

Let $\textit{Auts}_{\textit{1-lar}} = \{ \mathcal{A}_{\textit{l-lar}}^1, \mathcal{A}_{\textit{l-lar}}^2, \mathcal{A}_{\textit{l-lar}}^3, \mathcal{A}_{\textit{l-lar}}^4 \}$. The following lemma states that $\textit{EPQ}_1^{>}$ is co-regular.


\EPQOneLarisCoRegular*

\begin {proof}

We need to prove that, given a data-independence implementation $\mathcal{I}$, $\textit{Auts}_{\textit{1-lar}} \cap \mathcal{I} \neq \emptyset$, if and only if $\exists e \in \mathcal{I}_{\neq},$ $e' \in \textit{proj}(e),$ $\textit{EPQ}_1^{>} \in last(e') \wedge e'$ does not linearizable w.r.t. $\textit{MS}(\textit{EPQ}_1^{>})$.

By Lemma \ref{lemma:pri execution is enough} and Lemma \ref{lemma:Lin Equals Constraint for EPQ1Lar}, we need to prove the following fact:

\noindent {\bf $\textit{fact}_1$}: Given a data-independence implementation $\mathcal{I}$, $\textit{Auts}_{\textit{1-lar}} \cap \mathcal{I} \neq \emptyset$ if and only if $\exists e \in \mathcal{I}_{\neq},$ $e' \in \textit{proj}(e)$, $last(e')=\textit{EPQ}_1^{>}$, $x$ is the item with maximal priority $\textit{pri}$ in $e'$, $e'$ is a $\textit{pri}$-execution. And there is a cycle going through $x$ in $G$, where $G$ is the left-right constraint of $e'$.

\noindent The $\textit{only if}$ direction: Let us consider the case of $\mathcal{A}_{\textit{l-lar}}^1$. Assume that $e_1 \in \mathcal{I}$ is accepted by $\mathcal{A}_{\textit{l-lar}}^1$. By data-independence, there exists data-differentiated execution $e \in \mathcal{I}$ and renaming function $r_1$, such that $e_1 = r_1(e)$. Assume that $r_1$ maps $d$ into $b$ and maps $f_1,\ldots,f_m$ into $a$. Let $e'$ be obtained from $e$ by projection into $\{ d, f_1,\ldots,f_m \}$. Assume that the priority of $b$ is $p$. It is easy to see that $e'$ is a $p$-execution, $\textit{last}(e') = \textit{EPQ}_1^{>}$, and there is a cycle going through $d$ in $G$, where $G$ is the left-right constraint of $e'$. The case of $\mathcal{A}_{\textit{l-lar}}^2$, $\mathcal{A}_{\textit{l-lar}}^3$ and $\mathcal{A}_{\textit{l-lar}}^4$ can be similarly proved.

\noindent The $\textit{if}$ direction: Given such $e$, $e'$ and $x$. Let renaming function $r$ maps $x$ into $b$, maps items cover $x$ into $a$, and maps other items into $d$. By data-independence, $r(e) \in \mathcal{I}$. Then depending on the cases of $r(e) \vert_{b}$, we can see that $r(e)$ is accepted by $\mathcal{A}_{\textit{l-lar}}^1$, $\mathcal{A}_{\textit{l-lar}}^2$, $\mathcal{A}_{\textit{l-lar}}^3$ or $\mathcal{A}_{\textit{l-lar}}^4$. \qed
\end {proof}



\subsection{Proofs and Definitions in Subsection \ref{subsec:co-regular of EPQ1Equal}}
\label{sec:appendix proof and definition in section co-regular of EPQ1Equal}

Let $\textit{Items}(e,p)$ be the set of items with priority $p$ in execution $e$. The following lemma states a method to choose $\textit{itm}$ of $\textit{EPQ}_1^{=}$.

\begin{restatable}{lemma}{MaximalInPBadGPMakePQ1Equal}
\label{lemma:maximal in pb and gap-point make a candidate of EPQ1Equal}
Given a data-differentiated $\textit{pri}$-execution $e$ with $\textit{last}(e) = \textit{EPQ}_1^{=}$. If there exists an item $x$ with priority $\textit{pri}$, such that for each $y \in \textit{Items}(e,\textit{pri})$, (1) $x$ does not $<_{\textit{pb}}$ to $y$, and (2) the right-most gap-point of $x$ is after $\textit{cal}(\textit{put},y,\_)$ and $\textit{cal}(\textit{rm},y)$. Then $e \sqsubseteq \textit{MS}(\textit{EPQ}_1^{=})$.
\end{restatable}

\begin {proof}

Let $o$ be the right-most gap-point of $x$. We locate linearization points of each method event as follows:

\begin{itemize}
\setlength{\itemsep}{0.5pt}
\item[-] Locate the linearization point of $\textit{rm}(x)$ at $o$,

\item[-] If $\textit{put}(x,\textit{pri})$ overlaps with $\textit{rm}(x)$, then locate the linearization point of $\textit{put}(x,\textit{pri})$ just before the linearization point of $\textit{rm}(x)$. Otherwise, $\textit{put}(x,\textit{pri}) <_{\textit{hb}} \textit{rm}(x)$, and we locate the linearization point of $\textit{put}(x,\textit{pri})$ just before its return action.

\item[-] Locate linearization points of method event of each $y \in \textit{Items}(e,\textit{pri})$ (except for $x$) just after the call action of the method event.

\item[-] For item $z$ with priority smaller than $\textit{pri}$. If both $\textit{cal}(\textit{put},z,\_)$ and $\textit{cal}(\textit{rm},z)$ is before $o$, then locate the linearization points of $\textit{put}(z,\_)$ and $\textit{rm}(z)$ just after their call actions. If both $\textit{ret}(\textit{put},z)$ and $\textit{ret}(\textit{rm},z)$ (if exists) is after $o$, then locate the linearization points of $\textit{put}(z,\_)$ and $\textit{rm}(z)$ just before their return actions. Otherwise, $x$ is in interval of $z$, which contradicts the definition of gap-point, and is impossible.
\end{itemize}

Let $l$ be the sequence of linearization points constructed above. It is obvious that $e \sqsubseteq l$. Since for each $y \in \textit{Items}(e,\textit{pri})$, $o$ is after $\textit{cal}(\textit{put},y,\_)$ and $\textit{cal}(\textit{rm},x)$, we can see that $\textit{rm}(x)$ is after $\textit{put}(y,\textit{pri})$ and $\textit{rm}(y)$ in $l$. It is obvious that $\textit{put}(x,\textit{pri})$ is before $\textit{rm}(x)$ in $l$. Since $x$ does not $<_{\textit{pb}}$ to $y$, we can see that no $\textit{put}(y,\textit{pri})$ happens before $\textit{put}(x,\textit{pri})$. Then it is easy to see that $\textit{put}(x,\textit{pri})$ is after $\textit{put}(y,\textit{pri})$ in $l$. Since $\textit{last}(e) = \textit{EPQ}_1^{=}$, all other items in $\textit{Items}(e,\textit{pri})$ has matched $\textit{put}$ and $\textit{rm}$, and it is easy to see that their $\textit{put}$ and $\textit{rm}$ (except for that of $x$) are all before $\textit{rm}(x)$ in $l$.

For item $z$ with priority smaller than $\textit{pri}$, we can see that there are only two possibilities: (1) $\textit{put}(z,\_)$ and $\textit{rm}(z)$ are both before $\textit{rm}(x)$ in $l$, and (2) $\textit{put}(z,\_)$ and $\textit{rm}(z)$ (if exists) are after before $\textit{rm}(x)$ in $l$. Therefore, before $\textit{rm}(x)$ in $l$, the $\textit{put}$ and $\textit{rm}$ of $z$ are matched.

Therefore, it is easy to see that $l \in \textit{MS}(\textit{EPQ}_1^{=})$. \qed
\end {proof}

With Lemma \ref{lemma:maximal in pb and gap-point make a candidate of EPQ1Equal}, we can prove the following lemma, which states that getting rid of case in \figurename~\ref{fig:introduce pb order} is enough for ensure $\textit{last}(e) = \textit{EPQ}_1^{=} \Rightarrow e \sqsubseteq \textit{MS}(\textit{EPQ}_1^{=})$.


\EPQOneEqualAsPBandGP*

\begin {proof}

To prove the $\textit{if}$ direction, let $e_{x,y}$ be the execution that is obtained from $e$ by erasing all actions of items that has same priority as $x$, except for actions of $x$ and $y$. It is obvious that $\textit{last}(e_{x,y}) = \textit{EPQ}_1^{=}$. Since $y <_{\textit{pb}}^* x$, according to $\textit{EPQ}_1^{=}$, we can see that $x$ should be chosen as $\textit{itm}$ in $\textit{EPQ}_1^{=}$.

According to Lemma \ref{lemma:Lin Equals Constraint for EPQ1Lar} (Here we temporarily forget the existence of $y$), the only possible position for locating linearizaton point of $\textit{rm}(x)$ is at gap-point of $x$. Otherwise, if the linearizaton point of $\textit{rm}(x)$ is chosen at a position that is not a gap-point of $x$, then there exists unmatched method event before $\textit{rm}(x)$ with smaller priority. Since the rightmost gap-point of $x$ is before $\textit{cal}(\textit{put},y,\textit{pri})$ or $\textit{cal}(\textit{rm},y)$, if we locate linearizaton point of $\textit{rm}(x)$ at gap-point of $x$, then $\textit{rm}(x)$ will be before $\textit{cal}(\textit{put},y,\textit{pri})$ or $\textit{cal}(\textit{rm},x)$.

Therefore, for every sequence $l = u \cdot \textit{put}(x,\textit{pri}) \cdot v \cdot \textit{rm}(x) \cdot w$, if $e_{x,y} \sqsubseteq l$, then either $u \cdot v$ contains some unmatched method events of priority smaller than $\textit{pri}$, or $w$ contains $\textit{put}(y,\textit{pri})$ or $\textit{rm}(y)$. In both cases, $l \notin \textit{MS}(\textit{EPQ}_1^{=})$.

To prove the $\textit{only if}$ direction, we prove its contrapositive. Assume we already know that for each $x$ and $y$ has maximal priority in $e$, if $y <_{\textit{pb}}^* x$, then the rightmost gap-point of $x$ is after $\textit{cal}(\textit{put},y,\textit{pri})$ and $\textit{cal}(\textit{rm},x)$. We need to prove that $e \sqsubseteq \textit{MS}(\textit{EPQ}_1^{=})$. Recall that we already assume that each single-priority execution has FIFO property, and item with larger priority is not covered by items with smaller priority.

Our proof proceed as follows:

\begin{itemize}
\setlength{\itemsep}{0.5pt}
\item[-] Let $e_{\textit{pri}}$ be the projection of $e$ into operations of priority $\textit{pri}$. Since each single-priority execution has FIFO property, there exists sequence $l_{\textit{pri}}$, such that $e_{\textit{pri}} \sqsubseteq l_{\textit{pri}}$, and when we treat $\textit{put}$ as $\textit{enq}$ and $\textit{rm}$ as $\textit{deq}$, $l_{\textit{pri}}$ belongs to queue.

\item[-] Let $a_1$ be the last inserted item of $l_{\textit{pri}}$.

    Step $1$: Check whether for each $b \in \textit{Items}(e,\textit{pri})$, (1) $a_1$ does not $<_{\textit{pb}}$ to $b$, and (2) the right-most gap-point of $a$ is after $\textit{cal}(\textit{put},b,\textit{pri})$ and $\textit{cal}(\textit{rm},b)$.

    It is easy to see that $a_1$ is of priority $\textit{pri}$, and $a_1$ does not $<_{\textit{pb}}$ to any $b \in \textit{Items}(e,\textit{pri})$. If for each $b \in \textit{Items}(e,\textit{pri})$, the rightmost gap-point of $a_1$ is after $\textit{cal}(\textit{put},b,\textit{pri})$ and $\textit{cal}(\textit{rm},b)$. Then by Lemma \ref{lemma:maximal in pb and gap-point make a candidate of EPQ1Equal}, we can obtain that $e \sqsubseteq \textit{MS}(\textit{EPQ}_1^{=})$.


\item[-] Otherwise, there exists $a_2 \in \textit{Items}(e,\textit{pri})$, such that the rightmost gap-point of $a_1$ is before $\textit{cal}(\textit{put},a_2,\textit{pri})$ or $\textit{cal}(\textit{rm},a_2)$ in $e$. We can see that each gap-point of $a_2$ is after the rightmost gap-point of $a_1$.%, and thus, the right-most gap-point of $a_2$ is after the rightmost gap-point of $a_1$.
    By assumption, we know that $a_2$ does not $<_{\textit{pb}}$ to $a_1$.

    \begin{itemize}
    \setlength{\itemsep}{0.5pt}
    \item[-] If for each item $b \in \textit{Items}(e,\textit{pri})$, $a_2$ does not $<_{\textit{pb}}$ to $b$. Then we go to step $1$ and treat $a_2$ similarly as $a_1$.
    \item[-] Otherwise, there exists $a_3$ with priority $\textit{pri}$ such that $a_2 <_{\textit{pb}}^* a_3$.

    Since $l_{\textit{pri}}$ has FIFO property, it is easy to see that there is no cycle in $<_{\textit{pb}}$ order. It is safe to assume that $a_3$ is maximal in the sense of $<_{\textit{pb}}^*$. Or we can say, there does not exists $a_4$, such that $a_3 <_{\textit{pb}}^* a_4$.

    By assumption,we know that the rightmost gap-point of $a_3$ is after $\textit{cal}(\textit{put},a_2,\textit{pri})$ and $\textit{cal}(\textit{rm},a_2)$. Therefore, we can see that the rightmost gap-point of $a_3$ is after the rightmost gap-point of $a_1$. Then we go to step $1$ and treat $a_3$ similarly as $a_1$.
    \end{itemize}
\end{itemize}

Let $a^i$ be the $a_1$ in the $\textit{i-th}$ loop of our proof. It is not hard to see that, given $i<j$, the rightmost gap-point of $a^j$ is after the rightmost gap-point of $a^i$. Therefore, the loop finally stop at some $a^f$. $a^f$ satisfies the check of Step $1$. By Lemma \ref{lemma:maximal in pb and gap-point make a candidate of EPQ1Equal}, this implies that $e \sqsubseteq \textit{MS}(\textit{EPQ}_1^{=})$. This completes the proof of $\textit{if}$ direction. \qed
\end {proof}



According to the definition of $<_{\textit{ob}}^*$, if $a <_{\textit{pb}}^* b$, then there exists $a_1,\ldots,a_m$, such that $a <_{\textit{pb}} a_1 <_{\textit{pb}} \ldots <_{\textit{pb}} a_m <_{\textit{pb}} b$. The following lemma states that, the number of intermediate items $a_i$ is in fact bounded.

\OBOrderHasBoundedLength*

\begin {proof}

Our proof proceed as follows:

\begin{itemize}
\setlength{\itemsep}{0.5pt}
\item[-] ($<_{\textit{pb}}^A \cdot <_{\textit{pb}}^A$,$<_{\textit{pb}}^B \cdot <_{\textit{pb}}^B$ and $<_{\textit{pb}}^C \cdot <_{\textit{pb}}^C$): If $c_3 <_{\textit{pb}}^A c_2 <_{\textit{pb}}^A c_1$, then $\textit{put}(c_3,\_)$ happens before $\textit{put}(c_2,\_)$, and $\textit{put}(c_2,\_)$ happens before $\textit{put}(c_1,\_)$. Therefore, it is obvious that $\textit{put}(c_3,\_)$ happens before $\textit{put}(c_1,\_)$ and $c_3 <_{\textit{pb}}^A c_1$.

    Similarly, if $c_3 <_{\textit{pb}}^B c_2 <_{\textit{pb}}^B c_1$, then $c_3 <_{\textit{pb}}^B c_1$.

    If $c_3 <_{\textit{pb}}^C c_2 <_{\textit{pb}}^C c_1$: Since $c_2 <_{\textit{pb}}^C c_1$, $\textit{ret}(\textit{rm},c_2)$ is before $\textit{cal}(\textit{put},c_1,\_)$. Since $\textit{rm}(c_2)$ does not happen before $\textit{put}(c_2,\_)$, $\textit{cal}(\textit{put},c_2,\_)$ is before $\textit{ret}(\textit{rm},c_2)$. Since $c_3 <_{\textit{pb}}^C c_2$, $\textit{ret}(\textit{rm},c_3)$ is before $\textit{cal}(\textit{put},c_2,\_)$. Therefore, $\textit{ret}(\textit{rm},c_3)$ is before $\textit{cal}(\textit{put},c_1,\_)$, and $c_3 <_{\textit{pb}}^C c_1$.

    Therefore, when we meet successive $<_{\textit{pb}}^A$, it is safe to leave only the first and the last elements and ignore intermediate elements. Similar cases hold for $<_{\textit{pb}}^B$ and $<_{\textit{pb}}^C$.

\item[-] $<_{\textit{pb}}^A$ and $<_{\textit{pb}}^C$:

    \begin{itemize}
    \setlength{\itemsep}{0.5pt}
    \item[-] ($<_{\textit{pb}}^A \cdot <_{\textit{pb}}^C$): If $c_3 <_{\textit{pb}}^A c_2 <_{\textit{pb}}^C c_1$. Since $c_2 <_{\textit{pb}}^C c_1$, $\textit{ret}(\textit{rm},c_2)$ is before $\textit{cal}(\textit{put},c_1,\_)$. Since $\textit{rm}(c_2)$ does not happen before $\textit{put}(c_2,\_)$, $\textit{cal}(\textit{put},c_2,\_)$ is before $\textit{ret}(\textit{rm},c_2)$. Since $c_3 <_{\textit{pb}}^A c_2$, $\textit{ret}(\textit{put},c_3)$ is before $\textit{cal}(\textit{put},c_2,\_)$. Therefore, $\textit{ret}(\textit{put},c_3)$ is before $\textit{cal}(\textit{put},c_1,\_)$, and $c_3 <_{\textit{pb}}^A c_1$.

    \item[-] ($<_{\textit{pb}}^C \cdot <_{\textit{pb}}^A$): If $c_3 <_{\textit{pb}}^C c_2 <_{\textit{pb}}^A c_1$. Since $c_2 <_{\textit{pb}}^A c_1$, $\textit{ret}(\textit{put},c_2)$ is before $\textit{cal}(\textit{put},c_1,\_)$. It is obvious that $\textit{cal}(\textit{put},c_2,\_)$ is before $\textit{ret}(\textit{put},c_2)$. Since $c_3 <_{\textit{pb}}^C c_2$, $\textit{ret}(\textit{rm},c_3)$ is before $\textit{cal}(\textit{put},c_2,\_)$. Therefore, $\textit{ret}(\textit{rm},c_3)$ is before $\textit{cal}(\textit{put},c_1,\_)$, and $c_3 <_{\textit{pb}}^C c_1$.

    %Since $c_2 <_{\textit{pb}}^A c_1$, $\textit{put}(c_2,\_)$ happens before $\textit{put}(c_1,\_)$. Since $c_3 <_{\textit{pb}}^C c_2$, $\textit{rm}(c_3)$ happens before $\textit{put}(c_2,\_)$. Therefore, $\textit{rm}(c_3)$ happens before $\textit{put}(c_1,\_)$, and $c_3 <_{\textit{pb}}^C c_1$.
    \end{itemize}

\item[-] $<_{\textit{pb}}^B$ and $<_{\textit{pb}}^C$:

    \begin{itemize}
    \setlength{\itemsep}{0.5pt}
    \item[-] ($<_{\textit{pb}}^B \cdot <_{\textit{pb}}^C$): If $c_3 <_{\textit{pb}}^B c_2 <_{\textit{pb}}^C c_1$. Since $c_2 <_{\textit{pb}}^C c_1$, $\textit{ret}(\textit{rm},c_2)$ is before $\textit{cal}(\textit{put},c_1,\_)$. It is obvious that $\textit{cal}(\textit{rm},c_2)$ is before $\textit{ret}(\textit{rm},c_2)$. Since $c_3 <_{\textit{pb}}^B c_2$, $\textit{ret}(\textit{rm},c_3)$ is before $\textit{cal}(\textit{rm},c_2)$. Therefore, $\textit{ret}(\textit{rm},c_3)$ is before $\textit{cal}(\textit{put},c_1,\_)$, and $c_3 <_{\textit{pb}}^C c_1$.

    \item[-] ($<_{\textit{pb}}^C \cdot <_{\textit{pb}}^B$): If $c_3 <_{\textit{pb}}^C c_2 <_{\textit{pb}}^B c_1$. Since $c_2 <_{\textit{pb}}^B c_1$, $\textit{ret}(\textit{rm},c_2)$ is before $\textit{cal}(\textit{rm},c_1)$. Since $\textit{rm}(c_2)$ does not happen before $\textit{put}(c_2,\_)$, $\textit{cal}(\textit{put},c_2,\_)$ is before $\textit{ret}(\textit{rm},c_2)$. Since $c_3 <_{\textit{pb}}^C c_2$, $\textit{ret}(\textit{rm},c_3)$ is before $\textit{cal}(\textit{put},c_2,\_)$. Therefore, $\textit{ret}(\textit{rm},c_3)$ is before $\textit{cal}(\textit{rm},c_1)$, and $c_3 <_{\textit{pb}}^B c_1$.
    \end{itemize}

\item[-]  ($<_{\textit{pb}}^A \cdot <_{\textit{pb}}^B \cdot <_{\textit{pb}}^A$): If $c_4 <_{\textit{pb}}^A c_3 <_{\textit{pb}}^B c_2 <_{\textit{pb}}^A c_1$:
    \begin{itemize}
    \setlength{\itemsep}{0.5pt}
    \item[-] If $\textit{cal}(\textit{rm},c_2)$ is before $\textit{cal}(\textit{put},c_1,\_)$: Since $c_3 <_{\textit{pb}}^B c_2$, $\textit{ret}(\textit{rm},c_3)$ is before $\textit{cal}(\textit{rm},c_2)$. Then $\textit{ret}(\textit{rm},c_3)$ is before $\textit{cal}(\textit{put},c_1,\_)$, and $c_3 <_{\textit{pb}}^C c_1$. This implies that $c_4 <_{\textit{pb}}^A c_3 <_{\textit{pb}}^C c_1$. According to the fact for $<_{\textit{pb}}^A \cdot <_{\textit{pb}}^C$, we know that $c_4  <_{\textit{pb}}^A c_1$.

    \item[-] If $\textit{cal}(\textit{rm},c_2)$ is after $\textit{cal}(\textit{put},c_1,\_)$: Since $c_2 <_{\textit{pb}}^A c_1$, $\textit{ret}(\textit{put},c_2,\_)$ is before $\textit{cal}(\textit{put},c_1,\_)$. Since $c_3 <_{\textit{pb}}^B c_2$, $\textit{rm}(c_3)$ happens before $\textit{rm}(c_2)$, and then we know that $\textit{put}(c_2,\_)$ can not happen before $\textit{put}(c_3,\_)$. Since $\textit{put}(c_2,\_)$ does not happen before $\textit{put}(c_3,\_)$, $\textit{cal}(\textit{put},c_3,\_)$ is before $\textit{ret}(\textit{put},c_2,\_)$. Since $c_4 <_{\textit{pb}}^A c_3$, $\textit{ret}(\textit{put},c_4)$ is before $\textit{cal}(\textit{put},c_3,\_)$. Therefore, $\textit{ret}(\textit{put},c_4)$ is before $\textit{cal}(\textit{put},c_1,\_)$, and $c_4 <_{\textit{pb}}^A c_1$.
    \end{itemize}

\item[-]  ($<_{\textit{pb}}^B \cdot <_{\textit{pb}}^A \cdot <_{\textit{pb}}^B$): If $c_4 <_{\textit{pb}}^B c_3 <_{\textit{pb}}^A c_2 <_{\textit{pb}}^B c_1$: Since $c_2 <_{\textit{pb}}^B c_1$, $\textit{ret}(\textit{rm},c_2)$ is before $\textit{cal}(\textit{rm},c_1)$. Since $c_3 <_{\textit{pb}}^A c_2$, we can see that $\textit{put}(c_3,\_) <_{\textit{hb}} \textit{put}(c_2,\_)$. Since each single-priority execution has FIFO property, we know that $\textit{rm}(c_2)$ does not happen before $\textit{rm}(c_3)$, and thus, $\textit{cal}(\textit{rm},c_3)$ is before $\textit{ret}(\textit{rm},c_2)$. Since $c_4 <_{\textit{pb}}^B c_3$, $\textit{ret}(\textit{rm},c_4)$ is before $\textit{cal}(\textit{rm},c_3)$. Therefore, $\textit{ret}(\textit{rm},c_4)$ is before $\textit{cal}(\textit{rm},c_1)$, and $c_4 <_{\textit{pb}}^B c_1$.

\end{itemize}

Based on above results, given $a <_{\textit{pb}}^{b_1} a_1 <_{\textit{pb}} \ldots <_{\textit{pb}}^{b_m} a_m <_{\textit{pb}}^{b_{\textit{m+1}}} b$, where each $b_i$ is in $\{ A,B,C \}$, we can merge relations, until we got one of the following facts:

\begin{itemize}
\setlength{\itemsep}{0.5pt}
\item[-] $a <_{\textit{pb}}^A b$, $a <_{\textit{pb}}^B b$ or $a <_{\textit{pb}}^C b$,

\item[-] $a <_{\textit{pb}}^A a_i <_{\textit{pb}}^B b$, or $a <_{\textit{pb}}^B a_i <_{\textit{pb}}^A b$, for some $i$,
\end{itemize}

This completes the proof of this lemma. \qed
\end {proof}

There are many enumerations of method events of $a$, $b$ and $a_1$ that may makes $a <_{\textit{pb}}^* b$. The following lemma states that with the help of gap-points, the number of potential enumerations can be further reduced into only five.

\FiveEnmuerationisEnoughForEPQOneEqual*

\begin {proof}

Let us prove by consider all the possible reason of $a <_{\textit{pb}}^* b$. According to Lemma \ref{lemma:ob order has bounded length}, we need to consider five reasons: Let $o$ be the right-most gap-point of $b$.

\begin{itemize}
\setlength{\itemsep}{0.5pt}
\item[-] Reason $1$, $a <_{\textit{pb}}^A b$:

    Since $a <_{\textit{pb}}^A b$, $\textit{put}(a,\textit{pri}) <_{\textit{hb}} \textit{put}(b,\textit{pri})$. Since $o$ is after $\textit{cal}(\textit{put},b,\textit{pri})$, and thus, after $\textit{cal}(\textit{put},a,\textit{pri})$, we can see that $o$ is before $\textit{cal}(\textit{rm},b)$.

    Since single-priority execution must satisfy the FIFO property, $\textit{rm}(b)$ does not happen before $\textit{rm}(a)$, and thus, $\textit{cal}(\textit{rm},a)$ is before $\textit{ret}(\textit{rm},b)$. If $\textit{cal}(\textit{rm},a)$ is before $\textit{cal}(\textit{rm},b)$, then $o$ is also a gap-point of $a$ and contradicts our assumption. So we know that $\textit{cal}(\textit{rm},a)$ is after $\textit{cal}(\textit{rm},b)$. If $\textit{ret}(\textit{rm},a)$ is before $\textit{ret}(\textit{rm},b)$, since we already assume that there exists gap-point of $a$, this gap-point is also a gap-point of $b$, and is after $o$, which contradicts that $o$ is the rightmost gap-point of $b$. Therefore, $\textit{ret}(\textit{rm},a)$ is after $\textit{ret}(\textit{rm},b)$.

    According to above discussion, there are two possible enumeration of operations of $a$ and $b$, as shown in \figurename~\ref{fig:history enumeration 1 for PQ1Equal} and \figurename~\ref{fig:history enumeration 2 for PQ1Equal}. Here we explicitly draw the leftmost gap-point of $a$ as $o'$. Since the position of $\textit{ret}(\textit{put},b)$ does not influence the correctness, we can simply ignore it.

\item[-] Reason $2$, $a <_{\textit{pb}}^B b$:

    Since $a <_{\textit{pb}}^B b$, $\textit{ret}(\textit{rm},a)$ is before $\textit{cal}(\textit{rm},b)$. Since $o$ is after $\textit{cal}(\textit{rm},b)$, we can see that $o$ is before $\textit{cal}(\textit{put},a,\textit{pri})$. This implies that $\textit{ret}(\textit{rm},a)$ is before $\textit{cal}(\textit{put},a,\textit{pri})$, and then $\textit{rm}(a) <_{\textit{hb}} \textit{put}(a)$, which is impossible. Therefore, we can safely ignore this reason.

\item[-] Reason $3$, $a <_{\textit{pb}}^C b$:

    Since $a <_{\textit{pb}}^B b$, $\textit{ret}(\textit{rm},a)$ is before $\textit{cal}(\textit{put},b,\textit{pri})$. Since $o$ is after $\textit{cal}(\textit{put},b)$, we can see that $o$ is before $\textit{cal}(\textit{put},a,\textit{pri})$. This implies that $\textit{ret}(\textit{rm},a)$ is before $\textit{cal}(\textit{put},a,\textit{pri})$, and then $\textit{rm}(a) <_{\textit{hb}} \textit{put}(a)$, which is impossible. Therefore, we can safely ignore this reason.

\item[-] Reason $4$, $a <_{\textit{pb}}^A a_1 <_{\textit{pb}}^B b$:

    Since $a_1 <_{\textit{pb}}^B b$, $\textit{rm}(a_1) <_{\textit{hb}} \textit{rm}(b)$, and $\textit{ret}(\textit{rm},a_1)$ is before $\textit{cal}(\textit{rm},b)$. Since $\textit{rm}(a_1)$ does not happen before $\textit{put}(a_1)$, $\textit{cal}(\textit{put},a_1,\textit{pri})$ is before $\textit{ret}(\textit{rm},a_1)$. Since $a <_{\textit{pb}}^A a_1$, $\textit{ret}(\textit{put},a,\textit{pri})$ is before $\textit{cal}(\textit{put},a_1,\textit{pri})$. Therefore, $\textit{ret}(\textit{put},a,\textit{pri})$ is before $\textit{cal}(\textit{rm},b)$. Since $\textit{cal}(\textit{rm},b)$ is before $o$, we can see that $o$ is before $\textit{cal}(\textit{rm},a)$.

    If $\textit{cal}(\textit{rm},a)$ is after $\textit{ret}(\textit{rm},b)$, then $e \vert_{ \{ a,a_1,b \} }$ violates the FIFO property. Therefore, $\textit{cal}(\textit{rm},a)$ is before $\textit{ret}(\textit{rm},b)$. Similarly as the case of reason $1$, we can see that $\textit{ret}(\textit{rm},b)$ is before $\textit{ret}(\textit{rm},a)$.

    According to above discussion, there are three possible enumeration of operations of $a$, $a_1$ and $b$, as shown in \figurename~\ref{fig:history enumeration 3 for PQ1Equal}, \figurename~\ref{fig:history enumeration 4 for PQ1Equal} and \figurename~\ref{fig:history enumeration 5 for PQ1Equal}. Here we explicitly draw the leftmost gap-point of $a$ as $o'$. Since the position of $\textit{ret}(\textit{put},a_1,\textit{pri})$ and $\textit{cal}(\textit{put},a,\textit{pri})$ do not influence the correctness, we can simply ignore it. We also ignore $\textit{cal}(\textit{put},b,\textit{pri})$ and $\textit{ret}(\textit{put},b)$, since the only requirements of them are (1) $\textit{rm}(b)$ does not happen before $\textit{put}(b)$ and (2) $\textit{cal}(\textit{put},b,\textit{pri})$ is before $o$.

\item[-] Reason $5$, $a <_{\textit{pb}}^B a_1 <_{\textit{pb}}^A b$:

    Since $a_1 <_{\textit{pb}}^A b$, $\textit{ret}(\textit{put},a_1)$ is before $\textit{call}(\textit{put},b,\textit{pri})$. Since $\textit{call}(\textit{put},b,\textit{pri})$ is before $o$, we can see that $\textit{ret}(\textit{put},a_1)$ is before $o$.

    \begin{itemize}
    \setlength{\itemsep}{0.5pt}
    \item[-] If $o$ is before $\textit{cal}(\textit{rm},a)$: Then $o$ is obviously before $\textit{ret}(\textit{rm},a)$. Since $a <_{\textit{pb}}^B a_1$, $\textit{ret}(\textit{rm},a)$ is before $\textit{cal}(\textit{rm},a_1)$. Then we can see that, $o$ is before $\textit{cal}(\textit{rm},a_1)$, and remember that $a_1 <_{\textit{pb}}^A b$. Then we can goto the case of reason $1$ and treat $a_1$ as $a$. Therefore, we can safely ignore this.

    \item[-] If $o$ is before $\textit{cal}(\textit{put},a,\textit{pri})$: Since $\textit{rm}(a)$ does not happen before $\textit{put}(a,\textit{pri})$, we can see that $\textit{cal}(\textit{put},a,\textit{pri})$ is before $\textit{ret}(\textit{rm},a)$, and then $o$ is before $\textit{ret}(\textit{rm},a)$. Then similarly as above case, we can see that $o$ is before $\textit{cal}(\textit{rm},a_1)$, and $a_1 <_{\textit{pb}}^A b$. Then we can goto the case of reason $1$ and treat $a_1$ as $a$. Therefore, we can safely ignore this.
    \end{itemize}
\end{itemize}

This completes the proof of this lemma. \qed
\end {proof}

\begin{figure}[htbp]
  \centering
  \includegraphics[width=0.4 \textwidth]{figures/PIC-HIS-PQ1Equal-1.pdf}
%\vspace{-10pt}
  \caption{The first possible enumeration.}
  \label{fig:history enumeration 1 for PQ1Equal}
\end{figure}


\begin{figure}[htbp]
  \centering
  \includegraphics[width=0.4 \textwidth]{figures/PIC-HIS-PQ1Equal-2.pdf}
%\vspace{-10pt}
  \caption{The second possible enumeration.}
  \label{fig:history enumeration 2 for PQ1Equal}
\end{figure}

\begin{figure}[htbp]
  \centering
  \includegraphics[width=0.4 \textwidth]{figures/PIC-HIS-PQ1Equal-3.pdf}
%\vspace{-10pt}
  \caption{The third possible enumeration.}
  \label{fig:history enumeration 3 for PQ1Equal}
\end{figure}

\begin{figure}[htbp]
  \centering
  \includegraphics[width=0.4 \textwidth]{figures/PIC-HIS-PQ1Equal-4.pdf}
%\vspace{-10pt}
  \caption{The forth possible enumeration.}
  \label{fig:history enumeration 4 for PQ1Equal}
\end{figure}

\begin{figure}[htbp]
  \centering
  \includegraphics[width=0.4 \textwidth]{figures/PIC-HIS-PQ1Equal-5.pdf}
%\vspace{-10pt}
  \caption{The fifth possible enumeration.}
  \label{fig:history enumeration 5 for PQ1Equal}
\end{figure}


Let us begin to represent several witness automata that is used to capture the existence of a data-differentiated $\_$-execution $e$, $e$ has a projection $e'$, $\textit{last}(e') = \textit{EPQ}_1^{=}$, there exists items $a$ and $b$ with maximal priority in $e'$, $a <_{\textit{pb}}^* b$, and the rightmost gap-point of $b$ is before $\textit{cal}(\textit{put},a,\_)$ or $\textit{cal}(\textit{rm},a)$.

Given a data-differentiated $\_$-execution $e$, two actions $\textit{act}_1$, $\textit{act}_2$ of maximal priority in $e$, and assume that $\textit{act}_1$ is before $\textit{act}_2$ in $e$.
we say that $\textit{act}_1$, $\textit{act}_2$ is covered by items $d_1,\ldots,d_m$ in $e$, if the priorities of $d_1,\ldots,d_m$ is smaller than that of $\textit{act}_1$ and $\textit{act}_2$, and

\begin{itemize}
\setlength{\itemsep}{0.5pt}
\item[-] $\textit{ret}(\textit{put},d_m,\_)$ is before $\textit{act}_1$,

\item[-] For each $i < 1 \leq m$,$\textit{put}(d_{\textit{i-1}},\_)$ happens before $\textit{rm}(d_i)$,

\item[-] $\textit{act}_2$ is before $\textit{cal}(\textit{rm},d_1)$.
\end{itemize}

According to Lemma \ref{lemma:EPQ1Equal as pb order and gap-point}, Lemma \ref{lemma:ob order has bounded length} and Lemma \ref{lemma:five enumeration is enough for EPQ1Equal}, it is not hard to prove that, given a data-differentiated $\_$-execution $e$ with $\textit{last}(e) = \textit{EPQ}_1^{=}$, $e$ does not linearizable with respect to $\textit{EMS}(\textit{PQ}_1^{=})$, if and only if, one of enumerations holds in $e$ (permit renaming), while $\textit{cal}(\textit{rm},a)$ and $\textit{ret}(\textit{rm},b)$ is covered by some $d_1,\ldots,d_m$, $\textit{cal}(\textit{rm},b)$ is before $\textit{ret}(\textit{put},d_m,\_)$, and $\textit{cal}(\textit{rm},d_1)$ is before $\textit{ret}(\textit{rm},a)$. We say that such $d_1,\ldots,d_m$ constitute the rightmost gap of $b$.


An automaton $\mathcal{A}_{\textit{l-eq}}^1$ is given in \figurename~\ref{fig:automata for first enumeration of PQ1Equal}, and it is constructed for the first enumeration in \figurename~\ref{fig:history enumeration 1 for PQ1Equal}. Here we rename the items that covers $\textit{cal}(\textit{rm},a)$ and $\textit{ret}(\textit{rm},b)$ into $d$, and rename the remanning items into $e$. In this figure, $c = \textit{cal}(\textit{put},e,\textit{anyPri}),\textit{ret}(\textit{put},e)$, $\textit{cal}(\textit{rm},e), \textit{ret}(\textit{rm},e),\textit{cal}(\textit{rm},\textit{empty}),\textit{ret}(\textit{rm},\textit{empty})$, $c_1 = c + \textit{cal}(\textit{put},d,\textit{les}_p)$, $c_2 = c_1 + \textit{ret}(\textit{put},b)$, $c_3 = c_2 + \textit{cal}(\textit{put},d),\textit{ret}(\textit{rm},d)$, $c_4 = c + \textit{ret}(\textit{put},b) + \textit{ret}(\textit{rm},d)$.

\begin{figure}[htbp]
  \centering
  \includegraphics[width=0.8 \textwidth]{figures/PIC_AUTO_PQ1Equ-1.pdf}
%\vspace{-10pt}
  \caption{Automaton $\mathcal{A}_{\textit{l-eq}}^1$}
  \label{fig:automata for first enumeration of PQ1Equal}
\end{figure}


An automaton $\mathcal{A}_{\textit{l-eq}}^2$ is given in \figurename~\ref{fig:automata for second enumeration of PQ1Equal}, and it is constructed for the second enumeration in \figurename~\ref{fig:history enumeration 2 for PQ1Equal}. In \figurename~\ref{fig:automata for second enumeration of PQ1Equal}, $c_1$, $c_2$, $c_3$ and $c_4$ is same as that in \figurename~\ref{fig:automata for first enumeration of PQ1Equal}.

\begin{figure}[htbp]
  \centering
  \includegraphics[width=0.8 \textwidth]{figures/PIC_AUTO_PQ1Equ-2.pdf}
%\vspace{-10pt}
  \caption{Automaton $\mathcal{A}_{\textit{l-eq}}^2$}
  \label{fig:automata for second enumeration of PQ1Equal}
\end{figure}

For the third enumeration in \figurename~\ref{fig:history enumeration 3 for PQ1Equal}. Since we want to ensure that $a$ and $b$ are putted only once, we need to explicitly record the positions of $\textit{cal}(\textit{put},a,p)$ and $\textit{cal}(\textit{put},b,p)$. Since the positions of $\textit{cal}(\textit{put},a,p)$ and $\textit{cal}(\textit{put},b,p)$ are not fixed, there are finite possible cases to consider, as shown below:

\begin{itemize}
\setlength{\itemsep}{0.5pt}
\item[-] If $\textit{cal}(\textit{put},b,p)$ is after $\textit{cal}(\textit{rm},b)$ and before $\textit{cal}(\textit{rm},a)$: There are two possible positions of $\textit{cal}(\textit{put},a,p)$: (1) before $\textit{cal}(\textit{rm},a_1)$, and (2) after $\textit{cal}(\textit{rm},a_1)$, and before $\textit{ret}(\textit{put},a)$.

\item[-] If $\textit{cal}(\textit{put},b,p)$ is after $\textit{ret}(\textit{rm},a_1)$ and before $\textit{cal}(\textit{rm},b)$: same as above case.

\item[-] If $\textit{cal}(\textit{put},b,p)$ is after $\textit{cal}(\textit{put},a_1,\_)$ and before $\textit{ret}(\textit{rm},a_1)$: same as above case.

\item[-] If $\textit{cal}(\textit{put},b,p)$ is after $\textit{ret}(\textit{put},a)$ and before $\textit{cal}(\textit{put},a_1,p)$: same as above case.

\item[-] If $\textit{cal}(\textit{put},b,p)$ is after $\textit{cal}(\textit{rm},a_1)$ and before $\textit{ret}(\textit{put},a)$: There are three possible positions of $\textit{cal}(\textit{put},a,p)$: (1) after $\textit{cal}(\textit{put},b,p)$ and before $\textit{ret}(\textit{put},a)$, (2) after $\textit{cal}(\textit{rm},a_1)$ and before $\textit{cal}(\textit{put},b,p)$, and (3) before $\textit{call}(\textit{rm},a_1)$.

\item[-] If $\textit{cal}(\textit{put},b,p)$ is before $\textit{cal}(\textit{rm},a_1)$: There are three possible positions of $\textit{cal}(\textit{put},a,p)$: (1) after $\textit{cal}(\textit{rm},a_1)$ and before $\textit{ret}(\textit{put},a)$, (2) after $\textit{cal}(\textit{put},b,p)$ and before $\textit{cal}(\textit{rm},a_1)$, and (3) before $\textit{cal}(\textit{put},b,p)$.
\end{itemize}

Therefore, there are fourteen possible cases that satisfy the third enumeration in \figurename~\ref{fig:history enumeration 3 for PQ1Equal}. For each case, we construct an finite automaton. Let $\textit{Auts}_{\textit{1-eq}}^{3}$ be the set of finite automata that is constructed for above fourteen cases. For example, for the case $\textit{ca}_1$ when $\textit{cal}(\textit{put},a,p)$ is before $\textit{cal}(\textit{rm},a_1)$, $\textit{cal}(\textit{put},b,p)$ is after $\textit{ret}(\textit{rm},a_1)$, and $\textit{cal}(\textit{put},b,p)$ is before $\textit{cal}(\textit{rm},b)$, we construct a finite automaton $\mathcal{A}_{\textit{l-eq}}^{\textit{3-1}}$ in \figurename~\ref{fig:automata for ca1 of third enumeration of Rpr2}. In \figurename~\ref{fig:automata for ca1 of third enumeration of Rpr2}, let $c$ and $c_1 = c + \textit{cal}(\textit{put},d,\textit{les}_p)$ the same as that in \figurename~\ref{fig:automata for first enumeration of PQ1Equal}. Let $c_2 = c_1 + \textit{ret}(\textit{put},a_1)$, $c_3 = c_2 + \textit{ret}(\textit{put},b)$, $c_4 = c_3 + \textit{cal}(\textit{put},d) + \textit{ret}(\textit{rm},d)$, and $c_5 = c + \textit{ret}(\textit{put},b) + \textit{ret}(\textit{put},a_1) + \textit{ret}(\textit{rm},d)$. The other witness automata in $\textit{Auts}_{\textit{1-eq}}^{3}$ can be similarly constructed.

\begin{figure}[htbp]
  \centering
  \includegraphics[width=0.8 \textwidth]{figures/PIC_AUTO_PQ1Equ-3-1.pdf}
%\vspace{-10pt}
  \caption{Automaton $\mathcal{A}_{\textit{l-eq}}^{\textit{3-1}}$}
  \label{fig:automata for ca1 of third enumeration of Rpr2}
\end{figure}

Similarly, we construct sets $\textit{Auts}_{\textit{1-eq}}^{4}$ and $\textit{Auts}_{\textit{1-eq}}^{5}$ of witness automata for the forth enumeration in \figurename~\ref{fig:history enumeration 4 for PQ1Equal} and the fifth enumeration in \figurename~\ref{fig:history enumeration 5 for PQ1Equal}, respectively.

Let $\textit{Auts}_{\textit{1-eq}} = \{ \mathcal{A}_{\textit{l-eq}}^1, \mathcal{A}_{\textit{l-eq}}^2 \} \cup \textit{Auts}_{\textit{1-eq}}^{3} \cup \textit{Auts}_{\textit{1-eq}}^{4} \cup \textit{Auts}_{\textit{1-eq}}^{5}$. The following lemma states that $\textit{PQ}_1^{=}$ is co-regular.


\EPQOneEqualIsCoRegular*

\begin {proof}

We need to prove that, given a data-independence implementation $\mathcal{I}$, $\textit{Auts}_{\textit{1-eq}} \cap \mathcal{I} \neq \emptyset$, if and only if $\exists e \in \mathcal{I}_{\neq},$ $e' \in \textit{proj}(e),$ $\textit{EPQ}_1^{=} \in \textit{last}(e') \wedge e'$ does not linearizable w.r.t. $\textit{MS}(\textit{EPQ}_1^{=})$.

By Lemma \ref{lemma:pri execution is enough} and Lemma \ref{lemma:EPQ1Equal as pb order and gap-point}, we need to prove the following fact:

\noindent {\bf $\textit{fact}_1$}: Given a data-independence implementation $\mathcal{I}$, $\textit{Auts}_{\textit{1-eq}} \cap \mathcal{I} \neq \emptyset$ if and only if $\exists e \in \mathcal{I}_{\neq},$ $e' \in \textit{proj}(e)$, $\textit{last}(e')=\textit{EPQ}_1^{=}$, $a$ and $b$ are two items with maximal priority $\textit{pri}$ in $e'$, $e'$ is a $\textit{pri}$-execution, $a <_{\textit{pb}}^* b$ in $e'$, and the rightmost gap-point of $b$ is before $\textit{cal}(\textit{put},a,\textit{pri})$ or $\textit{cal}(\textit{rm},a)$ in $e'$.

\noindent The $\textit{only if}$ direction: Assume that $e_1 \in \mathcal{I}$ is accepted by some witness automata in $\textit{Auts}_{\textit{1-eq}}$. By data-independence, there exists data-differentiated execution $e_2 \in \mathcal{I}$ and a renaming function $r$, such that $e_1=r(e_2)$. Since $e_1$ is accepted by some witness automata in  $\textit{Auts}_{\textit{1-eq}}$, let $x$, $y$ and $z$ (if exists) be the items that are renamed into $b$, $a$ and $a_1$ (if exists) by $r$, respectively, and let $d_1,\ldots,d_m$ be the items that are renamed into $d$ by $r$.

let $e'' = e_2 \vert_{ \{ x,y,z,d_1,\ldots,d_m \} }$. It is obvious that $e'' \in \textit{proj}(e_2)$ is a $\textit{pri}$-execution, $\textit{last}(e'') = \textit{EPQ}_1^{=}$. According to our construction of automata in $\textit{Auts}_{\textit{1-eq}}$, it is not hard to see that $x$ and $y$ has maximal priority in $h_2$, $y <_{\textit{pb}}^* x$, and the rightmost gap-point of $x$ is before $\textit{cal}(\textit{put},y,\textit{pri})$ or $\textit{cal}(\textit{rm},y)$ in $e''$.

\noindent The $\textit{if}$ direction: Assume that there exists $e \in \mathcal{I}_{\neq},e' \in \textit{proj}(e)$, such that $last(e')=\textit{PQ}_1^{=}$, $a'$ and $b'$ are two items with maximal priority $\textit{pri}$ in $e'$, $e'$ is a $\textit{pri}$-execution, $a' <_{\textit{pb}}^* b'$ in $e'$, and the rightmost gap-point of $b'$ is before $\textit{cal}(\textit{put},a',\textit{pri})$ or $\textit{cal}(\textit{rm},a')$ in $e'$. By data-independence, we can obtain execution $e_1$ as follows: (1) rename $a'$ and $b'$ into $a$ and $b$, respectively, (2) for the items $d_1,\ldots,d_m$ that constitute the rightmost gap of $b'$, we rename them into $d$, (3) if $a' <_{\textit{pb}}^A a'_1 <_{\textit{pb}}^B b$, we rename $a'_1$ into $a_1$, and (4) rename the other items into $e$. It is easy to see that $\textit{last}(e_1) = \textit{EPQ}_1^{=}$, $a$ and $b$ has maximal priority in $e_1$, $a <_{\textit{pb}}^* b$ in $e_1$, and the rightmost gap-point of $b$ is before $\textit{cal}(\textit{put},a,\textit{pri})$ or $\textit{cal}(\textit{rm},a)$ in $e_1$. By Lemma \ref{lemma:five enumeration is enough for EPQ1Equal}, there are five possible enumeration of operations of $a$, $b$, $a_1$ (if exists). Then


\begin{itemize}
\setlength{\itemsep}{0.5pt}
\item[-] If $a <_{\textit{pb}}^* b$ because of the first enumeration, it is easy to see that $h_1$ is accepted by $\mathcal{A}_{\textit{l-eq}}^1$.

\item[-] If $a <_{\textit{pb}}^* b$ because of the second enumeration, it is easy to see that $h_1$ is accepted by $\mathcal{A}_{\textit{l-eq}}^2$.

\item[-] If $a <_{\textit{pb}}^* b$ because of the third enumeration, it is easy to see that $h_1$ is accepted by some witness automaton in $\textit{Auts}_{\textit{1-eq}}^{3}$.

\item[-] If $a <_{\textit{pb}}^* b$ because of the forth enumeration, it is easy to see that $h_1$ is accepted by some witness automaton in $\textit{Auts}_{\textit{1-eq}}^{4}$.

\item[-] If $a <_{\textit{pb}}^* b$ because of the fifth enumeration, it is easy to see that $h_1$ is accepted by some witness automaton in $\textit{Auts}_{\textit{1-eq}}^{5}$.
\end{itemize}

This completes the proof of this lemma. \qed
\end {proof}





\subsection{Co-Regular of $\textit{EPQ}_2^{>}$}
\label{subsec:appendix co-regular of EPQ2Lar}


\begin{restatable}{lemma}{EPQ2LarIsAlwaysCoRegular}
\label{lemma:EPQ2Lar is always co-regular}
Given a data-differentiated $\_$-execution $e$, if $\textit{last}(e) = \textit{EPQ}_2^{>}$, then $e \sqsubseteq \textit{MS}(\textit{EPQ}_2^{>})$.
\end{restatable}

\begin {proof}
Since $\textit{last}(e) = \textit{EPQ}_2^{>}$, the actions with maximal priority in $e$ is some unmatched $\textit{put}$. Therefore, no matter how we locate linearization points, we can always obtain a sequence $l$ of method events that contains unmatched $\textit{put}$ with maximal priority, and this satisfy the guard of $\textit{MS}(\textit{EPQ}_2^{>})$. This completes the proof of this lemma. \qed
\end {proof}




\subsection{Co-Regular of $\textit{EPQ}_2^{=}$}
\label{subsec:appendix co-regular of EPQ2Equal}

\begin{restatable}{lemma}{EPQ2EqualAsHappenBefore}
\label{lemma:EPQ2Equal as happen before}
Given a data-differentiated $\textit{pri}$-execution $e$ with $\textit{last}(e) = \textit{EPQ}_2^{=}$. $e$ does not linearizable to $\textit{MS}(\textit{EPQ}_2^{=})$, if and only if there exists $x$ and $y$ with priority $\textit{pri}$, $x$ has unmatched $\textit{put}$, $y$ has matched $\textit{put}$ and $\textit{rm}$, and $\textit{put}(x,\textit{pri}) <_{\textit{hb}} \textit{put}(y,\textit{pri})$.
\end{restatable}

\begin {proof}

The $\textit{if}$ direction is obvious.

To prove the $\textit{only if}$ direction, we prove its contrapositive. Assume that for each pair of $x$ and $y$ with maximal priority in $e$, if $x$ has unmatched $\textit{put}$, $y$ has matched $\textit{put}$ and $\textit{rm}$, then $\textit{put}(x,\textit{pri})$ does not happen before $\textit{put}(y,\textit{pri})$. We need to prove that $e \sqsubseteq \textit{MS}(\textit{EPQ}_2^{=})$.

Let $x_1,\ldots,x_m$ be the set of items with priority $\textit{pri}$ and has unmatched $\textit{put}$ in $e$, let $y_1,\ldots,y_n$ be the set of items with priority $\textit{pri}$ and has matched $\textit{put}$ and $\textit{rm}$ in $e$. By assumption, we know that $\textit{cal}(\textit{put},y_i,\textit{pri})$ is before $\textit{ret}(\textit{put},x_j)$ for each $i,j$. Then we explicitly construction the linearization of $e$ by locating the linearization points of $e$ as follows:

\begin{itemize}
\setlength{\itemsep}{0.5pt}
\item[-] For each $x_i$, locate the linearization point of $\textit{put}(x_i,\textit{pri})$ just before its return action.

\item[-] For each $y_j$, locate the lineariztion point of $\textit{put}(y_j,\textit{pri})$ jest after its call action.

\item[-] For other method events, locate their linearization points at an arbitrary location after its call action and before its return action.
\end{itemize}

Let $l$ be the sequence of linearization points. It is easy to see that $e \sqsubseteq l$. Since linearization points of $\textit{put}(x_i,\textit{pri})$ is after the linearization point of $\textit{put}(y_j,\textit{pri})$ for each $i,j$, it is easy to see that $l \in \textit{MS}(\textit{EPQ}_2^{=})$. This completes the proof of this lemma. \qed
\end {proof}

Lemma \ref{lemma:EPQ2Equal as happen before} shows how to check violation to $\textit{MS}(\textit{EPQ}_2^{=})$. However, the case in Lemma \ref{lemma:EPQ2Equal as happen before} violates our assumption that each single-priority execution is FIFO. Therefore, we know that $\textit{EPQ}_2^{=}$ is always co-regular, as states by the following lemma.

\begin{restatable}{lemma}{EPQ2EqualIsAlwaysCoRegular}
\label{lemma:EPQ2Equal is always co-regular}
Given a data-differentiated $\textit{pri}$-execution $e$, if $\textit{last}(e) = \textit{EPQ}_2^{=}$, then $e \sqsubseteq \textit{MS}(\textit{EPQ}_2^{=})$.
\end{restatable}

\begin {proof}

According to Lemma \ref{lemma:EPQ2Equal as happen before}, if $\textit{last}(e) = \textit{EPQ}_2^{=}$ and $e$ does not linearizable to $\textit{MS}(\textit{EPQ}_2^{=})$, then there exists $x$ and $y$ with priority $\textit{pri}$, $x$ has unmatched $\textit{put}$, $y$ has matched $\textit{put}$, and $\textit{put}(x,\textit{pri}) <_{\textit{hb}} \textit{put}(y,\textit{pri})$. Let $e_1 = e \vert_{ \{ x,y \} }$. It is obvious that $e_1$ does not satisfy FIFO property. This contradicts the assumption that every single-priority execution has FIFO property, and thus, we can safely ignore this case. \qed
\end {proof}




\subsection{Co-Regular of $\textit{EPQ}_3$}
\label{subsec:co-regular of EPQ3}

In this subsection we prove that $\textit{EPQ}_3$ is co-regular. The notion of left-right constraint of $\textit{rm}(\textit{empty})$ is inspired by left-right constraint of queue \cite{Bouajjani:2015}.

\begin{definition}\label{def:left-right constraint for rmEmpty operation}
Given a data-differentiated execution $e$, and $o = \textit{rm}(\textit{empty})$ of $e$. The left-right constraint of $o$ is the graph $G$ where:

\begin{itemize}
\setlength{\itemsep}{0.5pt}
\item[-] the nodes are the items of $e$ or $o$, to which we add a node,

\item[-] there is an edge from item $d_1$ to $o$, if $\textit{put}(d_1,\_)$ happens before $o$,

\item[-] there is an edge from $o$ to item $d_1$, if $o$ happens before $\textit{rm}(d_1)$ or $\textit{rm}(d_1)$ does not exists in $h$,

\item[-] there is an edge from item $d_1$ to item $d_2$, if $\textit{put}(d_1,\_)$ happens before $\textit{rm}(d_2,\_)$.
\end{itemize}
\end{definition}


Given a data-differentiated execution $e$ and $o = \textit{rm}(\textit{empty})$ of $e$, it is obvious that $\textit{last}(e) = \textit{EPQ}_3$. Let $\textit{USet}_1(e,o) = \{ \textit{op} \vert$ $\textit{op}$ is an operation of some item, and either $\textit{op} <_{\textit{hb}} o$, or there is $\textit{op}'$ with the same item of $\textit{op}$, such that $\textit{op}' <_{\textit{hb}} o \}$. For each $i \geq 1$, let $\textit{USet}_{\textit{i+1}}(e,o) = \{ \textit{op} \vert$ $\textit{op}$ is an operation of some item, $\textit{op}$ is not in $\textit{USet}_k(e,o)$ for each $k \leq i$, and either $\textit{op}$ happens before some $o' \in \textit{USet}_i(e,o)$, or there is $\textit{op}''$ with the same item of $o$ and $\textit{op}''$ happens before some $o' \in \textit{USet}_i(e,o) \}$. We can see that $\textit{USet}_i(e,o) \cap \textit{USet}_j(e,o) = \emptyset$ for any $i \neq j$. Let $\textit{USet}(e,o) = \textit{USet}_1(e,o) \cup \textit{USet}_2(e,o) \cup \ldots$.


Similarly as $\textit{UVSet}$, we can prove the following two lemmas for $\textit{USet}$.

\begin{restatable}{lemma}{USetHasMatchedPutandRm}
\label{lemma:USet has matched put and rm}
Given a data-differentiated execution $e$ with $\textit{last}(e) = \textit{EPQ}_3$. Let $o$ be a $\textit{rm}(\textit{empty})$ of $e$. Let $G$ be the graph representing the left-right constraint of $o$. Assume that $G$ has no cycle going through $o$. Then, $\textit{USet}(e,o)$ contains only matched $\textit{put}$ and $\textit{rm}$.
\end{restatable}

This Lemma can be similarly proved as Lemma \ref{lemma:UVSet has matched put and rm}.


\begin{restatable}{lemma}{RmxDoesNotHappenBeforeUSetForEPQ3}
\label{lemma:Rmx does not happen before USet for EPQ3}
Given a data-differentiated $\_$execution $e$ with $\textit{last}(e) = \textit{EPQ}_3$. Let $o$ be a $\textit{rm}(\textit{empty})$ of $e$. Let $G$ be the graph representing the left-right constraint of $o$. Assume that $G$ has no cycle going through $o$. Then, $o$ does not happen before any operation in $\textit{USet}(e,o)$.
\end{restatable}

This Lemma can be similarly proved as Lemma \ref{lemma:Rmx does not happen before UVSet for EPQ1Lar}.

Then we can prove that getting rid of cycle though $o$ in left-right constraint is enough for ensure linearizable w.r.t $\textit{MS}(\textit{EPQ}_3)$, as stated by the following lemma.


\begin{restatable}{lemma}{LINEqualsConstraintforEPQ3}
\label{lemma:Lin Equals Constraint for EPQ3}
Given a data-differentiated execution $e$ with $\textit{last}(e) = \textit{EPQ}_3$. $e$ does not linearizable w.r.t $\textit{MS}(\textit{PQ}_3)$, if and only if there exists $o = \textit{rm}(\textit{empty})$ in $e$, $G$ has a cycle going through $o$, where $G$ is the graph representing the left-right constraint of $o$.
\end{restatable}

\begin {proof}

To prove the $\textit{if}$ direction, assume that there is such a cycle. Assume by contradiction that $e \sqsubseteq \textit{MS}(\textit{EPQ}_3)$, and let $U$ and $V$ be the set of operations in $u$ and $v$. Let the cycle be $d_1 \rightarrow d_2 \rightarrow \ldots \rightarrow d_m \rightarrow o \rightarrow d_1$ in $G$. Since $d_m \rightarrow o$, $\textit{put}(d_m,\_)$ happens before $o$, and it is easy to see that $\textit{put}(d_m,\_)$ is in $U$. Since $U$ contains matched $\textit{put}$ and $\textit{rm}$, we can see that operations of $d_m$ is in $U$. Similarly, we can see that method events of $d_{\textit{m-1}},\ldots,d_1$ is in $U$. If $\textit{rm}(d_1)$ does not exists, then this contradicts that $U$ contains matched $\textit{put}$ and $\textit{rm}$. Else, if $\textit{rm}(d_1)$ exists, since $o$ happens before $\textit{rm}(d_1)$, we can see that $\textit{rm}(d_1) \in V$, which contradicts that $\textit{rm}(d_1) \in U$. This completes the proof of the $\textit{if}$ direction.

To prove the $\textit{only if}$ direction, we prove its contrapositive. Assume that for each such $o$ and $G$, $G$ has no cycle going through $o$. Let $O$ be the set of operations of $e$, except for $\textit{rm}(\textit{empty})$. Let $O_L = \textit{USet}(e,o)$, $O_R = O \setminus O_L$.

By Lemma \ref{lemma:USet has matched put and rm}, we can see that $O_L = \textit{USet}(e,o)$ contains only matched $\textit{put}$ and $\textit{rm}$. Let $O'_L$ be the union of $O_L$ and all the $\textit{rm}(\textit{empty})$ that happens before some operations in $O_L \cup \{ o \}$. Let $O'_R$ be the union of $O_R$ and the remanning $\textit{rm}(\textit{empty})$. It remains to prove that for $O'_L$, $\{ o \}$, $O'_R$, no elements of the latter set happens before elements of the former set. We prove this by showing that all the following cases are impossible:

\begin{itemize}
\setlength{\itemsep}{0.5pt}
\item[-] Case $1$: If some operation $o_r \in O'_R$ happens before $o$. Then we can see that $o_r \in \textit{USet}(e,o)$ or is a $\textit{rm}(\textit{empty})$ that happens before $o$, and then $o_r \in O'_L$, which contradicts that $o_r \in O'_R$.

\item[-] Case $2$: If some operation $o_r \in O'_R$ happens before some operation $o_l \in O'_L$. Then we know that $o_r \in \textit{USet}(e,o)$ or is a $\textit{rm}(\textit{empty})$ that happens before some operations in $O_L \cup \{ o \}$, and then $o_r \in O'_L$, which contradicts that $o_r \in O'_R$.

\item[-] Case $3$: If $o$ happens before some $o_l \in O'_L$. If $o_l \in \textit{USet}(e,o)$, then by Lemma \ref{lemma:Rmx does not happen before USet for EPQ3} we know that this is impossible. Else, $o_l$ is a $\textit{rm}(\textit{empty})$ that happens before some operations in $O_L \cup \{ o \}$, and $o$ happens before some operations in $O_L \cup \{ o \}$, which is impossible by Lemma \ref{lemma:Rmx does not happen before USet for EPQ3}.
\end{itemize}

This completes the proof of the $\textit{only if}$ direction.

\qed
\end {proof}

Let us begin to represent an automaton that is used for capture the case that, in a sub-execution $e'$ of an execution $e$, $\textit{last}(e')=\textit{EPQ}_3$, $e'$ does not linearizable to $\textit{MS}(\textit{EPQ}_3)$, and the reason is that there is a cycle going through some $\textit{rm}(\textit{empty})$ $o$ in the left-right constraint of $o$. The automaton is $\mathcal{A}_{\textit{EPQ}}^3$, which is given in \figurename~\ref{fig:automata for PQ3}. In \figurename~\ref{fig:automata for PQ3}, let $c = \textit{cal}(\textit{put},d,\textit{anyPri}),\textit{ret}(\textit{put},d), \textit{cal}(\textit{rm},d), \textit{ret}(\textit{rm},d),\textit{cal}(\textit{rm},\textit{empty}),\textit{ret}(\textit{rm},\textit{empty})$, $c_1 = c + \textit{cal}(\textit{put},b,\textit{anyPri})$, $c_2 = c_1 + \textit{ret}(\textit{rm},b)$, and $c_3 = c + \textit{ret}(\textit{rm},b)$.

\begin{figure}[htbp]
  \centering
  \includegraphics[width=0.8 \textwidth]{figures/PIC_AUTO_PQ3.pdf}
%\vspace{-10pt}
  \caption{Automaton $\mathcal{A}_{\textit{EPQ}}^3$}
  \label{fig:automata for PQ3}
\end{figure}

Given a data-differentiated execution $e$, we say that $o = \textit{rm}(\textit{empty})$ in $e$ is covered by items $d_1,\ldots,d_m$ in $h$, if

\begin{itemize}
\setlength{\itemsep}{0.5pt}
\item[-] $\textit{put}(d_m,\_)$ happens before $o$,

\item[-] For each $i < 1 \leq m$,$\textit{put}(d_{\textit{i-1}},\_)$ happens before $\textit{rm}(d_i)$,

\item[-] $o$ happens before $\textit{rm}(d_1)$, or $\textit{rm}(d_1)$ does not exists in $e$
\end{itemize}

According to the definition of left-right constraint for $o$, in a data-differentiated execution $e$, there is a cycle going through $o$, if and only if there exists items $d_1,\ldots,d_m$, such that $o$ is covered by $d_1,\ldots,d_m$.


\begin{restatable}{lemma}{EPQ3IsCoRegular}
\label{lemma:EPQ3 is co-regular}
$\textit{EPQ}_3$ is co-regular.
\end{restatable}

\begin {proof}

We need to prove that, given a data-independence implementation $\mathcal{I}$, $\mathcal{A}_{\textit{EPQ}}^3 \cap \mathcal{I} \neq \emptyset$ if and only if $\exists e \in \mathcal{I}_{\neq},e' \in \textit{proj}(e), last(e')=\textit{EPQ}_3 \wedge e$ does not linearizable w.r.t. $\textit{MS}(\textit{EPQ}_3)$.

By Lemma \ref{lemma:Lin Equals Constraint for EPQ3}, we need to prove the following fact:

\noindent {\bf $\textit{fact}_1$}: Given a data-independence implementation $\mathcal{I}$, $\mathcal{A}_{\textit{EPQ}}^3 \cap \mathcal{I} \neq \emptyset$ if and only if $\exists e \in \mathcal{I}_{\neq},e' \in \textit{proj}(e), last(e')=\textit{EPQ}_3$, $o = \textit{rm}(\textit{empty})$ is in $e'$, and $o$ is covered by some items $d_1,\ldots,d_m$ in $e'$.


\noindent The $\textit{only if}$ direction: Assume that $e_1 \in \mathcal{I}$ is accepted by $\mathcal{A}_{\textit{EPQ}}^3$. By data-independence, there exists data-differentiated execution $e_2 \in \mathcal{I}$ and a renaming function $r$, such that $e_1=r(e_2)$. Let $d_1,\ldots,d_m$ be the items in $e_2$ such that $r(d_i)=b$ for each $1 \leq i \leq m$. Let $e_3 = e_2 \vert_{ \{ o, d_1, \ldots, d_m \} }$. It is obvious that $e_3 \in \textit{proj}(e_2)$ and $\textit{last}(e_3) = \textit{EPQ}_3$. It is easy to see that $o$ is covered by $d_1,\ldots,d_m$.

\noindent The $\textit{if}$ direction: Assume that there exists such $e$, $e'$, $o$ and $d_1,\ldots,d_m$. Then, let $e_1$ be obtained from $e$ by renaming $d_1,\ldots,d_m$ into $b$ and renaming other items into $d$. By data-independence, $e_1 \in \mathcal{I}$. It is easy to see that $e_1$ is accepted by $\mathcal{A}_{\textit{EPQ}}^3$.

This completes the proof of this lemma. \qed
\end {proof}


\section{Proofs and Definitions in Section \ref{sec:relate other data structures with extended priority queue}}
\label{sec:appendix proof and definition in section relate other data structures with extended priority queue}


\subsection{Proofs and Definitions in Subsection \ref{subsec:relate multiSet with extended priority queue}}
\label{subsec:appendix proof and definition in section relate multiset with extended priority queue}

In this section, we use the notions of inductive rules, $\textit{last}$ of a sequential execution, step-by-step and co-regular in \cite{Bouajjani:2015}. Since they are quite similar to the corresponding notions of extended priority queues in Section \ref{sec:inductive rules of extended priority queue}, Section \ref{sec:step-by-step linearizability of extended priority queues} and Section \ref{sec:co-regular of extended priority queues}, we do not introduce their definitions here.

Let us first define three predicates:

\begin{itemize}
\setlength{\itemsep}{0.5pt}
\item[-] Given a sequential execution $l$ of multi-set, $\textit{noDE}(l)$ is satisfied when each method event of $l$ is not $\textit{delete}(\textit{empty})$.

\item[-] Given a sequential execution $l$ of multi-set, $\textit{matched-MS}(l)$ is satisfied, if (1) for each item $a \in \mathbb{D}$, if $\textit{insert}(a)$ is in $l$, then $\textit{delete}(a)$ is in $l$, and (2) for each item $a \in \mathbb{D}$, if $\textit{delete}(a)$ is in $l$, then $\textit{insert}(a)$ is in $l$.

\item[-] Given a sequential execution $l$ of multi-set, $l \in \textit{Insert}^*$ is satisfied when each method event of $l$ is a $\textit{insert}$ event.
\end{itemize}

Let $\textit{MSet}$ be the set of sequential executions $w$ which can be derived from the empty word by inductive rules of multi-set. $\textit{MSet}$ is defined by the following inductive rules:

\begin{itemize}
\setlength{\itemsep}{0.5pt}
\item[-] $\textit{MSet}_0 \equiv \epsilon \in \textit{MSet}$.

\item[-] $\textit{MSet}_1 \equiv (u \in \textit{MSet}) \wedge
(u \in \textit{Insert}^*)
\Rightarrow
(u \cdot \textit{insert}(\textit{itm}) \in \textit{MSet})$.

\item[-] $\textit{MSet}_2 \equiv
(u \cdot v \cdot w \in \textit{MSet}) \wedge
(\textit{noDE}(u \cdot v \cdot w))
\Rightarrow
(u \cdot \textit{insert}(\textit{itm}) \cdot v \cdot \textit{delete}(\textit{itm}) \cdot w \in \textit{MSet})$.

\item[-] $\textit{MSet}_3 \equiv
(u \cdot v \in \textit{MSet}) \wedge
(\textit{matched-MS}(u) )
\Rightarrow
(u \cdot \textit{delete}(\textit{empty}) \cdot v \in \textit{MSet})$.
\end{itemize}

Thus, given a sequential execution $e$, we define $\textit{last}(e)$ as the last possible rule to generate $e$ according to the rules of multi-set:

\begin{itemize}
\setlength{\itemsep}{0.5pt}
\item[-] If $e$ contains $\textit{delete}(\textit{empty})$, then $\textit{last}(e) = \textit{MS}_3$.

\item[-] Else, if $e$ contains $\textit{delete}$, then $\textit{last}(e) = \textit{MSet}_2$.

\item[-] Else, if $e$ contains only $\textit{insert}$, then $\textit{last}(e) = \textit{MSet}_1$.

\item[-] Else ($e = \epsilon$), $\textit{last}(e) = \textit{MSet}_0$.
\end{itemize}

The following three lemmas state that the rules for multi-set are step-by-step linearizability.

\begin{restatable}{lemma}{MS1isStepByStepLinearizability}
\label{lemma:MS1 is step-by-step linearizability}
If a differentiated concurrent execution $e$ is linearizable w.r.t. $\textit{MS}(\textit{MSet}_1)$ with witness $x$, then $e \setminus x \sqsubseteq \textit{MSet} \Rightarrow e \sqsubseteq \textit{MSet}$.
\end{restatable}

\begin {proof}
Let $h$ be the data-differentiated history of $e$, and $l$ be an sequential execution such that $h \sqsubseteq l$ and $l$ matches $\textit{MSet}_1$ with witness $x$. Let $h'=h \setminus x$ and assume that $h' \sqsubseteq l' \in \textit{MSet}$. Let $e_{\textit{lp}}$ be an execution with linearization points of $e$ and the linearization points is added according to $l'$. Or we can say, $e_{\textit{lp}}$ is generated from $e$ by instrumenting linearization points, and the projection of $e_{\textit{lp}}$ into method event is $l'$.

Let sequence $e'_{\textit{lp}}$ be generated from $e_{\textit{lp}}$ by adding $\textit{insert}(x)$ at an arbitrary time point between $\textit{cal}(\textit{insert},x)$ and $\textit{ret}(\textit{insert},x)$. Let $l''$ be the projection of $e'_{\textit{lp}}$ into method events.

It is easy to see that $h \sqsubseteq l''$. Since $l''$ is obtained from $l'$ by adding one $\textit{insert}(x)$, we can see that $l''$ contains only $\textit{insert}$, and then $l'' \in \textit{MSet}$. \qed
\end {proof}


\begin{restatable}{lemma}{MS2isStepByStepLinearizability}
\label{lemma:MS2 is step-by-step linearizability}
If a differentiated concurrent execution $e$ is linearizable w.r.t. $\textit{MS}(\textit{MSet}_2)$ with witness $x$, then $e \setminus x \sqsubseteq \textit{MSet} \Rightarrow e \sqsubseteq \textit{MSet}$.
\end{restatable}

\begin {proof}
Let $h$ be the data-differentiated history of $e$, and $l$ be an sequential execution such that $h \sqsubseteq l$ and $l$ matches $\textit{MSet}_1$ with witness $x$. Let $h'=h \setminus x$ and assume that $h' \sqsubseteq l' \in \textit{MSet}$. Let $e_{\textit{lp}}$ be an execution with linearization points of $e$ and the linearization points is added according to $l'$. Or we can say, $e_{\textit{lp}}$ is generated from $e$ by instrumenting linearization points, and the projection of $e_{\textit{lp}}$ into method event is $l'$.

It is easy to see that $\textit{delete}(x)$ does not happen before $\textit{insert}(x)$, and then $\textit{cal}(\textit{insert},x)$ is before $\textit{ret}(\textit{delete},x)$ in $e$. Let sequence $e'_{\textit{lp}}$ be generated from $e_{\textit{lp}}$ by adding $\textit{insert}(x)$ just after $\textit{cal}(\textit{insert},x)$ and adding $\textit{delete}(x)$ just before $\textit{ret}(\textit{delete},x)$. Let $l''$ be the projection of $e'_{\textit{lp}}$ into method events.

It is easy to see that $h \sqsubseteq l''$. Since (1) $l' \in \textit{MSet}$, (2) $l''$ is obtained from $l'$ by adding one $\textit{insert}(x)$ and one $\textit{delete}(x)$, while $\textit{insert}(x)$ is before $\textit{delete}(x)$ (3)and $l'$ does not contain $\textit{delete}(\textit{empty})$, we can see that $l'' \in \textit{MSet}$. \qed
\end {proof}


\begin{restatable}{lemma}{MS3isStepByStepLinearizability}
\label{lemma:MS3 is step-by-step linearizability}
If a differentiated concurrent execution $e$ is linearizable w.r.t. $\textit{MS}(\textit{MSet}_3)$ and $o$ is a $\textit{delete}(\textit{empty})$ event, then $e \setminus o \sqsubseteq \textit{MSet} \Rightarrow e \sqsubseteq \textit{MSet}$.
\end{restatable}

\begin {proof}
This Lemma can be similarly proved as Lemma \ref{lemma:EPQ3 is step-by-step linearizability}. \qed
\end {proof}

The following lemma states that $\textit{MSet}_1$ is always co-regular.

\begin{restatable}{lemma}{MS1IsAlwaysCoRegular}
\label{lemma:MS1 is always co-regular}
Given a differentiated execution $e$, if $\textit{last}(e) = \textit{MSet}_1$, then $e \sqsubseteq \textit{MS}(\textit{MSet}_1)$.
\end{restatable}

\begin {proof}
Since $\textit{last}(e) = \textit{MSet}_1$, there are only $\textit{insert}$ in $e$. Then no matter how we locate linearization points of operations of $e$, we can always obtain a sequence in $\textit{MS}(\textit{MSet}_1)$. This completes the proof of this lemma. \qed
\end {proof}

The following lemma shows how to detect violation to $\textit{MS}(\textit{MSet}_2)$.

\begin{restatable}{lemma}{ReduceMS2intoOneValue}
\label{lemma:reduce MS2 into one value}
Given a differentiated execution $e$, there exists some $e' \in \textit{proj}(e)$, such that $\textit{last}(e') = \textit{MSet}_2$ and $e'$ doe not linearizable w.r.t $\textit{MS}(\textit{MSet}_2)$, if and only if one of the following case holds for some $x \in \mathbb{D}$.
\begin{itemize}
\setlength{\itemsep}{0.5pt}
\item[-] $\textit{delete}(x)$ is in $e$ while $\textit{insert}(x)$ is not in $e$.

\item[-] there is more than one $\textit{delete}(x)$ in $e$ and one $\textit{insert}(x)$ in $e$.

\item[-] $\textit{delete}(x) <_{\textit{hb}} \textit{insert}(x)$ in $e$.
\end{itemize}
\end{restatable}

\begin {proof}

The $\textit{only if}$ is obvious and omitted here.

To prove the $\textit{if}$ direction, we prove its contrapositive. Assume that for each item $x$ of $e$, the following conditions are satisfied:

\begin{itemize}
\setlength{\itemsep}{0.5pt}
\item[-] If $\textit{delete}(x)$ is in $e$, then $\textit{insert}(x)$ is also in $e$.

\item[-] There is at most $\textit{delete}(x)$ in $e$.

\item[-] $\textit{delete}(x)$ does not happen before $\textit{insert}(x)$.
\end{itemize}

Assume by contradiction that there exists some $e' \in \textit{proj}(e)$, such that $\textit{last}(e') = \textit{MSet}_2$ and $e'$ doe not linearizable w.r.t $\textit{MS}(\textit{MSet}_2)$. Since $\textit{last}(e') = \textit{MSet}_2$, there exists $\textit{delete}$ in $e$. Let this $\textit{delete}$ operation be $\textit{delete}(a)$. By assumption, we know that in $e$ there exists one $\textit{insert}(a)$ and one $\textit{delete}(a)$, and $\textit{delete}(a)$ does not happen before $\textit{insert}(a)$. Then we know that $\textit{cal}(\textit{insert},x)$ is before $\textit{ret}(\textit{delete},x)$.

Let $e_{\textit{lp}}$ be generated from $e$ by (1) put the linearization of $\textit{insert}(x)$ just after $\textit{cal}(\textit{insert},x)$, (2) put the linearization of $\textit{delete}(x)$ just before $\textit{ret}(\textit{insert},x)$, and (3) for other operations, put its linearization point at an arbitrary between its call and return actions. Let $l'$ be the projection of $e_{\textit{lp}}$ into method events. It is obvious that $h \sqsubseteq l'$. According to our construction of $e_{\textit{lp}}$, we can see that in $l'$, $\textit{insert}(x)$ is before $\textit{delete}(x)$. Therefore, we can see that $l' \in \textit{MS}(\textit{MSet}_2)$, which contradicts our assumption. This completes the proof of the $\textit{id}$ direction. \qed
\end {proof}

According to Lemma \ref{lemma:reduce MS2 into one value}, to check violation to $\textit{MS}(\textit{MSet}_2)$, we need to consider three cases for some $b \in \mathbb{D}$: (1) there is $\textit{delete}(b)$ but there is not $\textit{insert}(x)$, (2) there is more than one $\textit{delete}(b)$ and one $\textit{insert}(b)$ in $e$, and (3) $\textit{delete}(b) <_{\textit{hb}} \textit{insert}(b)$.

For each such case, we construct a witness automata. We generate witness automata $\mathcal{A}_{\textit{MS}}^1$ for the first case, and it is shown in \figurename~\ref{fig:automata 1 for MS-2 in appendix}. Here $c_1 = \textit{cal}(\textit{insert},a),\textit{ret}(\textit{insert},a)$, $\textit{cal}(\textit{delete},a),\textit{ret}(\textit{delete},a),
\textit{cal}(\textit{delete},\textit{empty}),\textit{ret}(\textit{delete},\textit{empty})$, $c_2 = c_1 + \textit{cal}(\textit{delete},b) + \textit{ret}(\textit{delete},b)$.


\begin{figure}[htbp]
  \centering
  \includegraphics[width=0.3 \textwidth]{figures/PIC_AUTO_MS_1.pdf}
%\vspace{-10pt}
  \caption{Automaton $\mathcal{A}_{\textit{MS}}^1$}
  \label{fig:automata 1 for MS-2 in appendix}
\end{figure}


We generate witness automata $\mathcal{A}_{\textit{MS}}^2$ for the second case, and it is shown in \figurename~\ref{fig:automata 2 for MS-2 in appendix}. Here $c_1 = \textit{cal}(\textit{insert},a),\textit{ret}(\textit{insert},a), \textit{cal}(\textit{delete},a),\textit{ret}(\textit{delete},a),\textit{cal}(\textit{delete},\textit{empty})$, $\textit{ret}(\textit{delete},\textit{empty})$, $c_2 = c_1 + \textit{ret}(\textit{insert},b)$, $c_3 = c_2 + \textit{ret}(\textit{delete},b)$, $c_4 = c_3 + \textit{cal}(\textit{delete},b)$, $c_5 = c_1 + \textit{ret}(\textit{delete},b)$, $c_6 = c_5 + \textit{cal}(\textit{delete},b)$.

\begin{figure}[htbp]
  \centering
  \includegraphics[width=0.7 \textwidth]{figures/PIC_AUTO_MS_2.pdf}
%\vspace{-10pt}
  \caption{Automaton $\mathcal{A}_{\textit{MS}}^2$}
  \label{fig:automata 2 for MS-2 in appendix}
\end{figure}


We generate witness automata $\mathcal{A}_{\textit{MS}}^3$ for the third case, and it is shown in \figurename~\ref{fig:automata 3 for MS-2 in appendix}. Here $c_1 = \textit{cal}(\textit{insert},a)$, $\textit{ret}(\textit{insert},a)$, $\textit{cal}(\textit{delete},a)$, $\textit{ret}(\textit{delete},a),\textit{cal}(\textit{delete},b)$, $\textit{cal}($ $\textit{delete},\textit{empty}),\textit{ret}(\textit{delete},\textit{empty})$, $c_2 = c_1 + \textit{ret}(\textit{delete},b)$, $c_3 = c_2 + \textit{ret}(\textit{insert},b)$.

\begin{figure}[htbp]
  \centering
  \includegraphics[width=0.6 \textwidth]{figures/PIC_AUTO_MS_3.pdf}
%\vspace{-10pt}
  \caption{Automaton $\mathcal{A}_{\textit{MS}}^3$}
  \label{fig:automata 3 for MS-2 in appendix}
\end{figure}


Let $\textit{Auts}_{\textit{2-ms}} = \{ \mathcal{A}_{\textit{MS}}^1, \mathcal{A}_{\textit{MS}}^2, \mathcal{A}_{\textit{MS}}^3 \}$. The following lemma states that $\textit{MS}_2$ is co-regular.


\begin{restatable}{lemma}{MS2IsCoRegular}
\label{lemma:MS2 is co-regular}
$\textit{MSet}_2$ is co-regular.
\end{restatable}

\begin {proof}

According to \cite{Bouajjani:2015}, we need to prove that, given a independence implementation $\mathcal{I}$, $\textit{Auts}_{\textit{2-ms}} \cap \mathcal{I} \neq \emptyset$, if and only if $\exists e \in \mathcal{I}_{\neq},$ $e' \in \textit{proj}(e),$ $\textit{last}(e') = \textit{MSet}_2 \wedge e'$ does not linearizable w.r.t. $\textit{MS}(\textit{MSet}_2)$.

By Lemma \ref{lemma:reduce MS2 into one value}, we need to prove the following fact:

\noindent {\bf $\textit{fact}_1$}: given a independence implementation $\mathcal{I}$, $\textit{Auts}_{\textit{2-ms}} \cap \mathcal{I} \neq \emptyset$, if and only if $\exists e \in \mathcal{I}_{\neq}$, and one of the following case holds for some $x \in \mathbb{D}$.
\begin{itemize}
\setlength{\itemsep}{0.5pt}
\item[-] $\textit{delete}(x)$ is in $e$ while $\textit{insert}(x)$ is not in $e$.

\item[-] there is more than one $\textit{delete}(x)$ in $e$ and one $\textit{insert}(x)$ in $e$.

\item[-] $\textit{delete}(x) <_{\textit{hb}} \textit{insert}(x)$ in $e$.
\end{itemize}

\noindent The $\textit{only if}$ direction: Assume that $e_1 \in \mathcal{I}$ is accepted by some witness automata in $\textit{Auts}_{\textit{1-eq}}$. By data-independence, there exists data-differentiated execution $e_2 \in \mathcal{I}$ and a renaming function $r$, such that $e_1=r(e_2)$. Since $e_1$ is accepted by some witness automata in  $\textit{Auts}_{\textit{1-eq}}$, let $y$ be the item that are renamed into $b$ by $r$. Then it is not hard to see that $y$ satisfies one of three conditions in $e_2$.

\noindent The $\textit{if}$ direction: Assume that there exists such $e \in \mathcal{I}_{\neq}$ and $x$. Let renaming function $r$ maps $x$ into $b$ and all other items into $a$. By data-independence, we can see that $r(e) \in \mathcal{I}$. Then it is easy to see that $r(e)$ is accepted by some automaton in $\textit{Auts}_{\textit{2-ms}}$. \qed
\end {proof}

Similar to Lemma \ref{lemma:EPQ3 is co-regular}, we can prove that $\textit{MSet}_3$ is co-regular, as stated by the following lemma.

\begin{restatable}{lemma}{MSet3IsCoRegular}
\label{lemma:MSet3 is co-regular}
$\textit{MSet}_3$ is co-regular.
\end{restatable}

Similar as witness automata for $\textit{EPQ}_3$, we have witness automata $\mathcal{A}_{\textit{MS}}^4$ for $\textit{MSet}_3$, which is shown in \figurename~\ref{fig:automata 4 for MS-3 in appendix}. In \figurename~\ref{fig:automata 4 for MS-3 in appendix}, let $c = \textit{cal}(\textit{insert},d),\textit{ret}(\textit{insert},d), \textit{cal}(\textit{delete},d)$, $\textit{ret}(\textit{delete},d),\textit{cal}(\textit{delete},\textit{empty}),\textit{ret}(\textit{delete},\textit{empty})$, $c_1 = c + \textit{cal}(\textit{insert},b)$, $c_2 = c_1 + \textit{ret}(\textit{delete},b)$, and $c_3 = c + \textit{ret}(\textit{delete},b)$.

\begin{figure}[htbp]
  \centering
  \includegraphics[width=0.8 \textwidth]{figures/PIC_AUTO_MS_4.pdf}
%\vspace{-10pt}
  \caption{Automaton $\mathcal{A}_{\textit{MS}}^4$}
  \label{fig:automata 4 for MS-3 in appendix}
\end{figure}

The following lemma shows that our transformation from multi-set into extended priority queue is correct.

\RelateMultiSetwithEPQ*

\begin {proof}

For the $\textit{only if}$ direction, given an execution $e_m \in \mathcal{I}_m$ of multi-set, such that $e_m$ is accepted by some automaton in $\textit{Auts}_{\textit{MS}}$. If $e_m$ is accepted by $\mathcal{A}_{\textit{MS}}^1$, $\mathcal{A}_{\textit{MS}}^2$, $\mathcal{A}_{\textit{MS}}^3$ or $\mathcal{A}_{\textit{MS}}^4$, then it is easy to see that $\textit{MStoEPQ}(e_m)$ is accepted by $\mathcal{A}_{\textit{SinPri}}^2$, $\mathcal{A}_{\textit{SinPri}}^3$, $\mathcal{A}_{\textit{SinPri}}^1$ or $\mathcal{A}_{\textit{EPQ}}^3$, respectively.

For the $\textit{if}$ direction, given an execution $e_{\textit{epq}}$ of extended priority queue, such that $e_{\textit{epq}} = \textit{MStoEPQ}(e_m)$ for some $e_m$ of multi-set, and $e_{\textit{epq}}$ is accepted by some automaton in $\textit{Auts}_{\textit{EPQ}}$.

According to definition of $\textit{MStoEPQ}$, we can see that (1) in $e_{\textit{epq}}$ there does not exist two items with comparable priorities, and (2) in $e_{\textit{epq}}$, there does not exists two items with a same priority. Therefore, the witness automata for $\textit{EPQ}_1$ and the witness automaton $\mathcal{A}_{\textit{SinPri}}^4$ can be safely ignored. Then, if $e_{\textit{epq}}$ is accepted by $\mathcal{A}_{\textit{SinPri}}^1$, $\mathcal{A}_{\textit{SinPri}}^2$, $\mathcal{A}_{\textit{SinPri}}^3$ or $\mathcal{A}_{\textit{EPQ}}^3$, then it is easy to see that $e_m$ is accepted by $\mathcal{A}_{\textit{MS}}^3$, $\mathcal{A}_{\textit{MS}}^1$, $\mathcal{A}_{\textit{MS}}^2$ and $\mathcal{A}_{\textit{MS}}^4$, respectively. \qed
\end {proof}



\subsection{Proofs and Definitions in Subsection \ref{subsec:relate stack with extended priority queue}}
\label{subsec:appendix proof and definition in section relate stack with extended priority queue}













\end{document}








